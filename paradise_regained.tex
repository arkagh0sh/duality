% !TEX root = main.tex
%
%
\section{Column-finite and row-finite linear programs}\label{sec:regained}
%
%In the previous section we saw that weak duality in linear programming extend to the orbit-finite setting,
%and strong duality does not.
%In this section we define two natural variations of orbit-finite linear programming,
%respectively called column-finite and row-finite linear programming and prove duality between them (\Cref{thm:duality}).
%The techniques developed to show duality also allow us to prove that these two kinds of orbit-finite linear programs are solvable in fixed atom-dimension \ptime (\Cref{thm:col fin ptime,thm:row fin ptime}).
%This subsection is organised as follows.
%We start with the definitions of column-finite and row-finite linear programming and state \Cref{thm:duality} which shows that they are dual to each other.
%Then in \Cref{subsec:orbsum} we develop the theory using which we prove \Cref{thm:duality} in \Cref{subsec:duality proof}.
%Finally, in \Cref{subsec:col row fin ptime} we show that both column-finite and row-finite linear programs are solvable in fixed-atom dimension \ptime.
%
%\begin{definition}\label{def:row col fin}
%A matrix $\vr{A}\in\lin{B{\times} C}$ is called \emph'def{column-finite} if its column-vectors are finite,
%i.e.\ $\vr{A}(-,c)\in\flin{B}$ for every $c\in C$.
%The matrix $\vr{A}\in\lin{B{\times} C}$ is \emph'def{row-finite} if $\transpose{\vr{A}}$ is column-finite.
%\end{definition}
%%
%As opposed to general orbit-finite matrices,
%well-definedness is not an issue for products of column-finite or row-finite matrices.
%%
%\begin{lemma}\label{lem:row fin prod}
%%Let $D$ be an arbitrary orbit-finite set.
%%Let $\vr{A}\in\lin{B{\times} C}$ and $\vr{B} : (C{\times} D)\to\R$.
%%If $\vr{A}$ is row-finite then the product $\vr{A}\cdot\vr{B}$ is well-defined,
%%and if $\vr{B}$ is also row-finite then $\vr{A}\cdot\vr{B}$ is also row-finite.
%For matrices $\vr{A}\in\lin{B{\times} C}$ and $\vr{B} : (C{\times} D)\to\R$.
%If $\vr{A}$ is row-finite then the product $\vr{A}\cdot\vr{B}$ is well-defined,
%and if $\vr{B}$ is also row-finite then $\vr{A}\cdot\vr{B}$ is also row-finite.
%\end{lemma}
%%
%\begin{proof}[Proof of \Cref{lem:row fin prod}]
%First we prove $\vr{A}\cdot\vr{B}$ is well-defined.
%Pick arbitrary ${(b,d)\in (B{\times} D)}$.
%Then $(\vr{A}\cdot\vr{B})(b,d) = \vr{A}(b,-)\cdot\vr{B}(-,d)$.
%Since $\vr{A}$ is row-finite, then $\transpose{(\vr{A}(b,-))}$ is finite.
%By \Cref{lem:fin gen prod} $\vr{A}(b,-)\cdot\vr{B}(-,d)$ is well-defined.
%Since ${(b,d)\in (B{\times} D)}$ was arbitrarily chosen,
%this shows $\vr{A}\cdot\vr{B}$ is well defined.
%
%Now we show $\vr{A}\cdot\vr{B}$ is row-finite assuming $\vr{B}$ is row-finite.
%Pick arbitrary $b\in B$.
%We show $(\vr{A}\cdot\vr{B})(b,-)$ is finite.
%Let $C'\subseteq C$ be the (necessarily finite) subset of elements of $c\in C$ such that $\vr{A}(b,c)\neq 0$.
%\[
%C' = \setof{c\in C}{\vr{A}(b,c)\neq 0}
%\]
%For every $c\in C'$ let $D_c$ be the subset of elements of $d\in D$ such that $\vr{B}(c,d)\neq 0$.
%Since $\vr{B}$ is row-finite, $D_c$ is finite for every $c\in C'$.
%Let $D' = \cup_{c\in C'} D_c$.
%Since $C'$ is finite and $D_c$ is finite for every $c\in C'$,
%the set $D'$ is also finite.
%We claim $(\vr{A}\cdot\vr{B})(b,d) \neq 0$ only if $d\in D'$.
%Pick arbitrary $d\in (D\setminus D')$.
%For $d\in (D\setminus D')$ indeed since $\vr{A}(b,c) \neq 0$ only if $c\in C'$,
%we have
%\[
%(\vr{A}\cdot\vr{B})(b,d) =
%\sum_{c\in C} \vr{A}(b,c)\cdot\vr{B}(c,d) =
%\sum_{c\in C'} \vr{A}(b,c)\cdot\vr{B}(c,d)\ ,
%\]
%and since $\vr{B}(c,d)\neq 0$ only if $d\in D'$ (assuming $c\in C'$) we have,
%\[
%\sum_{c\in C'} \vr{A}(b,c)\cdot\vr{B}(c,d) = 0\ .
%\]
%Combining the above equations we get
%\[
%(\vr{A}\cdot\vr{B})(b,d) = 0\ .
%\]
%Thus $(\vr{A}\cdot\vr{B})(b,d) \neq 0$ only if $d\in D'$,
%which implies finiteness of \\ $(\vr{A}\cdot\vr{B})(b,-)$.
%\end{proof}
%%
%
%
In this and the following sections, we study orbit-finite linear programs that are either column-finite or row-finite.
That is,
they are of the form
\[
\primalAbc\qquad\qquad\dualAbc
\]
where $B$ and $C$ are orbit-finite sets,
the matrix $\vr{A}\in\lin{B{\times}C}$ is column-finite,
and the vector $\vr{b}\in\flin{B}$.
Note that we wrote the column-finite linear program in primal form and the row-finite linear program in the dual form.
We follow this convention for the rest of the chapter.

These linear programs are better behaved than orbit-finite linear programs.
Firstly, they satisfy strong duality.
%
\begin{theorem}[Strong Duality]\label{thm:col row duality}
For any primal-dual pair of orbit-finite linear programs, if the primal or the dual is column-finite and the optimum of any of linear programs is finite,
then it is equal to the optimum of the other.
\end{theorem}
%
Secondly, column-finite and row-finite linear programs are more robust compared to orbit-finite linear programs in general,
and they admit optimal solutions with small support when their optimum is finite
(\Cref{thm:col fin opt,thm:row fin opt} below).
We need a few definitions to state these theorems.
%
%\noindent
%Secondly, they admit optimal solutions whenever their optimum is finite.
%
%\begin{theorem}\label{thm:col fin small supp}
%The optimum of an $S$-supported column-finite linear program of atom dimension $d$ does not change if we restrict the solution to be finite and supported by $S\cup T$ where $T\subseteqfin(\A\setminus S)$ is any subset of size at least $d$.
%\end{theorem}
%%
%\begin{theorem}\label{thm:col fin opt}
%If the optimum of an $S$-supported column-finite linear program of atom dimension $d$ is finite then,
%for any $T\subseteqfin(\A\setminus S)$ of size at least $d$
%it has a finite optimal solution supported by $S\cup T$.
%\end{theorem}
%
\newcommand{\norm}[1]{\lVert #1 \rVert_1}
\newcommand{\normText}[1]{$\ell^1$-norm}
\newcommand{\normSp}[1]{\ell^1(#1)}
%
\begin{definition}
Let $P$ be an orbit-finite set.
For a vector $\vr{x} : P\to\R$, by $\norm{\vr{x}}$ we denote its \defindKey{\normText{}}{$l1norm$},
i.e.\
\[
\norm{\vr{x}} \defeq \sum_{p\in P} |\vr{x}(p)| \ ,
\]
where $|\vr{x}(p)|$ denotes the absolute value of $\vr{x}(p)$.
We write \emphdef{$\norm{\vr{x}} < \infty$} to denote that the sum is finite,
i.e., converges to a real number.
Otherwise we write \emphdef{$\norm{\vr{x}} = \infty$}.

Vectors $\vr{x} : P\to\R$ such that $\norm{\vr{x}} < \infty$ forms a vector space (we leave it to the reader to verify the details),
which we denote by \defindKey{$\normSp{P}$}{$l1set$}.
\end{definition}
%
\begin{remark}
For every $\vr{v}\in\lin{P}$,
the set $\setof{\vr{v}(p)}{p\in P}$ of its coefficients is finite,
and hence $\norm{\vr{v}} < \infty$ if and only if $\vr{v}\in\flin{P}$.
\end{remark}
%
\begin{example}\label{eg:l1 vec}
For any enumeration $\a_1,\a_2,\dots$ of the atoms,
the vector $\vr{y} : \otuequiv{2}\to\R$ defined as
\[
\vr{y}(\a\b) =
\begin{cases}
\frac{1}{2^n} & \text{if $\a = \a_n$ and $\b = \a_{n+1}$ for some $n\in\N$, and}\\
0 & \text{otherwise}
\end{cases}
\]
is a vector in $\normSp{\otuequiv{2}}$ which is not in $\flin{\otuequiv{2}}$.
\end{example}
%
\begin{lemma}\label{lem:lin prod norm}
For every orbit-finite set $P$ and vectors $\vr{x}\in\normSp{P}$,
$\vr{y}\in\lin{P}$ the vector
\[
p \mapsto \vr{x}(p)\cdot\vr{y}(p) \ :\  P\to \R
\]
is also in $\normSp{P}$.
\end{lemma}
%
\begin{proof}
Let $R = \max(\setof{|\vr{y}(p)|}{p\in P})$.
Since $\vr{y}\in\lin{P}$, $R$ is finite.
Then
\[
\sum_{p \in P} |\vr{x}(p)\cdot\vr{y}(p)| 
\leqslant R\cdot\sum_{p \in P} |\vr{x}(p)| < \infty \ .
\]
\end{proof}
%
\begin{theorem}\label{thm:col fin opt}
Consider an $S$-supported column-finite linear program of atom dimension $d$.
For any $T\subseteqfin(\A\setminus S)$ of size at least $d$:
\begin{enumerate}
\item The optimum of the linear program does not change if we restrict the solutions to be finite and supported by $S\cup T$,
or allow them to be orbit-infinite but of finite \normText{}.
%\arka{Updated}
\item If the optimum is finite then it has an optimal solution which is finite and supported by ${S\cup T}$.
\end{enumerate}
\end{theorem}
%
%\begin{theorem}\label{thm:row fin supp}
%The optimum of an $S$-supported row-finite linear program does not change if we restrict to solutions supported by $S$.
%\end{theorem}
%%
%%
%\begin{theorem}\label{thm:row fin opt}
%If the optimum of an $S$-supported row-finite linear program is finite then,
%it has a finite optimal solution supported by $S$.
%\end{theorem}
%%
%\begin{theorem}\label{thm:row fin arb sol}
%The optimum of a row-finite linear program does not change even if we allow the solutions to be orbit-infinite.
%\end{theorem}
%%
%
\begin{theorem}\label{thm:row fin opt}
For any $S$-supported row-finite linear program:
\begin{enumerate}
\item Its optimum does not change if we restrict the solutions to be supported by $S$, or allow them to be orbit-infinite.
\item If its optimum is finite then it has an optimal solution supported by $S$.
\end{enumerate}
\end{theorem}
%
Compare the above theorems with \Cref{ex:no-sol},
which shows that even solvability of orbit-finite linear programs can change if we allow orbit-infinite solutions,
and with \Cref{ex:lpmin} which shows orbit-finite linear programs may not have optimal solutions when their optimum is finite.
%
%%
%\begin{remark}
%The class of row-finite linear programs is a subclass of orbit-finite linear programs.
%The class of Column-finite linear programs is a subclass of finitary variants of orbit-finite linear programs,
%which itself can be thought as a subclass of orbit-finite linear programs
%\arka{add and cite remark}.
%\end{remark}
%
\begin{example}
%
We give an example of a column-finite system with a non-trivial orbit-infinite solution with bounded \normText{}.

Pick $\a\in\A$,
Consider the system
%
\begin{equation}\label{eq:l1 sys}
\lpMax{$\transpose{\idvec{\A}}\cdot\vr{x}$}{
$\vr{A}\cdot\vr{x} \leqslant \idvec{\a}$ \\
& $\vr{x} \geqslant \vr{0}$ \\
& $\vr{x} \in \lin{\otuequiv{2}}$
}
\end{equation}
%
where columns of the matrices are defined (using \Cref{notation:fin vec}) as
\[
\vr{A}(-,\a\b) = \a - \frac{1}{2}\cdot \b  \qquad (\a\b\in\otuequiv{2}) \ .
\]
The vector $\vr{y}$ defined in \Cref{eg:l1 vec} is a solution of the above system assuming the enumeration $\a_1,\a_2,\dots$ defining $\vr{y}$ starts from $\a_1 = \a$.
For any two atoms $\b,\g\neq \a$,
the vector $\vr{z} = \a\b + \a\g + \frac{1}{2}\cdot(\b\g + \g\b)$ is a finite solution of \eqref{eq:l1 sys} with the same value as $\vr{y}$
\[
\transpose{\idvec{\A}}\cdot\vr{y} = \transpose{\idvec{\A}}\cdot\vr{z} = 2 \ .
\] 
%
\end{example}
%
\begin{remark}
Column-finite linear programs are slightly less robust than row-finite linear programs.
The optimum of a row-finite linear program does not change if we allow its solutions to be orbit-infinite.
The optimum of a column-finite linear program does not change if we allow its solutions to be orbit-infinite of bounded \normText{}.
However, even solvability of a column-finite linear program may change if we allow orbit-infinite solutions without the bounded \normText{} restriction.

Recall the system defined in \Cref{ex:Kirchoff}.
We leave it to the reader to check that this system is column-finite.
We have shown that this system has no finite solutions.
\Cref{thm:col fin opt} implies it has neither any orbit-finite solution nor any orbit-infinite solution of bounded \normText{}.
However, as shown already, it has an orbit-infinite solution.
Note that the \normText{} of the solution of this system given in \Cref{ex:Kirchoff} is $\infty$.
\end{remark}
%
\section{Proof of strong duality}\label{sec:strong proof}
%
In this section we prove \Cref{thm:col row duality,thm:col fin opt,thm:row fin opt}.
We start by defining five functions and state one \emph{main lemma} for each of them.
The proofs of above theorems follow almost immediately from these lemmas.
The proofs of the lemmas appear in later sections.

\fixed{For the remainder of the chapter,
arbitrarily fix $S\subseteqfin\A$, orbit-finite sets $B$ and $C$ supported by $S$ and a column-finite maximisation problem $\cU$:
\begin{equation}\label{eq:col fin primal} 
\primalAbc
\end{equation}
\noindent
where $\vr{A}\in\lin{B{\times}C}$, $\vr{b}\in\flin{B}$ and $\vr{c}\in\lin{C}$,
and all of them are supported by $S$.
Fix $d\in\N$ to be the atom-dimension of $\cU$.
Fix an arbitrary $T\subseteq(\A\setminus S)$ of size at least $d$.}
%
\begin{definition}\label{def:orbres}
For an $S$-supported orbit-finite set $P$ and finite $K\supseteq S$,
define the \defindS{orbit restriction function} $\defindKey{$\orbres_K$}{$zeta$} : \lin{P}\to\flin{P}$ as
\[
(\orbres_K(\vr{x}))(p) =
\begin{cases}
\vr{x}(p), & \text{if }\supp{p}\subseteq K, \\
0,         & \text{otherwise.}
\end{cases}
\]
We write $\orbres(\vr{x})$ to denote $\orbres_{S\cup\supp{\vr{x}}}(\vr{x})$.
\end{definition}
%
For finite $K\supseteq S$,
the co-domain of $\orbres_K$ is indeed $\flin{P}$ since \ref{lem:supp count} implies that for every vector $\vr{x}\in\lin{P}$, the vector $\orbres_K(\vr{x})$ is finite.
%
\begin{lemma}\label{lem:orbres sol}
For every orbit-finite solution $\vr{x}$ of $\cU$,
$\orbres(\vr{x})$ is a finite solution of $\cU$
and $\transpose{\vr{c}}\cdot\orbres(\vr{x}) = \transpose{\vr{c}}\cdot\vr{x}$.
\end{lemma}
%
The above lemma is proven in \Cref{sec:orbres}.
%
\begin{definition}\label{def:orbsum vector}
For an orbit-finite set $P$ supported by $S$, 
define the \defindS{orbit summation function}
\[
\defindKey{$\orbsum{S}$}{$gammaS$} : \ell_1(P) \to \R^{\orbits[S]{P}}
\]
as
\[
\orbsum{S}(\vr{v}) : (K \in\orbits[S]{P}) \mapsto  \sum_{b\in K} \vr{v}(b)
\]
\end{definition}
%
\begin{remark}\label{rem:sum L1}
Theorem 3.55 in \cite{babyRudin} implies that $\orbsum{S}(\vr{x})$ is well-defined for vectors $\vr{x}$ in $\normSp{P}$:
for any enumeration $p_1,p_2,\dots$ of elements of $P$, the sequence of \emph{partial sums}
\[
\sum_{i=1}^n \vr{x}(p_i)
\]
converges to a real number, which is independent of the enumeration.
\end{remark}
%
\begin{remark}
The function $\orbsum{}$ (\Cref{def:orbsum linprog}) which was useful in solving systems of inequalities is a special case of $\orbsum{S}$ with $S=\emptyset$.
A similar definition (\Cref{def:orbsum ord}) was also used in solving linear equations.
\end{remark}
%
The orbit summation function $\orbsum{S}$ is used to convert column-finite and row-finite linear programs into finite linear programs.
To do this, we extend it to matrices.
%
\begin{definition}\label{def:orbsum matrix}
For orbit-finite sets $P$ and $Q$ and column-finite matrix $\vr{B}$ in $\lin{P{\times}Q}$,
all of them supported by $S$,
define $\orbSum{S}(\vr{B})$ to be the\\
${\orbits[S]{P}{{\times}}\orbits[S]{Q}}$-matrix with columns
\defind{}{$gamma$@$\orbSum{S}$}
\[
\emphdef{$(\orbSum{S}(\vr{B}))(-,K)$} \defeq \orbsum{S}(\vr{B}(-,q))
\text{, for some (every) } q\in K \ ,
\]
for $K\in \orbits[S]{Q}$.
\end{definition}
%
In \Cref{sec:orbsum} we prove $\orbSum{S}$ is well defined for \mbox{$S$-supported} matrices (\Cref{lem:orbsum matrix well def}),
and it commutes with matrix multiplication (\Cref{cor:orbsum mat prod}).
%
Using $\orbSum{S}$ we get from $\cU$ the linear program \fixed{$\orbSum{S}(\cU)$:
%
\begin{equation}\label{eq:finite primal}
\lpMax{$\orbSum{S}(\transpose{\vr{c}})\cdot \vr{x}$}{
  $\orbSum{S}(\vr{A})\cdot \vr{x}\leqslant \orbSum{S}(\vr{b})$ \\
& $\vr{x}\geqslant \vr{0}\ .$}
\end{equation}
%
}

\noindent
Note that $\orbSum{S}(\vr{b})$ is well-defined since $\vr{b}$ is supported by $S$.
Since $\vr{b}$ is a column-vector (i.e.\ a matrix with only one column)
\[
\orbSum{S}(\vr{b}) = \orbsum{S}(\vr{b}) \ .
\]
%
\begin{lemma}\label{lem:orbsum sol}
For any solution $\vr{x}\in\normSp{C}$ of $\cU$,
$\orbsum{S}(\vr{x})$ is a solution of $\orbSum{S}(\cU)$ and
$\orbSum{S}(\transpose{\vr{c}})\cdot\orbsum{S}(\vr{x}) =
\transpose{\vr{c}}\cdot\vr{x}$.
\end{lemma}
%
\begin{example}\label{eg:orbsum duality}
Let $S = \emptyset$.
Let $\star$ be an equivariant element,
i.e.\ $\pi(\star)=\star$ for every $\pi\in\aut{\A}$.
Let $B = \{\star\}\uplus\A$ and $C = \A^2$.
The set $B$ has two equivariant orbits,
namely $\{\star\}$ and $\A$,
and $C$ also has two equivariant orbits,
namely $\A^{(2)} = \setof{\a\b\in C}{\a\neq\b}$
and $I = \setof{\a\a}{\a\in\A}$.
Using \Cref{notation:fin vec}, for every $(\a,\b)\in C$ define $\vr{v}_{\a\b}\in\flin{B}$ as
\[
\vr{v}_{\a\b}\defeq
\begin{cases}
\a + \b + \star & \text{if }\a\neq\b \\
\star - \a& \text{otherwise.}
\end{cases}
\]
For every $(\a,\b)\in C$
we have $\orbsum{S}(\vr{v}_{\a\b})\in\R^{\orbits[S]{B}} = \R^2$ and
\[
\orbsum{S}(\vr{v}_{\a\b}) =
\begin{cases}
(2,1) & \text{if }\a\neq\b \\
(-1,1) & \text{otherwise.}
\end{cases}
\]
(assuming that the first and the second coordinate, respectively,
correspond to the orbits $\A$ and $\{\star\}$).
Define $\vr{A}$ to be the $(B{\times} C)$ matrix with columns $\vr{A}(-,\a\b) = \vr{v}_{\a\b}$ for $c\in C$.
Then $\vr{A}$ is a column-finite matrix.
We have
\[
\begin{tabular}{l l}

& \indexcolor{\phantom{...} $\otuequiv{2}$ $\phantom{.....}$ $I$} \\

$\orbSum{S}(\vr{A})\ \ =$

& $
\begin{bmatrix}
\phantom{a}2\phantom{a} & \phantom{a}-1\phantom{a} \\
\phantom{a}1\phantom{a} & \phantom{a.....}1\phantom{a} \\
\end{bmatrix}
$

\indexcolor{
$
\begin{tabular}{c}
$\A$ \\
$\set{\star}$
\end{tabular}
$
}

\end{tabular}
\]
Define $\vr{b}\in\flin{B}$ and $\vr{c}\in\lin{C}$ as
\[
\vr{b} = \star \quad\text{and}\quad \vr{c} = 2\cdot\idvec{\otuequiv{2}} +\idvec{I}
\ .
\]
Using $\vr{A}$, $\vr{b}$ and $\vr{c}$ we form $\cU$ to be the column-finite linear program
%
\begin{equation}\label{eq:primal eg}
\lpMatPrimal{\vr{A}}{\vr{b}}{\vr{c}}{\vr{x}}{B}{C}
\end{equation}
%
The vector $\vr{b}$ is an equivariant finite vector.
Hence $\orbSum{S}(\vr{b}) = \orbsum{S}(\vr{b}) = (0,1)$.
Since $\vr{c}$ is not a finite vector $\orbsum{S}(\vr{c})$ is not well-defined.
However, following usual convention we consider
$\transpose{\vr{c}}$ to be a row vector (i.e.\ matrix with only one row),
and hence it is automatically column-finite.
Moreover $\transpose{\vr{c}}$ is equivariant.
Hence $\orbSum{S}(\transpose{\vr{c}})$ is well-defined and is equal to
\[
\begin{tabular}{l l}
& \indexcolor{\phantom{..}$\otuequiv{2}$ \quad $I$} \\

$\orbSum{S}(\transpose{\vr{c}})\ \ =$

& $
\begin{bmatrix}
\phantom{.}2\phantom{...} & \phantom{...}1\phantom{.}
\end{bmatrix}
\ .
$
\end{tabular}
\]
%
Using $\orbSum{S}$ we get the finite linear program
\[
\orbSum{S}(\cU) :
\lpMax{$2\cdot x_1 + x_2$}
{   $         2\cdot x_1 - x_2 \leqslant 0$ \\
  & $\phantom{.....}x_1 + x_2 \leqslant 1$}
\]
\end{example}
%
Recall the definition of $d$ and $T$ fixed in the beginning of this section.
Also recall \Cref{notation:fin vec}.
%
\begin{definition}\label{def:orbdis}
For an $S$-supported orbit-finite set $P$ of atom dimension at most $d$,
define the \defindS{semi-orbit distribution function}
\[
\defindKey{$\orbdis{S}{T}$}{$deltaST$} : \R^{\orbits[S]{P}}\to \flin{P}
\]
as
\[
\orbdis{S}{T}(\vr{x}) \defeq
\sum_{K\in\orbits[S]{P}} \left(
\frac{\vr{x}(K)}{|K_{S\cup T}|} \cdot \sum_{p\in K_{S\cup T}} p
\right)
\]
where for $K\in\orbits[S]{P}$
\[
K_{S\cup T} = \setof{p\in P}{\supp{p}\subseteq S\cup T}
\]
\end{definition}
%
%\begin{example}\label{ex:ordis}
%Consider $S=\emptyset$, $B = \otuequiv{2}\cup\A$ and $T \in \binom{\A}{3}$.
%We have $\orbits[S]{B} = \set{\otuequiv{2},\A}$.
%Define $\vr{x} : \orbits[S]{B} \to \R$ as
%\[
%\vr{x}(\otuequiv{2}) = 1 \qquad \vr{x}(\A) = 2
%\]
%Then,
%\[
%\orbdis{S}{T}(\vr{x}) =
%\left(\frac{1}{6} \cdot \sum_{(\a,\b)\in \otufrom{T}{2}} (\a,\b) \right)
%+
%\left(\frac{2}{3} \cdot \sum_{\a\in T} \a\right)
%\]
%\end{example}
%
\begin{lemma}\label{lem:orbdis sol}
For any solution $\vr{x}$ of $\orbSum{S}(\cU)$,
$\orbdis{S}{T}(\vr{x})$ is an $(S\cup T)$-supported finite solution of $\cU$ and
$\transpose{\vr{c}}\cdot\orbdis{S}{T}(\vr{x}) =
 \orbSum{S}(\transpose{\vr{c}})\cdot\vr{x}$.
\end{lemma}
%
\begin{definition}\label{def:equiv lp}
Two orbit-finite linear programs are called \defind{equivalent}{equivalence of linear programs} if:
\begin{enumerate}
  \item they have the same optimum, and
  \item in case one of them has an optimal solution then so does the other. 
\end{enumerate}
\end{definition}
%
An immediate corollary of \Cref{lem:orbsum sol,lem:orbdis sol} is:
%
\begin{corollary}\label{cor:col fin solv}
The linear programs $\cU$ and $\orbSum{S}(\cU)$ are equivalent.
\end{corollary}
%
Now we are ready to prove \Cref{thm:col fin opt} using \Cref{lem:orbsum sol,lem:orbdis sol}.
The proof of these lemmas will appear in respectively \Cref{sec:orbsum} and \Cref{sec:orbdis}.
%
\begin{subsecproof}{Proof of \Cref{thm:col fin opt}}
The optimum of $\cU$ can only decrease if we restrict to finite solutions supported by $S\cup T$.
\Cref{lem:orbres sol,lem:orbsum sol,lem:orbdis sol} together imply that for any orbit-finite solution $\vr{x}$ of $\cU$,
$(\orbdis{S}{T}\circ\orbsum{S})(\orbres(\vr{x}))$ is a finite $(S\cup T)$-supported solution of $\cU$ with
\[
\transpose{\vr{c}}\cdot(\orbdis{S}{T}\circ\orbsum{S})(\orbres(\vr{x}))
= \transpose{\vr{c}}\cdot\vr{x}
\]
Hence the optimum of $\cU$ does not change if we restrict to finite $(S\cup T)$-supported solutions.

\Cref{lem:orbsum sol,lem:orbdis sol} together imply that for any solution $\vr{x}$ of $\cU$ in $\normSp{C}$,
$(\orbdis{S}{T}\circ\orbsum{S})(\vr{x})$ is a finite $(S\cup T)$-supported solution of $\cU$ with
\[
\transpose{\vr{c}}\cdot(\orbdis{S}{T}\circ\orbsum{S})(\vr{x})
= \transpose{\vr{c}}\cdot\vr{x}
\]
Hence the optimum of $\cU$ does not change if we allow orbit-infinite solutions in $\normSp{C}$.


Now assume the optimum of $\cU$ is finite (say $r\in\R$).
\Cref{cor:col fin solv} implies the optimum of $\orbSum{S}(\cU)$ is also $r$.
Finite linear programs admit optimal solutions when their optimums are finite (\cite[Theorem 2.6]{optbook}).
Let $\vr{z}$ be an optimal solution of $\orbSum{S}(\cU)$.
Then $\orbdis{S}{T}(\vr{z})$ is a $(S\cup T)$-supported finite solution of $\cU$ with
\[
\transpose{\vr{c}}\cdot\orbdis{S}{T}(\vr{z}) = r
\]
Since $r$ is the optimum of $\cU$, $\orbdis{S}{T}(\vr{z})$ is also an optimal solution.
\end{subsecproof}
%
\Cref{lem:orbsum sol,lem:orbdis sol}, and the proof of \Cref{thm:col fin opt} using them can be summarised by the following diagram of functions between solution sets of different linear programs.
All of the functions preserve the value of a solution.
\[
\begin{tikzpicture}
\node (cent) {} ;
\node (bot)   [below= of cent] {} ;
\node (top)   [above= of cent] {} ;
\node (ofsol) [above left= of top]
{\begin{tabular}{l}
 orbit-finite \\ solutions of $\cU$
 \end{tabular}};
\node (finsol) [above right= of top]
{\begin{tabular}{l}
 finite solutions \\ of $\cU$
 \end{tabular}};
\node (norm) [right=of cent]
{\begin{tabular}{l}
 possibly orbit-infinite \\
 solutions of $\cU$ of bounded  \\
 \normText{}
 \end{tabular}};
\node (solfin) [below= of bot]
{\begin{tabular}{l}
 solutions of $\orbSum{S}(\cU)$
 \end{tabular}};
\node (suppsol) [left=of cent]
{\begin{tabular}{l}
 $(S\cup T)$-supported \\ finite solutions of $\cU$
 \end{tabular}}; 
\draw[thick,->] (ofsol.east) -- (finsol.west) node[midway,above right] {$\orbres$};
\draw[draw=none] (finsol.south) -- (norm.north) node[midway,sloped] {$\boldsymbol{\subseteq}$};
\draw[thick,->] (norm.south) -- (solfin.east) node[midway,below right] {$\orbsum{S}$};
\draw[thick,->] (solfin.west) -- (suppsol.south) node[midway,below left] {$\orbdis{S}{T}$};
\draw[draw=none] (suppsol.north) -- (ofsol.south) node[midway,sloped] {$\boldsymbol{\subseteq}$};
\end{tikzpicture}
\]
%
%\[
%\xymatrix{
%& \text{orbit-finite solutions of $\cU$} & \\
%& &\text{finite solutions of $\cU$} \\
%\text{finite solutions of $\cU$} & &
%}
%\]
%\[
%\xymatrix{
%\text{orbit-finite solutions of $\cU$}
%\ar@{|->}[r]^{\orbsum{S}}
%& \text{finite solutions of $\cU$}
%\ar@{|->}[d]^{\orbsum{S}}
%& \cU \ar@{|->}[d]^{\orbsum{S}} \\
%& \text{solution of $\orbSum{S}(\cU)$} & \orbsum{S}(\cU)
%}
%\]
%
\fixed{
The dual of the column-finite linear program $\cU$ is the row-finite linear program $\transpose{\cU}$:
%
\begin{equation}\label{eq:row fin dual}
\dualAbc
\end{equation}
%
%Since $\cU$ is an arbitrary column-finite maximisation problem,
%$\transpose{\cU}$ is an arbitrary row-finite minimisation problem.
The dual of the finite linear program $\orbSum{S}(\cU)$ is the linear program $\transpose{\orbSum{S}(\cU)}$:
\begin{equation}\label{eq:finite dual}
\lpMin{$\transpose{\orbSum{S}(\vr{b})}\cdot \vr{y}$}{
  $\transpose{\orbSum{S}(\vr{A})}\cdot \vr{y}\leqslant
  \transpose{\orbSum{S}(\transpose{\vr{c}})}$ \\
& $\vr{y}\geqslant \vr{0}$}
\end{equation}
%
}
%
\begin{definition}\label{def:orbrow}
For an $S$-supported orbit-finite set $P$,
define the \defindS{orbit distribution function}
\[
\defindKey{$\orbrow{S}$}{$deltaS$} : \R^{\orbits[S]{P}}\to\lin{P}
\]
as
\[
\orbrow{S}(\vr{x}) : p \mapsto \vr{x}(\orbit[S]{p})\ . 
\]
\end{definition}
%
\begin{lemma}\label{lem:orbrow sol}
If $\vr{y}$ is a solution of $\transpose{\orbSum{S}(\cU)}$ then,
$\orbrow{S}(\vr{y})$ is a solution of $\transpose{\cU}$ and
\[
\transpose{\vr{b}}\cdot\orbrow{S}(\vr{y}) =
\transpose{\orbSum{S}(\vr{b})}\cdot\vr{y}
\]
\end{lemma}
%
Recall the definition of $d$ and $T$ fixed in the beginning of this section.
%
\begin{definition}\label{def:orbsmt}
For an $S$-supported orbit-finite set $P$ of atom-dimension at least $d$,
define the \defindS{semi-orbit summation function}
\[
\defindKey{$\orbsmt{S}{T}$}{$gammaST$} : (P\to\R)\to(\R^{\orbits[S]{P}})
\]
as
\[
\orbsmt{S}{T}(\vr{v}) :
(K\in\orbits[S]{D})\mapsto
\left(\frac{1}{|K_{S\cup T}|} \cdot \sum_{p\in K_{S\cup T}} \vr{v}(p) \right)
\]
where for $K\in\orbits[S]{D}$ 
\[
K_{S\cup T} = \setof{p\in X}{\text{$p$ is supported by $S\cup T$}} \ .
\]
\end{definition}
%
\begin{lemma}\label{lem:orbsmt sol}
If $\vr{y}$ is a solution of $\transpose{\cU}$ then,
$\orbsmt{S}{T}(\vr{y})$ is a solution of $\transpose{\orbSum{S}(\cU)}$ with
\[
\transpose{\vr{b}}\cdot \vr{y} =
\transpose{\orbSum{S}(\vr{b})}\cdot\orbsmt{S}{T}(\vr{y})
\]
\end{lemma}
%
As an immediate corollary of \Cref{lem:orbrow sol,lem:orbsmt sol} we get:
\begin{corollary}\label{cor:row fin solv}
%
The linear programs $\transpose{\cU}$ and $\transpose{\orbSum{S}(\cU)}$ are equivalent.
\end{corollary}
%
Now we are ready to prove \Cref{thm:row fin opt} using \Cref{lem:orbrow sol,lem:orbsmt sol}.
The proof of these lemmas will appear in respectively in \Cref{sec:orbrow} and \Cref{sec:orbsmt}.
%
\begin{subsecproof}{Proof of \Cref{thm:row fin opt}}
The optimum of $\transpose{\cU}$ can only decrease if we allow the solutions to be orbit-infinite and can only increase if we restrict the solutions to be supported by $S$.
\Cref{lem:orbrow sol,lem:orbsmt sol} together imply that for any solution $\vr{y}$ of $\transpose{\cU}$ (be it orbit-infinite or orbit-finite),
$(\orbrow{S}\circ\orbsmt{S}{T})(\vr{y})$ is an $S$-supported solution of $\transpose{\cU}$ with
\[
\transpose{\vr{b}}\cdot(\orbrow{S}\circ\orbsmt{S}{T})(\vr{y})
= \transpose{\vr{b}}\cdot\vr{y}
\]
Hence the optimum of $\transpose{\cU}$ does not change if either we allow the solutions to be orbit-infinite or we restrict the solutions to be $S$-supported.

Now assume the optimum of $\transpose{\cU}$ is finite (say $r\in\R$).
\Cref{cor:row fin solv} implies the optimum of $\transpose{\orbSum{S}(\cU)}$ is also $r$.
Finite linear programs admit optimal solutions when their optimums are finite.
Let $\vr{z}$ be an optimal solution of $\orbSum{S}(\cU)$.
Then $\orbrow{S}(\vr{z})$ is an $S$-supported finite solution of $\transpose{\cU}$ with
\[
\transpose{\vr{b}}\cdot\orbrow{S}(\vr{z}) = r
\]
Since $r$ is the optimum of $\transpose{\cU}$, $\orbrow{S}(\vr{z})$ is also an optimal solution.
%
%Due to the arbitrariness of the choice $S\subseteqfin\A$, row-finite linear program $\transpose{\cU}$,
%this proves \Cref{thm:row fin opt} for row-finite minimisation problems.
\end{subsecproof}
%
\Cref{lem:orbrow sol,lem:orbsmt sol}, and the proof of \Cref{thm:row fin opt} using them can be summarised by the following diagram of functions between solution sets of different linear programs.
All of the functions preserve the value of a solution.
\[
\begin{tikzpicture}
\node (cent) {} ;
\node (infsol) [right=of cent]
{\begin{tabular}{l}
 (possibly) orbit-infinite\\
 solutions of $\transpose{\cU}$
 \end{tabular}};
\node (solfin) [below=of cent]
{\begin{tabular}{l}
 solutions of $\transpose{\orbSum{S}(\cU)}$
 \end{tabular}};
\node (suppsol) [left=of cent]
{\begin{tabular}{l}
 $S$-supported \\ solutions of $\transpose{\cU}$
 \end{tabular}};
\draw[thick,->] (infsol.south) -- (solfin.east) node[midway,below right] {$\orbsmt{S}{T}$};
\draw[thick,->] (solfin.west) -- (suppsol.south) node[midway,below left] {$\orbrow{S}$};
\draw[draw=none] (suppsol.east) -- (infsol.west) node[midway] {$\boldsymbol{\subseteq}$};
\end{tikzpicture}
\]
%
Finally, we are ready to prove \Cref{thm:col row duality}.
%
\begin{subsecproof}{Proof of \Cref{thm:col row duality}}
%
%We show that if either the primal or dual is column-finite and
%if the optimum of either of them is finite then, they have the same optimum.
%We do the proof for the case when the primal is column-finite.
%The proof for the other case is similar.
Consider the primal-dual pair ($\cU$, $\transpose{\cU}$).
\Cref{cor:col fin solv} says that $\cU$ and $\orbSum{S}(\cU)$ have the same optimum.
Similarly, \Cref{cor:row fin solv} says that $\transpose{\cU}$ and $\transpose{\orbSum{S}(\cU)}$ have the same optimum.
The pair ($\orbSum{S}(\cU)$, $\transpose{\orbSum{S}(\cU)}$) is a primal-dual pair of finite linear programs.
Hence if the optimum of any of the linear programs $\cU$ or $\transpose{\cU}$ is finite then using the classical duality theorem we can conclude that the it is equal to the optimum of the other.
%Since the pair $\cU$-$\transpose{\cU}$ is an arbitrary primal-dual pair of orbit-finite linear program where the primal is column-finite, this finishes the proof of the theorem for the case when the primal is column-finite.
%The proof of the other case is similar and so we skip it.
\end{subsecproof}
%
\begin{example}\label{eg:orbsum duality cont}
%
%\begin{equation}\label{eq:finite primal eg}
%\lpMax{$\orbSum{S}(\transpose{\vr{c}})\cdot \vr{x}$}{
%  $\orbSum{S}(\vr{A}) \cdot \vr{x} \leqslant \orbSum{S}(\vr{b})$ \\
%& $\vr{x} \geqslant \vr{0}$
%}
%\end{equation}
%%
%Call this $\orbSum{S}(\cU)$.
%
Continuing \Cref{eg:orbsum duality},
fix some $\a_1\b_1\tau_1\in\otuequiv{3}$.
Define $\vr{x}\in\flin{B}$ as
\[
\vr{x} = \frac{1}{9}\cdot\left(2\cdot\a_1\b_1 + \b_1\tau_1
 + 2\cdot\a_1\a_1 + 3\cdot\b_1\b_1 + \tau_1\tau_1 \right)\ .
\]
Then $\vr{x}$ is a non-negative finite solution with value $\frac{4}{3}$, since
\[
\vr{A}\cdot\vr{x} = \star = \vr{b},
\quad\text{and}\quad
\transpose{\vr{c}}\cdot\vr{x} = \frac{4}{3} \ .
\]
We have $\orbsum{S}(\vr{x}) = \left(\frac{1}{3},\frac{2}{3}\right)$.
Since
\[
\orbSum{S}(\vr{A})\cdot\orbsum{S}(\vr{x}) = \left(1,0\right) = \orbSum{S}(\vr{b}),
\quad\text{and}\quad
\orbSum{S}(\transpose{\vr{c}})\cdot\orbsum{S}(\vr{x}) = \frac{4}{3}
\]
the vector $\orbsum{S}(\vr{x})$ is a non-negative solution of $\orbSum{S}(\cU)$ with value $\frac{4}{3}$.
Coincidentally it is also an optimal solution,
and in this case the unique one.
The atom dimension of $\cU$ is $2$.
Let $T = \set{\a_1,\b_1}$.
Then
\[
\orbdis{S}{T}(\orbsum{S}(\vr{x})) =
\frac{1}{6}\cdot\left(\a_1\b_1 + \b_1\a_1\right) + 
\frac{1}{3}\cdot\left(\a_1\a_1 + \b_1\b_1\right)
\]
The vector $\orbdis{S}{T}(\orbsum{S}(\vr{x}))$ is non-negative.
Furthermore, $\vr{A}\cdot\orbdis{S}{T}(\orbsum{S}(\vr{x})) = \star = \vr{b}$, and ${\transpose{\vr{c}}\cdot\orbdis{S}{T}(\orbsum{S}(\vr{x})) = \frac{4}{3}}$.
Hence $\orbdis{S}{T}(\orbsum{S}(\vr{x}))$ is a solution of $\cU$ and in this case an optimal one.
Notice that $\orbdis{S}{T}(\orbsum{S}(\vr{x}))$ uses fewer atoms that $\vr{x}$.

%The dual of the column-finite linear program \eqref{eq:primal eg} is the row-finite linear program
%\begin{equation}\label{eq:dual eg}
%\lpMinPrimal{\transpose{(\vr{A})}}{\vr{c}'}{(\vr{b})}{\vr{y}}{C}{B}
%\end{equation}
%And the dual of the finite linear program \eqref{eq:finite primal eg} is the linear program
%\begin{equation}\label{eq:finite dual eg}
%%\lpMinPrimal{\transpose{\orbSum{S}(\vr{A})}}{\orbSum{S}(\vr{b})}{\orbSum{S}\vr{c}}{\vr{y}}{C}{B}
%\lpMin{$\transpose{\orbSum{S}(\vr{b}')}\cdot \vr{y}$}{
%  $\transpose{\orbSum{S}(\vr{A})}\cdot \vr{y} \geqslant \transpose{\orbSum{S}(\transpose{\vr{c}})}$ \\
%  & $\vr{y} \geqslant \vr{0}$
%} 
%\end{equation}

Pick an infinite subset $W\subseteq\A$.
Define $\vr{y}_W = 3\cdot\star + \idvec{W} + 2\cdot\idvec{(\A\setminus W)}$.
The vector $\vr{y}_W$ is an orbit-infinite solution of $\transpose{\cU}$ with value $3$. 
The result $\orbsmt{S}{T}(\vr{y}_W)$ of applying $\orbsmt{S}{T}$ to $\vr{y}_W$ depends on the intersection $\set{\a,\b}\cap W$.
We focus on the case where $\set{\a,\b}\cap W = \set{\a}$,
the remaining cases can be dealt with similarly.
In this case, $\orbsmt{S}{T}(\vr{y}_W) = (\frac{3}{2},3)$ (assuming that the first and the second co-ordinate, respectively,
corresponds to the orbits $\A$ and $\set{\star}$).
Dualising $\orbSum{S}(\cU)$ we get
\[
\transpose{\orbSum{S}(\cU)} :
\lpMin{$y_2$}
{   $2\cdot y_1 + y_2 \geqslant 2$ \\
  & $\ \, -y_1 + y_2 \geqslant 1$}
\]
The vector $\orbsmt{S}{T}(\vr{y}_W)$ is a solution with of $\transpose{\orbSum{S}(\cU)}$ with value $3$.
Applying $\orbrow{S}$ to $\orbsmt{S}{T}(\vr{y}_W)$ we get
\[
\orbrow{S}(\orbsmt{S}{T}(\vr{y}_W)) = 3\cdot\star + \frac{3}{2}\cdot\idvec{\A}
\] 
which is an equivariant solution of $\transpose{\cU}$ with value $3$.
Note that although $\vr{y}_W$ is orbit-infinite,
but $\orbrow{S}(\orbsmt{S}{T}(\vr{y}_W))$ is equivariant irrespective of $W$.
\end{example}
%
It now remains to demonstrate \Cref{lem:orbres sol,lem:orbsum sol,lem:orbdis sol,lem:orbrow sol,lem:orbsmt sol}.
Which we do in the following \Cref{sec:orbres,sec:orbsum,sec:orbdis,sec:orbrow,sec:orbsmt}.
%
\section{The orbit restriction function}\label{sec:orbres}
%
\begin{lemma}\label{lem:orbres lin}
For any finite $K\supseteq S$,
$\orbres_K$ is a monotonic linear function.
\end{lemma}
%
\begin{proof}
Easy.
\end{proof}
%
\begin{lemma}\label{lem:orbres supp}
For every finite $K\supseteq S$ and vector $\vr{x}$, the vector $\orbres_K(\vr{x})$ is a finite vector supported by $K$.
\end{lemma}
%
\begin{proof}
Follows from the definition of $\orbres$ (\Cref{def:orbres}) and \Cref{lem:fin vec supp}.
\end{proof}
%
\begin{lemma}\label{lem:orbres fin vec}
For every finite vector $\vr{x}$, we have $\orbres(\vr{x}) = \vr{x}$.
\end{lemma}
%
\begin{proof}
Follows from the definition of $\orbres$ (\Cref{def:orbres}) and \Cref{lem:fin vec supp}.
\end{proof}
%
\begin{lemma}\label{lem:orbres mat vec}
For every $K\supseteqfin S$,
and every $\vr{x}\in\lin{C}$ supported by $K$ such that both $\transpose{\vr{c}}\cdot\vr{x}$ and $\vr{A}\cdot\vr{x}$ are well-defined,
we have
\[
\transpose{\vr{c}}\cdot\vr{x} = \transpose{\vr{c}}\cdot\orbres_K(\vr{x})
\quad\text{and}\quad
\orbres_K(\vr{A}\cdot\vr{x}) = \vr{A}\cdot\orbres_K(\vr{x}) \ .
\]
\end{lemma}
%
\begin{proof}
First we show $\transpose{\vr{c}}\cdot\vr{x} = \transpose{\vr{c}}\cdot\orbres_K(\vr{x})$.
Since $S\subseteq K$, $\vr{c}$ is supported by $K$.
Now using \Cref{lem:inner product fin} we get
\begin{equation}\label{eq:orbres 1}
\transpose{\vr{c}}\cdot\vr{x} =
\sum_{c\in C_K}\vr{c}(c)\cdot\vr{x}(c) \ ,
\end{equation}
where $C_K = \setof{c\in C}{\supp{c}\subseteq K}$.
By definition of $\orbres_K$ (\Cref{def:orbres}),
we have $\orbres_K(\vr{x})(c) = \vr{x}(c)$ whenever $c \in C_K$ and $\orbres_K(\vr{x})(c) = 0$ otherwise.
Hence
\begin{equation}\label{eq:orbres 2}
\sum_{c\in C_K}\vr{c}(c)\cdot\vr{x}(c) =
\sum_{c\in C}\vr{c}(c) \cdot \orbres_K(\vr{x})(c) =
\transpose{\vr{c}}\cdot\orbres(\vr{x}) \ .
\end{equation}
Combining \eqref{eq:orbres 1} and \eqref{eq:orbres 2} we get
\[
\transpose{\vr{c}}\cdot\vr{x} = \transpose{\vr{c}}\cdot\orbres_K(\vr{x}) \ .
\]

Now we show $\orbres_K(\vr{A}\cdot\vr{x}) = \vr{A}\cdot\orbres_K(\vr{x})$.
Let $B_K = \setof{c\in C}{\supp{c}\subseteq K}$.
%
\begin{claim}\label{clm:orbres fin vec}
Both $\orbres_K(\vr{A}\cdot\vr{x})$ and $\vr{A}\cdot\orbres_K(\vr{x})$ are finite vectors supported by $K$.
\end{claim}
%
\begin{claimproof}
By \Cref{lem:orbres supp},
$\orbres_K(\vr{A}\cdot\vr{x})$ is a finite vector supported by $K$.
The same lemma also implies $\orbres_K(\vr{x})$ is a finite vector supported by $K$.
Since $\vr{A}$ is supported by $S\subseteq K$,
$\vr{A}\cdot\orbres_K(\vr{x})$ is also supported by $K$.
Since $\vr{A}$ is column-finite,
\Cref{cor:col fin prod} implies $\vr{A}\cdot\orbres(\vr{x})$ is also a finite vector.
\end{claimproof}
%
Let $B_K = \setof{c\in C}{\supp{c}\subseteq K}$.
\Cref{clm:orbres fin vec} and \Cref{lem:fin vec supp} together imply that
\[
\dom{\orbres_K(\vr{A}\cdot\vr{x})},
\dom{\vr{A}\cdot\orbres_K(\vr{x})} \subseteq B_K \ .
\]
Hence to show $\orbres_K(\vr{A}\cdot\vr{x})$ and $\vr{A}\cdot\orbres_K(\vr{x})$ are equal it is enough to show that they agree on $B_K$.

Pick $b\in B_K$.
By definition of $\orbres_K$, we get
%
\begin{equation}\label{eq:orbres 3}
\orbres_K(\vr{A}\cdot\vr{x})(b) = (\vr{A}\cdot\vr{x})(b)
= \vr{A}(b,-)\cdot\vr{x}\ .
\end{equation}
%
Both $\vr{A}$ and $\vr{x}$ are supported by $K$.
Hence, using \ref{lem:row col supp} we get that the row $\vr{A}(b,-)$ is also supported by $K$.
Recall that in the beginning of the proof we defined $C_K \subseteq C$ as
\[
C_K = \setof{c\in C}{\supp{c}\subseteq K} \ .
\]
Applying \Cref{lem:inner product fin} we get
%
\begin{equation}\label{eq:orbres 4}
\vr{A}(b,-)\cdot\vr{x} = \sum_{c\in C_K} \vr{A}(b,c)\cdot\vr{x}(c) \ .
\end{equation}
%
By definition of $\orbres_K$,
the vector $\orbres_K(\vr{x})$ is $0$ outside $C_K$.
Hence,
%
\begin{equation}\label{eq:orbres 5}
\sum_{c\in C_K} \vr{A}(b,c)\cdot\vr{x}(c)
=
\sum_{c\in C} \vr{A}(b,c)\cdot\orbres_K(\vr{x})(c) = (\vr{A}\cdot\orbres_K(\vr{x}))(b)\ .
\end{equation}
%
Combining \eqref{eq:orbres 3}-\eqref{eq:orbres 5} we get
\[
\orbres_K(\vr{A}\cdot\vr{x})(b) = (\vr{A}\cdot\orbres_K(\vr{x}))(b) \ ,
\]
which finishes the proof.
\end{proof}
%
Finally we are ready to prove \Cref{lem:orbres sol}.
%
\begin{proof}[Proof of \Cref{lem:orbres sol}]
Pick an orbit-finite solution $\vr{x}$ of $\cU$.
Let $K = S\cup \supp{\vr{x}}$.
By definition, $\orbres(\vr{x}) = \orbres_K(\vr{x})$.
Clearly $\orbres_K(\vr{x})\geqslant\vr{0}$.
Using \Cref{lem:orbres mat vec} we get $\transpose{\vr{c}}\cdot\vr{x} = \transpose{\vr{c}}\cdot\orbres_K(\vr{x})$.
We also have
%
\begin{align*}
   \vr{A}\cdot\orbres_K(\vr{x})
&= \orbres_K(\vr{A}\cdot\vr{x}) & \text{\Cref{lem:orbres mat vec}}\ \\
&\leqslant \orbres_K(\vr{b})            & \text{\Cref{lem:orbres lin}}\ \\
&= \vr{b}                       & \text{\Cref{lem:orbres fin vec},} \\
\end{align*}
%
which finishes the proof.
\end{proof}
%
\section{The orbit summation function}\label{sec:orbsum}
%
In this section we present some lemmas regarding the orbit summation function $\orbsum{S}$ and prove \Cref{lem:orbsum sol} using them.
%
%\begin{lemma}\label{lem:cod orbsum S supp}
%Every element in $\R^{\orbits[S]{D}}$ is supported by $S$.
%\end{lemma}
%%
%\begin{proof}
%Pick arbitrary $\vr{v}\in\R^{\orbits[S]{D}}$, $\pi\in\aut[S]{\A}$ and $K\in\orbit[S]{D}$.
%We have $\pi^{-1}(K) = K$.
%Using this and \Cref{lem:perm fun}
%\[
%\pi(\vr{v})(K) = \vr{v}(\pi^{-1}(K)) = \vr{v}(K) \ .
%\]
%\end{proof}
%

\begin{lemma}\label{lem:cod orbsum S supp}
For any orbit-finite set $P$ supported by $S$,
every element in $\R^{\orbits[S]{P}}$ is supported by $S$.
\end{lemma}
%
\begin{proof}
Pick arbitrary $\vr{v}\in\R^{\orbits[S]{P}}$, $\pi\in\aut[S]{\A}$ and $K\in\orbit[S]{P}$.
Using \ref{lem:perm orb aut} we get $\pi^{-1}(K) = K$.
Applying \Cref{lem:perm fun} we conclude
\[
\pi(\vr{v})(K) = \vr{v}(\pi^{-1}(K)) = \vr{v}(K) \ .
\]
\end{proof}
%
Recall the definition of $\orbsum{S}$ (\Cref{def:orbsum vector}).
%
\begin{lemma}\label{lem:orbsum prop}
The function $\orbsum{S}$ % : \flin{D} \to \R^{\orbits[S]{D}}$
is an $S$-supported monotonic linear map.
\end{lemma}
%
\begin{proof}
Linearity and monotonicity of $\orbsum{S}$ follows easily from its definition.
We focus on proving that it is supported by $S$.

Let $P$ be an arbitrary orbit-finite set supported by $S$.
Pick arbitrary $\vr{v}\in\normSp{P}$ and $\pi\in\aut[S]{\A}$.
For any $K\in\orbits[S]{P}$% using \ref{lem:perm orb aut} we get
\begin{align*}
\orbsum{S}(\pi(\vr{v}))(K) & =
\sum_{p\in K} \pi(\vr{v})(p) &  \\
& = \sum_{p\in K} \vr{v}(\pi^{-1}(p)) &  (\text{\Cref{lem:perm fun}})\\
& = \sum_{p\in K} \vr{v}(p) & (\text{\ref{lem:perm orb aut}})\\
&= \orbsum{S}(\vr{v})(K) & \ .\\
\end{align*}
Hence $\orbsum{S}(\pi(\vr{v})) = \orbsum{S}(\vr{v})$.
Now using \Cref{lem:cod orbsum S supp} we get  $\orbsum{S}(\vr{v}) = \pi(\orbsum{S}(\vr{v}))$.
Hence $\orbsum{S}(\pi(\vr{v})) = \pi(\orbsum{S}(\vr{v}))$.
Since $\pi\in\aut[S]{\A}$ was chosen arbitrarily, \Cref{lem:perm im S supp} now implies $\orbsum{S}$ is supported by $S$.
\end{proof}
%
\begin{lemma}\label{lem:orbsum matrix well def}
The function $\orbSum{S}$ is well-defined for $S$-supported column-finite matrices.
%For any $L\in\orbits[S]{E}$ and $c,c'\in L$
%\[
%\orbsum{S}(\vr{B}(-,c)) = \orbsum{S}(\vr{B}(-,c'))
%\]
%
\end{lemma}
%
\begin{proof}
%
Pick orbit-finite sets $P$ and $Q$ and column-finite matrix $\vr{B}\in\lin{P{\times}Q}$,
all of them supported by $S$.
We show for any two elements $q$ and $q'$ in the same $S$-orbit of $Q$.
\[
\orbsum{S}(\vr{B}(-,q)) = \orbsum{S}(\vr{B}(-,q'))\ .
\]
%

Pick such $q,q'\in Q$ arbitrarily.
Since they are in the same $S$-orbit,
there exists $\pi\in\aut[S]{\A}$ such that $\pi(q)=q'$.
Using \Cref{lem:cod orbsum S supp} we get
\[
\orbsum{S}(\vr{B}(-,q)) = \pi(\orbsum{S}(\vr{B}(-,q))) \ .
\]
Since $\orbsum{S}$ is supported by $S$ (\Cref{lem:orbsum prop}),
from \Cref{lem:perm im S supp} it follows that
\[
\pi(\orbsum{S}(\vr{B}(-,q))) = \orbsum{S}(\pi(\vr{B}(-,q))) \ .
\]
%
The matrix $\vr{B}$ is assumed to be supported by $S$.
Using \ref{lem:perm mat row} we conclude
\[
\orbsum{S}(\pi(\vr{B}(-,q))) = \orbsum{S}(\vr{B}(-,\pi(q)))
= \orbsum{S}(\vr{B}(-,q'))\ .
\]
As a consequence of the above equalities we get
\[
\orbsum{S}(\vr{B}(-,q)) = \orbsum{S}(\vr{B}(-,q')) \ .
\]
\end{proof}
%
%As we promised,
%now we show that this definition indeed achieves what we intended it to.
%%
%\arka{actually this lemma is superfluous now}
%
%\begin{lemma}\label{lem:orbsum matrix}
%For a column-finite $B{\times} C$-matrix $\vr{A}$ supported by $S$
%\[
%\orbsum{S}({\setof{\vr{A}(-,c)}{c\in C}}) =
%\setof{(\orbSum{S}(\vr{A}))(-,L)}{L\in \orbits[S]{C}}
%\]
%\end{lemma}
%%
%\begin{proof}
%\[
%\begin{aligned}
%  & \orbsum{S}({\setof{\vr{A}(-,c)}{c\in C}}) \\
%= & \setof{\orbsum{S}(\vr{A}(-,c))}{c\in C} \\
%= & \setof{(\orbSum{S}(\vr{A}))(-,\orbit[S]{c})}{c\in C} \\
%= & \setof{(\orbSum{S}(\vr{A}))(-,L)}{L\in \orbits[S]{C}}
%\end{aligned}
%\]
%\end{proof}
%
%\begin{lemma}\label{lem:orbsum mat vec}
%For any $S$-supported column-finite matrix $\vr{B}\in\lin{D{\times}C}$ and vector $\vr{x}\in\normSp{C}$ we have
%$\orbsum{S}(\vr{B}\cdot\vr{x}) = \orbSum{S}(\vr{B})\cdot\orbsum{S}(\vr{x})$.
%\end{lemma}
%
\begin{lemma}\label{lem:orbsum mat vec}
For any vector $\vr{x}\in\normSp{C}$ we have
$\orbsum{S}(\vr{A}\cdot\vr{x}) = \orbSum{S}(\vr{A})\cdot\orbsum{S}(\vr{x})$
and
$\transpose{\vr{c}}\cdot\vr{x} = \orbSum{S}(\transpose{\vr{c}})\cdot\orbsum{S}(\vr{x})$.
\end{lemma}
%
\begin{proof}
Pick $\vr{x}\in\normSp{C}$.
We prove $\orbsum{S}(\vr{A}\cdot\vr{x}) = \orbSum{S}(\vr{A})\cdot\orbsum{S}(\vr{x})$.
The proof of $\transpose{\vr{c}}\cdot\vr{x} = \orbSum{S}(\transpose{\vr{c}})\cdot\orbsum{S}(\vr{x})$ is similar.

Pick arbitrary $K\in\orbits[S]{B}$.
We show
\[
\orbsum{S}(\vr{A}\cdot\vr{x})(K) = (\orbSum{S}(\vr{A})\cdot\orbsum{S}(\vr{x}))(K)
\ .
\]
The following sequence of equations finish the proof
\begin{align*}
&& \orbsum{S}(\vr{A}\cdot\vr{x})(K) =
&  \sum_{b\in K} (\vr{A}\cdot\vr{x})(b)
& \text{(\Cref{def:orbsum vector})} \\
&& =
& \sum_{b\in K}\sum_{c\in C}\vr{A}(b,c)\cdot\vr{x}(c)
& \\
&& =
& \sum_{b\in K}\sum_{M\in\orbits[S]{C}}\sum_{c\in M}\vr{A}(b,c)\cdot\vr{x}(c)
& \\
&& =
&\sum_{M\in\orbits[S]{C}}
 \sum_{c\in M}\vr{x}(c)\cdot\left(
 \sum_{b\in K}\vr{A}(b,c)\right)
& \text{(rearrangement)} \\
&& =
& \sum_{M\in\orbits[S]{C}}
  \sum_{c\in M}\vr{x}(c)\cdot\orbSum{S}(\vr{A})(K,M)
& \text{(\Cref{def:orbsum matrix})}\\
&& =
& \sum_{M\in\orbits[S]{C}}\orbSum{S}(\vr{A})(K,M)\cdot
  \sum_{c\in M}\vr{x}(c)
& \text{(rearrangement)} \\
&& =
& \sum_{M\in\orbits[S]{C}}\orbSum{S}(\vr{A})(K,M)\cdot\orbsum{S}(\vr{x})(M)
& \text{(\Cref{def:orbsum vector})}\\
&& =
& \ (\orbSum{S}(\vr{A})\cdot\orbsum{S}(\vr{x}))(K)\ . \\
\end{align*}
Note that we can rearrange the sums freely due to \Cref{lem:lin prod norm} and 
\Cref{rem:sum L1}.
%\cite[Theorem 3.55]{babyRudin}.
\end{proof}
%
The following lemma is not used in our current context.
But we believe it is a fundamental result and therefore worth stating.
\begin{lemma}\label{cor:orbsum mat prod}
The function $\orbSum{S}$ commutes with matrix multiplication.
%For $S$-supported column-finite matrices $\vr{B}\in\lin{D{\times}E}$ and $\vr{C}\in\lin{E{\times}F}$ we have
%$\orbSum{S}(\vr{B}\cdot\vr{C}) = \orbSum{S}(\vr{B})\cdot\orbSum{S}(\vr{C})$.
\end{lemma}
%
\begin{proof}%[Proof of \Cref{cor:orbsum mat prod}]
Pick arbitrary orbit-finite sets $P$, $Q$ and $R$,
and column-finite matrices $\vr{B}\in\lin{P{\times}Q}$ and $\vr{C}\in\lin{Q{\times}R}$, all supported by $S$.
We prove that $\orbSum{S}(\vr{B}\cdot\vr{C}) = \orbSum{S}(\vr{B})\cdot\orbSum{S}(\vr{C})$.

Let $L$ be an arbitrary $S$-orbit of $R$.
We show
\[
(\orbSum{S}(\vr{B}\cdot\vr{C}))(-,L) =
(\orbSum{S}(\vr{B})\cdot\orbSum{S}(\vr{C}))(-,L)
\]
Let $r$ be an arbitrary element of $L$.
\begin{align*}
   & (\orbSum{S}(\vr{B}\cdot\vr{C}))(-,L)
   & \\
=\ & \orbsum{S}((\vr{B}\cdot\vr{C})(-,r))
   & \text{(\Cref{def:orbsum matrix})} \\
=\ & \orbsum{S}(\vr{B}\cdot\vr{C}(-,r))
   & \\
=\ & \orbSum{S}(\vr{B})\cdot\orbsum{S}(\vr{C}(-,r))
   & \text{(\Cref{lem:orbsum mat vec})} \\
=\ & \orbSum{S}(\vr{B})\cdot(\orbSum{S}(\vr{C}))(-,L)
   & \text{(\Cref{def:orbsum matrix})} \\
=\ & (\orbSum{S}(\vr{B})\cdot\orbSum{S}(\vr{C}))(-,L) \ .
   &  \\
\end{align*}
\end{proof}
%
%\begin{corollary}\label{cor:orbsum prod}
%For any $\vr{x}\in\normSp{C}$ we have
%$\orbSum{S}(\transpose{\vr{c}})\cdot\orbsum{S}(\vr{x})
%=\transpose{\vr{c}}\cdot\vr{x}$.
%\end{corollary}
%%
%\begin{proof}
%Assuming $E = C$ and $D$ to be a singleton, from \Cref{lem:orbsum mat vec} we get
%$\orbSum{S}(\transpose{\vr{c}})\cdot\orbsum{S}(\vr{x})=
%\orbSum{S}(\transpose{\vr{c}}\cdot\vr{x})$.
%But $\transpose{\vr{c}}\cdot\vr{x}$ is a just a number,
%so it is safe to write
%$\orbSum{S}(\transpose{\vr{c}}\cdot\vr{x}) = \transpose{\vr{c}}\cdot\vr{x}$.
%\end{proof}
%
%\begin{lemma}\label{lem:orbsum ptime}
%The function $\orbSum{S}$ is computable in \exptime{} and in \ptime{} in fixed atom-dimension.
%\end{lemma}
%%
%\arka{TODO: cite representation given in \Cref{ch:equations}}
%%
%\begin{proof}[Proof of \Cref{lem:orbsum ptime}]
%The dimension of $\orbSum{S}(\vr{A})$ is $\orbits[S]{B}{\times}\orbits[S]{C}$,\\
%which is polynomial in the size of the standard representation of $\vr{A}$.
%Hence, it is enough to show that every entry of $\orbSum{S}(\vr{A})$ can be computed in \fadp.
%Pick arbitrary $(K,L)\in\orbits[S]{B}{\times}\orbits[S]{C}$.
%Say $b$ and $c$ are respectively the representative elements of $X$ and $Y$ in the standard representation of $\vr{A}$.
%We want to compute
%\[
%\orbSum{S}(\vr{A})(K,L) = \sum_{b'\in K} \vr{A}(b',c) \ .
%\]
%Let $K'\subseteq K$ be the set of elements in $K$ which are supported by ${(S\cup\supp{c})}$.
%\Cref{lem:col fin supp} implies
%\[
%\setof{b'\in K}{\vr{A}(b',c)\neq 0}\subseteq K' \ .
%\]
%Hence it is enough to compute $\sum_{b'\in K'} \vr{A}(b',c)$.
%We have two cases:
%%
%\paragraph{(Case 1: $|\supp{b}\setminus S| > |\supp{c}\setminus S|$)}
%%
%In this case, \Cref{lem:pick elem supp} implies $K'$ is empty.
%Hence $\sum_{b'\in K'} \vr{A}(b',c) = 0$.
%%
%\paragraph{(Case 2: $|\supp{b}\setminus S| \leqslant |\supp{c}\setminus S|$)}
%%
%In this case, we can compute some $b'\in K'$ using \Cref{lem:pick elem supp}.
%By \Cref{lem:supp aut T} we get
%\[
%K' = \aut{\supp{c}\setminus S}\cdot\set{b'} \ .
%\]
%Let $d$ be the atom-dimension of $\vr{A}$.
%Then $|\aut{\supp{c}\setminus S}|\leqslant d!$.
%Hence, using \Cref{lem:apply perm fadp,lem:eq check fadp} we can compute $K'$ in \fadp{}. 
%For every $b'\in K'$ we can compute $\vr{A}(b',c)$ in \fadp{} (\Cref{lem:matrix query ptime}).
%\arka{remove \fadp{}}
%Hence we can also compute $\sum_{b'\in K'} \vr{A}(b',c)$ in \fadp{}.
%%
%\end{proof}
%
\begin{subsecproof}{Proof of \Cref{lem:orbsum sol}}
%\begin{proof}
Consider an arbitrary solution $\vr{x}\in\normSp{C}$ of $\cU$.
By monotonicity of $\orbsum{S}$ (\Cref{lem:orbsum prop}), $\orbsum{S}(\vr{x})$ is non-negative.
Applying \Cref{lem:orbsum mat vec} we get
%
\begin{equation}\label{eq:orbsum sol 1}
\orbSum{S}(\vr{A})\cdot\orbsum{S}(\vr{x}) =
\orbsum{S}(\vr{A}\cdot\vr{x}) \ .
\end{equation}
%
Since $\vr{A}\cdot\vr{x}\leqslant\vr{b}$,
again using monotonicity of $\orbsum{S}$ (\Cref{lem:orbsum prop}) we get
%
\begin{equation}\label{eq:orbsum sol 2}
\orbsum{S}(\vr{A}\cdot\vr{x}) \leqslant \orbsum{S}(\vr{b}) \ .
\end{equation}
Combining equation \eqref{eq:orbsum sol 1} with inequality \eqref{eq:orbsum sol 2} we get
\[
\orbSum{S}(\vr{A})\cdot\orbsum{S}(\vr{x}) \leqslant
\orbsum{S}(\vr{b}) = \orbSum{S}(\vr{b})\ .
\]
Hence, $\orbsum{S}(\vr{x})$ is a solution of $\orbSum{S}(\cU)$.
Finally another use of \Cref{lem:orbsum mat vec} gives us
$
\orbSum{S}(\transpose{\vr{c}})\cdot\orbsum{S}(\vr{x}) =
\transpose{\vr{c}}\cdot\vr{x}
$.
%\end{proof}
\end{subsecproof}
%
\section{The semi-orbit distribution function}\label{sec:orbdis}
In this section we present some lemmas regarding the semi-orbit distribution function $\orbdis{S}{T}$,
and prove \Cref{lem:orbdis sol} using them.
Recall the definition of $\orbdis{S}{T}$ (\Cref{def:orbdis}).
Immediately from the definition we get:
%
\begin{lemma}\label{lem:orbdis lin}
The function $\orbdis{S}{T}$ %: \R^{\orbits[S]{D}}\to\flin{D}$
is a monotonic linear map.
\end{lemma}
%
Recall the definition of $\aut[S\cup\set{T}]{\A}$ given in \Cref{notation:aut set T}.
%
\begin{definition}\label{def:supp S T}
A set $x$ is said to be \defindS{supported by ${S\cup\{T\}}$} if $\pi(x) = x$ for any $\pi\in\aut[S\cup\set{T}]{\A}$.
%is supported by $S\cup T$ and is invariant under $\aut{T}$ (recall \Cref{notation:aut set T}), i.e.
%\[
%\pi(x) = x
%\ \ \text{for all}\ \
%\pi\in\aut{T}\ .
%\]
\end{definition}
%
\begin{remark}
\Cref{def:set T supp}, which was useful in solving orbit-finite linear programs is a special case of the above definition. 
\end{remark}
%
\begin{lemma}\label{lem:supp S implies S T}
If a set $x$ is supported by $S$ then it is also supported by $(S\cup\set{T})$.
\end{lemma}
%
\begin{proof}
Follows from the fact that
$\aut[S\cup\set{T}]{\A}\subseteq \aut[S]{\A}$.
\end{proof}
%
\begin{lemma}\label{lem:orbdis supp}
%Any vector in the image of $\orbdis{S}{T}$ is finite and supported by $S\cup\set{T}$.
%
%\arka{Which formulation is better}
%
For any orbit-finite set $P$ supported by $S$ and $\vr{x}\in\R^{\orbits[S]{P}}$, the vector $\orbdis{S}{T}(\vr{x})$ is finite and is supported by ${S\cup\set{T}}$.
\end{lemma}
%
\begin{proof}
First we show $\orbdis{S}{T}(\vr{x})$ is supported by $S\cup T$.
For $K\in\orbits[S]{P}$ define
\[
K_{S\cup T} = \setof{p\in P}{\supp{p}\subseteq (S\cup T)} \ . 
\]
The vector $(\orbdis{S}{T}(\vr{x}))(p)\neq 0$ only if $p\in K_{S\cup T}$ for some $K\in\orbits[S]{P}$.
%\ref{lem:supp count} implies that for every $K\in\orbits[S]{P}$,
%the set $K_{S\cup T}$ is finite.
%Hence  $\orbdis{S}{T}(\vr{x})$ is also a finite vector.
%By definition, every element of $K_{S\cup T}$ is supported by $S\cup T$.
Applying \Cref{lem:fin vec supp} we conclude $\orbdis{S}{T}(\vr{x})$ is supported by $S\cup T$.

Pick arbitrary $\tau\in\aut[S\cup\set{T}]{\A}$.
We prove $\tau(\orbdis{S}{T}(\vr{x})) = \orbdis{S}{T}(\vr{x})$.
There exists $\sigma\in\aut[S\cup T]{\A}$ and $\pi\in\aut{T}$ (recall \Cref{notation:aut set T}) such that $\tau = \pi\circ\sigma$.
We have proved that $\orbdis{S}{T}(\vr{x})$ is supported by $S\cup T$.
Hence $\sigma(\orbdis{S}{T}(\vr{x})) = \orbdis{S}{T}(\vr{x})$.
To finish the proof we show
$\pi(\orbdis{S}{T}(\vr{x})) = \orbdis{S}{T}(\vr{x})$.
Recall \Cref{notation:fin vec}.
Since addition of vectors commutes with the action of automorphims (\Cref{lem:perm add mult}),
we get
\begin{equation}\label{eq:orbdis supp 4}
\pi(\orbdis{S}{T}(\vr{x})) =
\left(
\sum_{K\in\orbits[S]{P}}
\frac{\vr{x}(K)}{|K_{S\cup T}|} \cdot \sum_{p\in K_{S\cup T}} \pi(p)
\right)
\end{equation}
\ref{lem:supp aut T} implies that $\pi(K_{S\cup T}) = K_{S\cup T}$.
Then $\pi$ induces a permutation of $K_{S\cup T}$ (with $\pi^{-1}$ inducing the inverse permutation).
Hence
\begin{equation}\label{eq:orbdis supp 5}
\sum_{p\in K_{S\cup T}} \pi(p) = \sum_{p\in K_{S\cup T}} p
\end{equation}
Using equations \eqref{eq:orbdis supp 4} and \eqref{eq:orbdis supp 5} we conclude
\[
\pi(\orbdis{S}{T}(\vr{x}))
=
\left(
\sum_{K\in\orbits[S]{P}}
\frac{\vr{x}(K)}{|K_{S\cup T}|} \cdot \sum_{p\in K_{S\cup T}} p
\right)
= \orbdis{S}{T}(\vr{x})
\]
%Since $\pi\in\aut{T}$ was arbitrarily chosen,
%%this proves for any $\pi\in\aut{T}$ we have
%%$(\pi\circ\orbdis{S}{T})(\vr{x}) = \vr{x}$.
%%Since $\vr{x}\in\R^{\orbits[S]{B}}$ was arbitrarily chosen,
%this finishes the proof of the lemma.
\end{proof}
%
\begin{lemma}\label{lem:orbsum orbdis inv}
$\orbsum{S}\circ\orbdis{S}{T} = \id$.
\end{lemma}
%
\begin{proof}
The proof is an easy application of the definitions of $\orbsum{S}$ and $\orbdis{S}{T}$ and is left to the reader.
%Pick arbitrary orbit-finite set $D$ and vector $\vr{x}\in\R^{\orbits[S]{D}}$.
%We show $(\orbsum{S}\circ\orbdis{S}{T})(\vr{x}) = \vr{x}$.
%Using the definition of $\orbdis{S}{T}(\vr{x})$
%\begin{equation}\label{eq:orbsum orbsum inv 1}
%\orbdis{S}{T}(\vr{x}) =
%\sum_{K\in\orbits[S]{D}}\left(
%\frac{\vr{x}(K)}{|K_{S\cup T}|} \cdot \sum_{b\in K_{S\cup T}} b
%\right)\ ,
%\end{equation}
%where for $K\in\orbits[S]{D}$ the set
%\[
%K_{S\cup T} = \setof{b\in K}{\supp{b}\subseteq S\cup T} \ .
%\]
%Pick arbitrary $K\in\orbits[S]{D}$.
%For $b\in D$ we have $((\orbdis{S}{T}(\vr{x}))(b))(K) = 1$ if $b\in K$,
%otherwise $((\orbdis{S}{T}(\vr{x}))(b))(K) = 0$.
%Now applying linearity of $\orbsum{S}$ (\Cref{lem:orbsum prop}) to equation \eqref{eq:orbsum orbsum inv 1} we get
%\[
%(\orbsum{S}(\orbdis{S}{T}(\vr{x})))(K) = 
%\frac{\vr{x}(K)}{|K_{S\cup T}|} \cdot |K_{S\cup T}| = \vr{x}(K) \ .
%\]
%Since $K\in\orbits[S]{D}$ was chosen arbitrarily this proves
%\[
%\orbsum{S}(\orbdis{S}{T}(\vr{x})) = \vr{x} \ .
%\]
\end{proof}
%
\begin{lemma}\label{lem:orbdis orbsum inv}
%For any vector $\vr{x}\in\flin{D}$ supported by ${(S\cup\set{T})}$
For any $S$-supported orbit-finite set $P$ and $(S\cup\set{T})$-supported finite vector $\vr{x}\in\flin{P}$
\[
(\orbdis{S}{T}\circ\orbSum{S})(\vr{x}) = \vr{x} \ .
\]
\end{lemma}
%
\begin{proof}%[Proof of \Cref{lem:orbdis orbsum inv}]
Pick arbitrary $p\in P$.
We show
\[
((\orbdis{S}{T}\circ\orbsum{S})(\vr{x}))(p) = \vr{x}(p)
\]
We split the proof into two cases.

\paragraph{(Case 1: $p$ is not supported by $S\cup T$)}
The vector $\vr{x}$ is a finite vector supported by $S\cup T$.
\Cref{lem:orbdis supp} implies $(\orbdis{S}{T}\circ\orbsum{S})(\vr{x})$ is also a finite vector supported by $S\cup T$.
Applying \Cref{lem:fin vec supp}
\[
((\orbdis{S}{T}\circ\orbsum{S})(\vr{x}))(p) = 0 = \vr{x}(p)
\]

\paragraph{(Case 2: $p$ is supported by $S\cup T$)}
For $K\in\orbit[S]{P}$ define $K_{S\cup T}$ as done in \Cref{def:orbdis}.
Expanding the expression $(\orbdis{S}{T}\circ\orbsum{S})(\vr{x})$ using the definition of $\orbsum{S}$ and $\orbdis{S}{T}$ (\Cref{def:orbsum vector,def:orbdis})
\[
(\orbdis{S}{T}\circ\orbsum{S})(\vr{x}) =
\sum_{K\in\orbits[S]{P}}
\frac{(\orbsum{S}(\vr{x}))(K)}{|K_{S\cup T}|} \cdot
\left(
\sum_{p'\in K_{S\cup T}} p'
\right)
\]
Let $L = \orbit[S]{p}$.
Since $p$ is supported by $S\cup T$, we have $p\in L_{S\cup T}$.
Hence
\[
((\orbdis{S}{T}\circ\orbsum{S})(\vr{x}))(p) =
\frac{(\orbsum{S}(\vr{x}))(L)}{|L_{S\cup T}|}
\]
To finish the proof we need to show
\[
\vr{x}(p) = \frac{(\orbsum{S}(\vr{x}))(L)}{|L_{S\cup T}|}
\]
%
\begin{claim}\label{clm:lem:orbdis orbsum inv}
For any $p'\in L_{S\cup T}$ we have $\vr{x}(p) = \vr{x}(p')$.
\end{claim}
%
\begin{claimproof}
Pick arbitrary $p'\in L_{S\cup T}$.
Using \ref{lem:supp aut T} we can find $\pi\in\aut{T}$ such that $\pi(p) = p'$.
Since $\vr{x}$ is assumed to be supported by $(S\cup\set{T})$, we have
\[
\pi^{-1}(\vr{x}) = \vr{x}
\]
Using \Cref{lem:perm fun} we get
\[
\vr{x}(p') = \vr{x}(\pi(p)) =
\pi(\pi^{-1}(\vr{x}))(\pi(p)) =
(\pi^{-1}(\vr{x}))(p) = \vr{x}(p)
\]
This finishes the proof of the claim.
\end{claimproof}
%
\Cref{lem:supp S implies S T} implies $\vr{x}$ is supported by $S\cup T$.
Hence, if $\vr{x}(p')\neq 0$ for some $p'\in L$ then $p'\in L_{S\cup T}$.
This implies
\[
  \sum_{p'\in L_{S\cup T}} \vr{x}(p')
= \sum_{p'\in L} \vr{x}(p')
= (\orbsum{S}(\vr{x}))(L)
\]
Now \Cref{clm:lem:orbdis orbsum inv} implies
\[
\vr{x}(p) = \frac{(\orbsum{S}(\vr{x}))(L)}{|L_{S\cup T}|}
\]
which finishes the proof for this case and also of the lemma.
\end{proof}
%
\begin{lemma}\label{lem:orbdis mat vec}
%For any $S$-supported column-finite matrix $\vr{B}\in\lin{D{\times}C}$ and vector ${\vr{x}\in\R^{\orbits[S]{C}}}$
%\[
%\vr{B}\cdot\orbdis{S}{T}(\vr{x}) =
%\orbdis{S}{T}(\orbSum{S}(\vr{B})\cdot\vr{x})
%\]
For every vector ${\vr{x}\in\R^{\orbits[S]{C}}}$
\[
\vr{A}\cdot\orbdis{S}{T}(\vr{x}) =
\orbdis{S}{T}(\orbSum{S}(\vr{A})\cdot\vr{x})
\quad\text{and}\quad
\transpose{\vr{c}}\cdot\orbdis{S}{T}(\vr{x}) =
\orbSum{S}(\transpose{\vr{c}})\cdot\vr{x} \ .
\]
\end{lemma}
%
\begin{proof}
We prove $\vr{A}\cdot\orbdis{S}{T}(\vr{x}) = \orbdis{S}{T}(\orbSum{S}(\vr{A})\cdot\vr{x})$,
and leave it to the reader to verify that
$\transpose{\vr{c}}\cdot\orbdis{S}{T}(\vr{x}) =
\orbSum{S}(\transpose{\vr{c}})\cdot\vr{x}$.

\begin{claim}\label{clm:orbdis mat vec}
The vector $\vr{A}\cdot\orbdis{S}{T}(\vr{x})$ is supported by $S\cup\set{T}$.
\end{claim}
%
\begin{claimproof}
%
Pick arbitrary $\tau\in\aut[S\cup\set{T}]{\A}$.
We show $\tau(\vr{A}\cdot\orbdis{S}{T}(\vr{x})) = \vr{A}\cdot\orbdis{S}{T}(\vr{x})$.
There exists $\sigma\in\aut[S\cup T]{\A}$ and $\pi\in\aut{T}$ such that
\[
\tau = \pi \circ \sigma \ .
\]

First we show $\sigma(\vr{A}\cdot\orbdis{S}{T}(\vr{x})) = \vr{A}\cdot\orbdis{S}{T}(\vr{x})$.
The matrix $\vr{A}$ is supported by $S$ and hence also by $S\cup T$.
\Cref{lem:orbdis supp} implies that the vector $\orbdis{S}{T}(\vr{x})$ is supported by $S\cup T$.
Using the fact that automorphisms commute with matrix multiplication (\ref{lem:mult equiv}) we get
\[
\sigma(\vr{A} \cdot        \orbdis{S}{T}(\vr{x})) =
\sigma(\vr{A})\cdot \sigma(\orbdis{S}{T}(\vr{x})) =
       \vr{A} \cdot        \orbdis{S}{T}(\vr{x}) \ .
\]

Now we show that
\[
\pi(\vr{A}\cdot\orbdis{S}{T}(\vr{x})) = \vr{A}\cdot\orbdis{S}{T}(\vr{x}) \ .
\]
Since $\aut{T}\subseteq\aut[S]{\A}$ and $\vr{A}$ is supported by $S$ we have
$\pi(\vr{A}) = \vr{A}$.
The vector $\orbdis{S}{T}(\vr{x})$ is supported by $S\cup\set{T}$ (\Cref{lem:orbdis supp}).
Hence
\[
\pi(\orbdis{S}{T}(\vr{x})) = \orbdis{S}{T}(\vr{x}) \ .\
\]
Applying \ref{lem:mult equiv} again we get
\[
\pi(\vr{A} \cdot    \orbdis{S}{T}(\vr{x})) =
\pi(\vr{A})\cdot\pi(\orbdis{S}{T}(\vr{x})) =
    \vr{A} \cdot    \orbdis{S}{T}(\vr{x}) \ .
\]
\end{claimproof}
%
Since $\orbdis{S}{T}\circ\orbsum{S}$ acts as identity on $S\cup\set{T}$-supported vectors (\Cref{lem:orbdis orbsum inv}),
the above claim gives us
%
\begin{equation}\label{eq:orbdis mat vec 1}
\vr{A}\cdot\orbdis{S}{T}(\vr{x}) =
(\orbdis{S}{T}\circ\orbsum{S})(\vr{A}\cdot\orbdis{S}{T}(\vr{x})) \ .
\end{equation}
%
Applying \Cref{lem:orbsum mat vec} we get
\begin{equation}\label{eq:orbdis mat vec 2}
(\orbdis{S}{T}\circ\orbsum{S})(\vr{A}\cdot\orbdis{S}{T}(\vr{x})) =
\orbdis{S}{T}(\orbSum{S}(\vr{A})\cdot(\orbsum{S}\circ\orbdis{S}{T})(\vr{x})) \ .
\end{equation}
Since $\orbsum{S}\circ\orbdis{S}{T} = \id{}$ (\Cref{lem:orbsum orbdis inv}),
we have
$(\orbsum{S}\circ\orbdis{S}{T})(\vr{x}) = \vr{x}$.
Hence
\begin{equation}\label{eq:orbdis mat vec 3}
  \orbdis{S}{T}(\orbSum{S}(\vr{A})\cdot(\orbsum{S}\circ\orbdis{S}{T})(\vr{x}))
= \orbdis{S}{T}(\orbSum{S}(\vr{A})\cdot\vr{x}) \ .
\end{equation}
Combining equations
\eqref{eq:orbdis mat vec 1},
\eqref{eq:orbdis mat vec 2} and
\eqref{eq:orbdis mat vec 3}
we get
\[
  \vr{A}\cdot\orbdis{S}{T}(\vr{x})
= \orbdis{S}{T}(\orbSum{S}(\vr{A})\cdot\vr{x})\ ,
\]
which finishes the proof of lemma.
\end{proof}
%
%Putting $\vr{B} = \transpose{\vr{c}}$ we get:
%%
%\begin{corollary}\label{cor:orbdis prod}
%For any $\vr{x}\in\R^{\orbits[S]{C}}$ we have
%$\transpose{\vr{c}}\cdot\orbdis{S}{T}(\vr{x}) =
%\orbSum{S}(\transpose{\vr{c}})\cdot\vr{x}$.
%\end{corollary}
%%
%\begin{proof}
%Using \Cref{lem:orbdis mat vec} we get
%\[
%\transpose{\vr{c}}\cdot\orbdis{S}{T}(\vr{x}) =
%\orbdis{S}{T}(\orbSum{S}(\transpose{\vr{c}})\cdot\vr{x})
%\]
%Since $B$ is a singleton, the vector $\orbdis{S}{T}(\orbSum{S}(\transpose{\vr{c}})\cdot\vr{x})$ is one dimensional,
%and can be replaced with the number $\orbSum{S}(\transpose{\vr{c}})\cdot\vr{x}$.
%\end{proof}
%
\begin{subsecproof}{Proof of \Cref{lem:orbdis sol}}
Pick a solution $\vr{x}$ of $\orbSum{S}(\cU)$.
We have to show $\orbdis{S}{T}(\vr{x})$ is a $(S\cup T)$-supported solution of $\cU$ such that
\[
\transpose{\vr{c}}\cdot\orbdis{S}{T}(\vr{x}) = \orbSum{S}(\transpose{\vr{c}})\cdot\vr{x} \ .
\]
\Cref{lem:orbdis mat vec} gives us $
\transpose{\vr{c}}\cdot\orbdis{S}{T}(\vr{x}) =
\orbSum{S}(\transpose{\vr{c}})\cdot\vr{x}$.
\Cref{lem:orbdis supp} implies $\orbdis{S}{T}(\vr{x})$ is a finite vector supported by $S\cup T$.
It remains to show $\orbdis{S}{T}(\vr{x})$ is a solution of $\cU$.

Because $\vr{x}$ is a solution of $\orbSum{S}(\cU)$ it is non-negative and
\[
\orbSum{S}(\vr{A}) \cdot\vr{x}\leqslant \orbSum{S}(\vr{b}) \ .
\]
Using \Cref{lem:orbdis mat vec} we conclude $\orbdis{S}{T}(\vr{x})$ is non-negative as well.
Now \Cref{lem:orbdis mat vec,lem:orbdis lin} gives us
%
\begin{equation}\label{eq:orbdis leq}
\vr{A}\cdot\orbdis{S}{T}(\vr{x}) = 
\orbdis{S}{T}(\orbSum{S}(\vr{A})\cdot\vr{x})
\leqslant \orbdis{S}{T}(\orbSum{S}(\vr{b})) \ .
\end{equation}
%
The vector $\vr{b}$ is supported by $S$ and hence also by $S\cup\set{T}$.
Using \Cref{lem:orbdis orbsum inv} we conclude $\orbdis{S}{T}(\orbSum{S}(\vr{b})) = \vr{b}$.
Along with \eqref{eq:orbdis leq}, this implies $\vr{A}\cdot\orbdis{S}{T}(\vr{x}) \leqslant \vr{b}$ and finishes the proof.
%
\end{subsecproof}
%
\section{The orbit distribution function}\label{sec:orbrow}
%
%\begin{lemma}
%For any vector $\vr{x}$ the vector $\orbrow{S}(\vr{x})$ is supported by $S$.
%\end{lemma}
%%
%\begin{proof}
%Immediate from the definition of $\orbrow{S}$.
%\end{proof}
%
In this section we present some lemmas regarding the orbit distribution function $\orbrow{S}$,
and prove \Cref{lem:orbrow sol} using them.
Immediately from the definition of the orbit distribution function $\orbrow{S}$ we get:
%
\begin{lemma}\label{lem:orbrow lin}
The function $\orbrow{S}$ % : \R^{\orbits[S]{D}} \to \lin{D}$
is a monotonic linear function.
\end{lemma}
%
\begin{lemma}\label{lem:orbrow mat vec}
For any vector $\vr{y}\in\R^{\orbits[S]{B}}$
\[
\transpose{\vr{A}}\cdot\orbrow{S}(\vr{y}) =
\orbrow{S}(\transpose{\orbSum{S}(\vr{A})}\cdot\vr{y})
\quad\text{and}\quad
\transpose{\vr{b}}\cdot\orbrow{S}(\vr{y}) =
\transpose{\orbSum{S}(\vr{b})}\cdot\vr{y} \ .
\]
\end{lemma}
%
\begin{proof}
We prove the first equality, the proof of the second is left as an exercise.

Pick arbitrary $c\in C$.
Let $L = \orbit[S]{c}$.
We have the following sequence of equations proving 
$(\transpose{\vr{A}}\cdot\orbrow{S}(\vr{y}))(c) =
(\orbrow{S}(\orbSum{S}(\vr{A})\cdot\vr{y}))(c)$.
%
\begin{align*}
%       & (\transpose{\vr{A}}\cdot(\orbrow{S}(\vr{y})))(c)                    \\
       & (\transpose{\vr{A}}\cdot(\orbrow{S}(\vr{y})))(c) \\
=\quad & \sum_{b\in B} \transpose{\vr{A}}(c,b)\cdot(\orbrow{S}(\vr{y}))(b)   \\
=\quad & \sum_{b\in B} \vr{A}(b,c)\cdot(\orbrow{S}(\vr{y}))(b)               \\
=\quad & \sum_{K\in \orbits[S]{B}}
         \sum_{b\in K} \vr{A}(b,c)\cdot\vr{y}(K)
       & \text{(\Cref{def:orbrow})}                           \\
=\quad & \sum_{K\in \orbits[S]{B}}\vr{y}(K)\cdot
         \sum_{b\in K} \vr{A}(b,c)     & \text{(rearrangement)}          \\
=\quad & \sum_{K\in \orbits[S]{B}}\vr{y}(K)\cdot \orbSum{S}(\vr{A})(K,L)  
       & \text{(\Cref{def:orbsum matrix})}   \\
=\quad & \sum_{K\in \orbits[S]{B}}
         \transpose{\orbSum{S}(\vr{A})}(L,K)\cdot\vr{y}(K)  
                          \\
=\quad & (\orbSum{S}(\vr{A})\cdot\vr{y})(L)                                  \\
=\quad & (\orbrow{S}(\orbSum{S}(\vr{A})\cdot\vr{y}))(c) \ .
\end{align*}
\end{proof}
%
%For the special case when $D$ is singleton we get:
%
%\begin{corollary}\label{cor:orbrow prod}
%For any vector $\vr{y}\in\R^{(\orbits[S]{B})}$
%\[
%\transpose{\vr{b}}\cdot\orbrow{S}(\vr{y}) =
%\transpose{\orbSum{S}(\vr{b})}\cdot\vr{y}
%\]
%\end{corollary}
%
\begin{subsecproof}{Proof of \Cref{lem:orbrow sol}}
Observing that $\orbrow{S}(\transpose{\orbSum{S}(\transpose{\vr{c}})}) = \vr{c}$,
this lemma can be proven using \Cref{lem:orbrow lin,lem:orbrow mat vec} in the same way that \Cref{lem:orbsum sol} is proven using \Cref{lem:orbsum prop,lem:orbsum mat vec}.
\end{subsecproof}
%
\section{The semi-orbit summation function}\label{sec:orbsmt}
%
In this section we present some lemmas regarding the semi-orbit summation function $\orbsmt{S}{T}$,
and prove \Cref{lem:orbsmt sol} using them.
%
\begin{lemma}\label{lem:orbsmt lin supp}
The function $\orbsmt{S}{T}$ %: (D\to\R)\to\R^{\orbits[S]{D}}$
is a $(S\cup\set{T})$-supported monotonic linear map.
\end{lemma}
%
\begin{proof}
Left to the reader.
\end{proof}
%
\begin{lemma}\label{lem:orbrow orbsmt inv}
For any $S$-supported vector $\vr{y}$ we have %\in\lin{C}$
$
(\orbrow{S}\circ\orbsmt{S}{T})(\vr{y}) = \vr{y}
$.
\end{lemma}
%
\begin{proof}%[Proof of \Cref{lem:orbrow orbsmt inv}]
Follows from the definitions of $\orbrow{S}$ and $\orbsmt{S}{T}$ and left to the reader.
\end{proof}
%
\begin{lemma}\label{lem:orbsmt orbsum}
For any orbit-finite set $P$ and vector $\vr{y}\in\lin{P}$,
both supported by $S$ %\in\lin{B}$
\[
\transpose{(\orbsmt{S}{T}(\vr{y}))} = \orbSum{S}(\transpose{\vr{y}}) \ .
\]
\end{lemma}
%
\begin{proof}%[Proof of \Cref{lem:orbsmt orbsum}]
Using the same notation as \Cref{def:orbsmt} let
\[
K_{S\cup T} = \setof{p\in K}{\supp{p}\subseteq S\cup T}
\]
We have
\begin{align*}
       & (\orbsmt{S}{T}(\vr{y}))(K) \\
=\quad & \frac{1}{|K_{S\cup T}|} \cdot
         \left(
         \sum_{p\in K_{S\cup T}} \vr{y}(p)
         \right) \\
=\quad & \frac{1}{|K_{S\cup T}|} \cdot
         \left(
         \sum_{p\in K_{S\cup T}} \vr{y}(K)
         \right) \quad\text{(recall \Cref{lem:const dom})}\\
=\quad & \vr{y}(K) \\
=\quad & \transpose{(\orbSum{S}(\transpose{\vr{y}}))}(K)
\end{align*}
%
This finishes the proof of the lemma. 
%
\end{proof}
%
\begin{lemma}\label{lem:orbsmt aut T}
For every $S$-supported orbit-finite set $P$, vector $\vr{y}:P\to\R$,
$K\in\orbits[S]{P}$ and $(S\cup T)$-supported element $p\in K$,
\[
(\orbsmt{S}{T}(\vr{y}))(K) =
\left(
\frac{1}{|\aut{T}|} \cdot
\sum_{\pi\in\aut{T}} \pi(\vr{y})
\right)(p) \ .
\]
\end{lemma}
%
\begin{proof}%[Proof of \Cref{lem:orbsmt aut T}]
Define $K_{S\cup T}$ as done inside the definition of $\orbsmt{S}{T}$
(\Cref{def:orbsmt}).
We have
\[
(\orbsmt{S}{T}(\vr{y}))(K) =
\frac{1}{|K_{S\cup T}|}\cdot\sum_{p\in K_{S\cup T}} \vr{y}(p)
\]
Let
\[
\vr{z} = \frac{1}{|\aut{T}|} \cdot \sum_{\pi\in\aut{T}} \pi(\vr{y})
\]
Then
\begin{align*}
%       & \vr{z}(b) & \\
\vr{z}(p)=\quad & \left(
         \frac{1}{|\aut{T}|} \cdot \sum_{\pi\in\aut{T}} \pi(\vr{y})
         \right)(p) & \\
=\quad & \frac{1}{|\aut{T}|} \cdot \sum_{\pi\in\aut{T}} (\pi(\vr{y}))(p) & \\
=\quad & \frac{1}{|\aut{T}|} \cdot \sum_{\pi\in\aut{T}} \vr{y}(\pi^{-1}(p)) \ . &
\text{(\Cref{lem:perm fun})}
\end{align*}
%
Using \ref{lem:supp aut T} we get
\begin{align*}
       & \frac{1}{|\aut{T}|} \cdot \sum_{\pi\in\aut{T}} \vr{y}(\pi^{-1}(p)) & \\
=\quad & \frac{1}{|\aut{T}|} \cdot \left(\sum_{p'\in K_{S\cup T}} 
         \left|\setof{\pi\in\aut{T}}{\pi^{-1}(p) = p'}\right|\cdot
         \vr{y}(p') \right) & \\
=\quad & \frac{1}{|\aut{T}|} \cdot \left(\sum_{p'\in K_{S\cup T}} 
         \left|\setof{\pi\in\aut{T}}{\pi(p') = p}\right|\cdot
         \vr{y}(p') \right) & \\
\end{align*}
%
Hence
\[
\vr{z}(p) = \frac{1}{|\aut{T}|} \cdot \left(\sum_{p'\in K_{S\cup T}} 
         \left|\setof{\pi\in\aut{T}}{\pi(p') = p}\right|\cdot
         \vr{y}(p') \right)
\]
%
To finish the proof we show that for every $b'\in K_{S\cup T}$
\[
\left|\setof{\pi\in\aut{T}}{\pi(p') = p}\right| = \frac{|\aut{T}|}{|K_{S\cup T}|}
\]
For $p'\in K_{S\cup T}$ let $W_{p'} = \setof{\pi\in\aut{T}}{\pi(p') = p}$.
Then $\aut{T}$ is the disjoint union of the sets $W_{p'}$ for $p'$.
Hence
\begin{equation}\label{eq:orbsmt aut T 1}
|\aut{T}| = \sum_{p'\in K_{S\cup T}} |W_{p'}|
\end{equation}
Pick any $p'\in K_{S\cup T}$.
\ref{lem:supp aut T} implies that there exists $\sigma\in\aut{T}$ be such that $\sigma(p') = p$.
Then $\pi\mapsto (\sigma\circ\pi)$ is a bijection from $W_{p'}$ to $W_p$ with $\pi\mapsto (\sigma^{-1}\circ\pi)$ being its inverse.
Since $p'$ is an arbitrary element of $K_{S\cup T}$,
this implies for every $p'\in K_{S\cup T}$ we have $|W_{p'}| = |W_p|$.
Which together with equation \eqref{eq:orbsmt aut T 1} imply that for every $p'\in K_{S\cup T}$
\[
|W_{p'}| = \frac{|\aut{T}|}{|K_{S\cup T}|}
\]
This finishes the proof of the lemma.
\end{proof}
%
\begin{lemma}\label{lem:orbsmt mat vec}
For every vector $\vr{y} : B\to\R$
\[
\transpose{\orbSum{S}(\vr{A})}\cdot\orbsmt{S}{T}(\vr{y}) =
\orbsmt{S}{T}(\transpose{\vr{A}}\cdot\vr{y})
\quad\text{and}\quad
\transpose{\orbSum{S}(\vr{b})}\cdot\orbsmt{S}{T}(\vr{y}) =
\transpose{\vr{b}}\cdot\vr{y} \ .
\]
\end{lemma}
%
\begin{proof}[Proof of \Cref{lem:orbsmt mat vec}]
%
We prove $\transpose{\orbSum{S}(\vr{A})}\cdot\orbsmt{S}{T}(\vr{y}) =
\orbsmt{S}{T}(\transpose{\vr{A}}\cdot\vr{y})$,
the proof of $\transpose{\orbSum{S}(\vr{b})}\cdot\orbsmt{S}{T}(\vr{y}) =
\transpose{\vr{b}}\cdot\vr{y}$ is left as an exercise.

Pick arbitrary $L\in\orbits[S]{C}$.
Using \ref{lem:pick elem supp} pick $c\in L$ supported by $S\cup T$.
%
\begin{align*}
       & (\orbsmt{S}{T}(\transpose{\vr{A}}\cdot\vr{y}))(L)
       & \\
=\ & \frac{1}{|\aut{T}|} \cdot
         \left(
         \sum_{\pi\in\aut{T}}\pi(\transpose{\vr{A}}\cdot\vr{y})
         \right)(c)
       & \text{(\Cref{lem:orbsmt aut T})} \\
=\ & \frac{1}{|\aut{T}|} \cdot
         \left(
         \sum_{\pi\in\aut{T}}\pi(\transpose{\vr{A}})\cdot\pi(\vr{y})
         \right)(c)
       & \text{(\ref{lem:mult equiv})} \\
=\ & \frac{1}{|\aut{T}|} \cdot
         \left(
         \sum_{\pi\in\aut{T}}\transpose{\vr{A}}\cdot\pi(\vr{y})
         \right)(c)
       & \text{($\vr{A}$ is $S$-supported)} \\
=\ & \left(
         \transpose{\vr{A}}\cdot
         \frac{1}{|\aut{T}|} \cdot
         \left(
         \sum_{\pi\in\aut{T}}\pi(\vr{y})
         \right)
         \right)(c)
       & \\
=\ & \sum_{b\in B}
         \transpose{\vr{A}}(c,b)\cdot
         \left(
         \frac{1}{|\aut{T}|}
         \cdot
         \left(
         \sum_{\pi\in\aut{T}}\pi(\vr{y})
         \right)
         \right)
         (b)
       & \\
=\ & \sum_{b\in B}
         \transpose{\vr{A}}(c,b)\cdot
         (\orbsmt{S}{T}(\vr{y}))(\orbit[S]{b})
       & \text{(\Cref{lem:orbsmt aut T})} \\
=\ & \sum_{K\in\orbits[S]{B}}
         (\orbsmt{S}{T}(\vr{y}))(K)\cdot
         \sum_{b\in K} \vr{A}(b,c)
       & \text{(rearrangement)} \\
=\ & \sum_{K\in\orbits[S]{B}}
         (\orbsmt{S}{T}(\vr{y}))(K)\cdot\orbSum{S}(\vr{A})(K,L)
       & \text{(\Cref{def:orbsum matrix})} \\
=\ & (\transpose{\orbSum{S}(\vr{A})}\cdot\orbsmt{S}{T}(\vr{y}))(L) \ .
       &              
\end{align*}
%
This finishes the proof.
%%
%\begin{claim}\label{clm:orbsmt sol 1}
%The vector $\vr{z}$ defined as
%\[
%\vr{z} = \frac{1}{|\aut{T}|}\cdot\left(
%\sum_{\pi\in\aut{T}} \pi(\vr{y})
%\right)
%\]
%is a solution of $\transpose{\cU}$ with
%\[
%\transpose{\vr{b}}\cdot\vr{y} =
%\transpose{\vr{b}}\cdot\vr{z} 
%\]
%\end{claim}
%%
%\begin{claimproof}[Proof of \Cref{clm:orbsmt sol 1}]
%First we show that for every $\pi\in\aut{T}$, the vector $\pi(\vr{y})$ is also a solution of $\transpose{\cU}$ with
%$\transpose{\vr{b}}\cdot\pi(\vr{y}) = \transpose{\vr{b}}\cdot\vr{y}$.
%Pick arbitrary $\pi\in\aut{T}$.
%Since $\vr{b}$ is a finite vector, the product
%$\transpose{\vr{b}}\cdot\pi(\vr{y})$ is well defined.
%Since $\vr{y}$ is non-negative,
%$\pi(\vr{y})$ is also non-negative for every $\pi\in\aut{T}$.
%The sets $S$ and $T$ are disjoint.
%Hence, $\aut{T}\subseteq\aut[S]{\A}$.
%The matrix $\vr{A}$ and the vectors $\vr{b}$ and $\vr{c}$ are both supported by $S$.
%Hence,
%$\pi(\transpose{\vr{A}}) = \transpose{\pi(\vr{A})} = \transpose{\vr{A}}$,
%$\pi(\vr{b}) = \vr{b}$, and $\pi(\vr{c}) = \vr{c}$.
%Using this fact and applying \mbox{\Cref{lem:mult equiv}} we get
%\[
%\transpose{\vr{A}}\cdot\pi(\vr{y}) = \pi(\transpose{\vr{A}})\cdot\pi(\vr{y}) = \pi(\transpose{\vr{A}}\cdot\vr{y})
%\geqslant \pi(\vr{c}) = \vr{c}
%\]
%and
%\[
%\transpose{\vr{b}}\cdot\pi(\vr{y})
%= \pi(\transpose{\vr{b}})\cdot\pi(\vr{y}) = \transpose{\vr{b}}\cdot\vr{y}
%\]
%Since $\pi\in\aut{T}$ was chosen arbitrarily,
%this shows for every $\pi\in\aut{T}$, the vector $\pi(\vr{y})$ is also a solution of $\transpose{\cU}$ with 
%$\transpose{\vr{b}}\cdot\pi(\vr{y}) = \transpose{\vr{b}}\cdot\vr{y}$.
%
%Since $\vr{b}$ is a finite vector, the product $\transpose{\vr{b}}\cdot\vr{z}$ is well defined.
%The vector $\vr{z}$ is a convex combination of the vectors in the finite set
%\[
%\setof{\pi(\vr{y})}{\pi\in\aut{T}}
%\]
%Hence $\vr{z}$ is non-negative,
%$\transpose{\vr{A}}\cdot\vr{z}\geqslant\vr{c}$ and
%$\transpose{\vr{b}}\cdot\vr{z} = \transpose{\vr{b}}\cdot\vr{y}$.
%This finishes the proof.
%\end{claimproof}
%%
%\begin{claim}\label{clm:orbsmt sol}
%For any $b\in B$ supported by $S\cup T$
%\[
%\vr{z}(b) = (\orbsmt{S}{T}(\vr{y}))(\orbit[S]{b})
%\]
%\end{claim}
%%
%\begin{claimproof}
%Pick arbitrary $b\in B$ supported by $S\cup T$.
%\Cref{lem:sum aut T} implies
%\[
%\vr{z}(b) = \frac{1}{|\aut{T}\cdot b|}\sum_{b'\in \aut{T}\cdot b} \vr{b'} 
%\]
%%
%Let $K = \orbit[S]{b}$.
%%
%\end{claimproof}
\end{proof}
%
\begin{subsecproof}{Proof of \Cref{lem:orbsmt sol}}
This lemma follows from \Cref{lem:orbsmt lin supp,lem:orbsmt orbsum,lem:orbsmt mat vec} in the same way that \Cref{lem:orbdis sol} follows from \Cref{lem:orbdis lin,lem:orbdis supp,lem:orbdis orbsum inv,lem:orbdis mat vec}.
Note that we do not need a counterpart of \Cref{lem:orbdis supp} since unlike in the proof \Cref{lem:orbdis sol},
where we had to show that $\orbdis{S}{T}(\vr{x})$ is supported by $S\cup T$,
in \Cref{lem:orbsmt sol} we do not have to prove any similar assertion regarding $\orbsmt{S}{T}(\vr{y})$.
%
\end{subsecproof}
%
%\section{Solvability of column-and row-finite systems} 
%%
%\arka{Maybe no need for this section}
%
\section
[Do orbit-finite LP\MakeLowercase{s} approximate large LP\MakeLowercase{s}?]
{Do orbit-finite linear programs approximate large finite linear programs?}
\label{sec:approx}
\arka{Do we keep this?}


%
In this section we try to address the question raised in \Cref{sec:intro approx}.
Consider $S\subseteqfin \A$ and orbit-finite sets $B$ and $C$ supported by $S$.
Let $\cU$ be an orbit-finite linear program
\[
\primalAbc 
\]
%
where $\vr{A}\in\lin{B\times C}$, $\vr{b}\in\lin{B}$ and $\vr{c}\in\lin{C}$.
For any finite set $T\subseteqfin\A\setminus S$ let $B_{S\cup T}$ and $C_{S\cup T}$ be the subsets of elements respectively of $B$ and $C$,
which are supported by $S\cup T$.
Removing all variables indexed by elements outside $C_{S\cup T}$ and considering only the equalities indexed by $B_{S\cup T}$ we get a finite linear program.
Call this $\cU_{S\cup T}$.
Note that the function $T\mapsto \cU_{S\cup T}$ is $S$-supported.
For $n\in\N$,
we say $\cU$ is a \defindS{good approximation}, if there exists $n\in\N$ such that for all $T\subseteqfin\A\setminus S$ of size at least $n$,
the linear programs $\cU_{S\cup T}$ and $\cU$ are equivalent (\Cref{def:equiv lp}).
The \defindS{approximation degree} of $\cU$ is the smallest such $n$.

Orbit-finite systems are not always good approximations even if we ignore the objective function and focus on just the solvability of the constraints.
We illustrate this with the following example.
%
\begin{example}
Let $\star$ be an equivariant element,
i.e.\ $\pi(\star) = \star$ for all $\pi$ in $\aut{\A}$.
Let $\cU$ be the following system of inequalities with variables
$\setof{\vr{x}(\a)}{\a\in\A}$:
\begin{equation}\label{eg:not approx}
  \begin{aligned}
    \sum_{\a \in \A\setminus\set{\b}} \vr{x}(\a) & \leqslant  1 \quad (\b \in \A) \\
    \sum_{\a\in\A} \vr{x}(\a) & > 1 \ .
  \end{aligned}  
\end{equation}
We prove that $\cU_{S\cup T}$ is solvable for every $T\subseteqfin \A$ of size at least $2$,
but $\cU$ is unsolvable,
even of we allow orbit-infinite solutions.

Pick $T\subseteqfin \A$ of size at least $2$.
The system $\cU_{S\cup T}$ contains the following inequalities:
\[
  \begin{aligned}
    \sum_{\a \in T\setminus\set{\b}} \vr{x}(\a) & \leqslant  1 \quad (\b \in T) \\
    \sum_{\a\in T} \vr{x}(\a) & > 1 \ .
  \end{aligned}
\]
The vector solution $\vr{x}_T$ which assigns the value $\frac{1}{|T| - 1}$ to all the variables is a solution of $\cU_{S\cup T}$. 

Now we show unsolvability of $\cU$.
Assume otherwise.
Let $\vr{y} : \A\to\R$ be a solution of $\cU$.
The sum $\sum_{\a\in\A} \vr{y}(\a)$ is well defined.
\footnote{Here, by well-defined we mean that the series $\sum_{n\in\N}\vr{y}(\a_n)$ converges absolutely for some (every) enumeration $\a_1,\a_2,\dots$ of atoms \cite[Page 71]{babyRudin}.
This ensures that the sum $\sum_{n\in\N}\vr{y}(\a_n)$ converges to a unique real number independent of the enumeration $\a_1,\a_2,\dots$ of atoms (cf. \cite[Theorems 3.54 and 3.55]{babyRudin}).}
Hence for any $n \geqslant 1$ there exists $\b_n\in \A$ such that $\vr{y}(\b_n) \leqslant \frac{1}{n}$.
This implies for every $n\geqslant 1$
\[
  \sum_{\a\in\A} \vr{y}(\a)
  \ =\
  \left(\sum_{\a\in\A\setminus\set{\b_n}} \vr{y}(\a)\right) + \vr{y}(\b_n)
  \ \leqslant \
  1 + \frac{1}{n} \ .
\]
Since the above is true for all $n$,
we conclude
\[
  \sum_{\a\in\A} \vr{y}(\a) \leqslant 1 \ .
\]
This contradicts the fact the $\vr{y}$ is a solution of $\cU$.
\end{example}
%
The situation improves if we restrict our attention to orbit-finite linear programs which are either column-finite or row-finite.
%
\begin{theorem}\label{thm:col row approx}
Column-finite and row-finite linear programs are good approximations.
The approximation degree of a column(row)-finite linear program is not bigger than its atom-dimension.
%Let $\cU$ be an orbit-finite $B{\times}C$-linear program supported by some $S\subseteqfin\A$ and of atom-dimension $d$.
%If $\cU$ is either column-finite or row-finite,
%then for every $T\subseteqfin\A\setminus S$ of size at least $d$,
%the linear programs $\cU_{S\cup T}$ and $\cU$ are equivalent.
\end{theorem}
%
The remainder of this section is devoted to proving the above theorem.
%
\section*{Proof of \Cref{thm:col row approx}}
%
%\arka{Lemma commented out}
%
%\begin{lemma}\label{lem:col fin supp}
%For %a column-finite $(B{\times} C)$-matrix $\vr{A}$ and
%$(b,c)\in B{\times} C$, if $\vr{A}(b,c)\neq 0$ then,
%\[
%\supp{b}\subseteq (\supp{c}\cup\supp{\vr{A}})
%\]
%\end{lemma}
%%
%\begin{proof}
%We do a proof by contradiction.
%Suppose there exists $(b,c)\in(B{\times} C)$ such that $\vr{A}(b,c)\neq 0$ and
%\[
%\supp{b}\nsubseteq(\supp{c}\cup\supp{\vr{A}}) \ .
%\]
%Let $S = (\supp{c}\cup\supp{\vr{A}})$.
%Choose $\a\in (\supp{b}\setminus S)$.
%Let $T = \A\setminus(\supp{b}\cup S)$.
%Then $T$ is infinite.
%For every $\b\in T$ define $\pi_{\b}\in\aut{\A}$ as
%\[
%\pi_{\b}(\g) =
%\begin{cases}
%\b & \text{ if }\g = \a \\
%\a & \text{ if }\g = \b \\
%\g & \text{ otherwise.} \\ 
%\end{cases}
%\]
%Then $\pi_{\b}\in\aut[S]{\A}$ for every $\b\in T$.
%Hence $\pi_{\b}(c)=c$ for all such $\b$.
%\Cref{lem:supp fun equiv} says that for every $\b\in T$
%\[
%\supp{\pi_{\b}(b)} = \pi_{\b}(\supp{b}) = (\supp{b}\setminus\{\a\})\cup\{\b\}
%\ .
%\]
%This implies for any two distinct $\b,\b'\in T$,
%\[
%\supp{\pi_{\b}(b)}\neq\supp{\pi_{\b'}(b)}
%\]
%and in consequence $\pi_{\b}(b)\neq \pi_{\b'}(b)$
%This implies the set $\setof{\pi_{\b}(b)}{\b\in T}$ is infinite.
%However, for any
%$\b\in\A\setminus(\supp{b}\cup\supp{c})$
%\begin{align*}
%   & \vr{A}(\pi_{\b}(b),c) & \\
%=\ & \vr{A}(b,\pi_{\b}(c)) &
%\text{(since $\vr{A}$ is supported by $S$)} \\
%=\ & \vr{A}(b,c) & \text{(since $\pi_{\b}(c)=c$)}\\
%\neq\ & 0\,. &
%\end{align*}
%Hence $\vr{A}(-,c)\notin\flin{B}$,
%which contradicts the assumption that $\vr{A}$ is column-finite.
%\end{proof}
%
Pick an arbitrary orbit-finite column-finite linear program $\cU$.
WLOG assume it is a maximisation problem.
Its dual $\transpose{\cU}$ is an arbitrary orbit-finite row-finite linear program.
Let $S\subseteqfin\A$ be the support of $\cU$ and $d$ be its atom-dimension.
Then $S$ and $d$ are also respectively the support and atom-dimension of $\transpose{\cU}$.
We can write $\cU$ and $\transpose{\cU}$ as
\[
\primalAbc \qquad \dualAbc
\]
for some orbit-finite sets $B$ and $C$, column-finite matrix $\vr{A}\in\lin{B{\times}C}$, vectors $\vr{b}\in\flin{B}$ and $\vr{c}\in\lin{C}$, all of them supported by $S$.

Pick $T\subseteqfin \A\setminus S$ of size at least $d$.
Define $B_{S\cup T}$ and $C_{S\cup T}$ to be the subsets of respectively $B$ and $C$ containing elements which are supported by $S\cup T$:
%
\begin{gather*}
B_{S\cup T} = \setof{b\in B}{\supp{b}\subseteq S\cup T} \ \ \ \\
C_{S\cup T} = \setof{c\in C}{\supp{c}\subseteq S\cup T} \ .
\end{gather*}
%
Note that both $B_{S\cup T}$ and $C_{S\cup T}$ are finite due to \ref{lem:supp count}.
The linear programs $\cU_{S\cup T}$ and $\transpose{\cU_{S\cup T}}$ can be written as
\[
\lpMatPrimal{\vr{A}_{S\cup T}}{\vr{b}_{S\cup T}}{\vr{c}_{S\cup T}}{\vr{u}}{B_{S\cup T}}{C_{S\cup T}}
\qquad\qquad
\lpMatDual{\vr{A}_{S\cup T}}{\vr{b}_{S\cup T}}{\vr{c}_{S\cup T}}{\vr{v}}{B_{S\cup T}}{C_{S\cup T}}
\]
where,
$\vr{A}_{S\cup T}$, $\vr{b}_{S\cup T}$ and $\vr{c}_{S\cup T}$ respectively be the restrictions of $\vr{A}$, $\vr{b}$ and $\vr{c}$ to $B_{S\cup T}{\times}C_{S\cup T}$, $B_{S\cup T}$ and $C_{S\cup T}$.
Note that
$\transpose{(\vr{A}_{S\cup T})} = (\transpose{\vr{A}})_{S\cup T}$,
$\transpose{(\vr{b}_{S\cup T})} = (\transpose{\vr{b}})_{S\cup T}$,
and
$\transpose{(\vr{c}_{S\cup T})} = (\transpose{\vr{c}})_{S\cup T}$.
Hence we can write $\transpose{\vr{A}_{S\cup T}}$, $\transpose{\vr{b}_{S\cup T}}$, $\transpose{\vr{c}_{S\cup T}}$ without any confusion.
As a consequence, $\transpose{(\cU_{S\cup T})} = (\transpose{\cU})_{S\cup T}$.
So we can write $\transpose{\cU_{S\cup T}}$ without any ambiguity.

To prove \Cref{thm:col row approx} we show that the linear programs $\cU$ and $\cU_{S\cup T}$ are equivalent,
and so are the linear programs $\transpose{\cU}$ and $\transpose{\cU_{S\cup T}}$.

Observe that the current premise allows us to reuse \Cref{lem:restr sol,lem:restr sol dual} proven inside the proof of \Cref{thm:weak o.f. duality} in \Cref{subsec:weak dual proof}.
From \Cref{lem:restr sol} and \Cref{thm:col fin opt} we get:
%
\begin{lemma}\label{clm:opt sE < opt sET}
  The optimum of $\cU$ is smaller than or equal to the optimum of $\cU_{S\cup T}$.
%
 %For every solution of $\vr{x}'$ of $\cU$ there exists a solution $\vr{u}'$ of $\cU_{S\cup T}$ such that
  %$
  %\transpose{\vr{c}}\cdot\vr{x}' = \transpose{\vr{c}_{S\cup T}}\cdot\vr{u}'
  %$.
\end{lemma}
%
Similarly, \Cref{lem:restr sol dual} and \Cref{thm:row fin opt} gives us:
%
\begin{lemma}\label{clm:row opt sE > opt sET}
The optimum of $\transpose{\cU}$ is bigger than or equal to the optimum of $\transpose{\cU_{S\cup T}}$.
\end{lemma}
%
We will prove the following two lemmas:
%
\begin{lemma}\label{clm:opt sE > opt sET}
  The optimum of $\cU$ is larger than or equal to the optimum of $\cU_{S\cup T}$.
  %
  %For every solution of $\vr{u}'$ of $\cU_{S\cup T}$ there exists a solution $\vr{x}'$ of $\cU$ such that
  %$
  %\transpose{\vr{c}}\cdot\vr{x}' = \transpose{\vr{c}_{S\cup T}}\cdot\vr{u}'
  %$.
\end{lemma}
  %
\begin{lemma}\label{clm:row opt sE < opt sET}
The optimum of $\transpose{\cU}$ is smaller than or equal to the optimum of $\transpose{\cU_{S\cup T}}$.
\end{lemma}
%
Before proving the above two lemmas, we finish the proof of \Cref{thm:col row approx} using them.
We argue that the equivalence between $\cU$ and $\cU_{S\cup T}$ follows from \Cref{clm:opt sE < opt sET,clm:opt sE > opt sET} and \Cref{thm:col fin opt}.
\Cref{clm:opt sE < opt sET,clm:opt sE > opt sET} imply $\cU$ and $\cU_{S\cup T}$ have the same optimum.
Hence we just need to show that if one of them have an optimal solution,
so does the other.
Since $\cU_{S\cup T}$ is a finite linear program,
it has an optimal solution whenever its optimum is finite.
The same is true for $\cU$ due to \Cref{thm:col fin opt}.
Since the optimums of $\cU$ and $\cU_{S\cup T}$ are the same,
this shows $\cU$ and $\cU_{S\cup T}$ are equivalent.

%Similarly, \Cref{clm:row opt sE < opt sET,clm:row opt sE > opt sET} imply $\transpose{\cU}$ and $\transpose{\cU_{S\cup T}}$ have the same optimum.
%Hence, to show that 
%
%
%Since $\cU_{S\cup T}$ and $\transpose{\cU_{S\cup T}}$ are finite linear programs,
%they have optimal solutions when their optimum is finite.
%The same is true for $\cU$ and $\transpose{\cU}$ due to \Cref{thm:col row duality,thm:col fin opt,thm:row fin opt}.
%Hence for proving the equivalences between $\cU$ and $\cU_{S\cup T}$, and between $\transpose{\cU}$ and $\transpose{\cU_{S\cup T}}$,
%we just have to show that $\cU$ and $\cU_{S\cup T}$ have the same optimum,
%and likewise for $\transpose{\cU}$ and $\transpose{\cU_{S\cup T}}$.
The equivalence between $\transpose{\cU}$ and $\transpose{\cU_{S\cup T}}$ follows from \Cref{clm:row opt sE > opt sET,clm:row opt sE < opt sET} and \Cref{thm:row fin opt} in a similar manner.



Now we prove \Cref{clm:opt sE > opt sET,clm:row opt sE < opt sET}
%
\begin{proof}[Proof of \Cref{clm:opt sE > opt sET}]
%
Pick a solution $\vr{u}$ of $\cU_{S\cup T}$.
Define $\vr{x}\in\flin{C}$ as
\[
\vr{x}(c) \defeq
\begin{cases}
\vr{u}(c) & \text{when $c\in C_{S\cup T}$}\\
0 & \text{when $c\in C\setminus C_{S\cup T}$.}
\end{cases}
\]
Then $\transpose{\vr{c}} \cdot \vr{x} = \transpose{\vr{c}_{S\cup T}} \cdot \vr{u}$.
We finish the proof by showing $\vr{x}$ is a solution of $\cU$.
Since $\vr{u}$ is a solution of $\cU_{S\cup T}$, we have $\vr{u}\geqslant\vr{0}$.
Hence we also have $\vr{x}\geqslant \vr{0}$.
We just have to show $\vr{A}\cdot\vr{x} \leqslant \vr{b}$.

We have $\dom{\vr{x}}\subseteq C_{S\cup T}$.
%The set $C_{S\cup T}$ is finite due to \ref{lem:supp count}.
\Cref{lem:fin vec supp} implies $\vr{x}$ is a finite vector supported by $S\cup T$.
The matrix $\vr{A}$ is supported by $S$ and is column-finite.
Hence $\vr{A}\cdot\vr{x}$ is supported by $S\cup T$,
and using \Cref{cor:col fin prod} we get that it is also a finite vector.
The vector $\vr{b}$ is a finite vector supported by $S$.
Hence, another application of \Cref{lem:fin vec supp} gives us
\[
\dom{\vr{A}\cdot\vr{x}},\dom{\vr{b}}\subseteq B_{S\cup T} \ .
\]
Therefore, we just have to prove that $(\vr{A}\cdot\vr{x})(b)\leqslant\vr{b}(b)$ for all $b\in B_{S\cup T}$.
For every $b\in B_{S\cup T}$ we have
\begin{align*}
(\vr{A}\cdot\vr{x})(b) & = (\vr{A}\cdot\vr{x})_{S\cup T}(b) \\
& = (\vr{A}_{S\cup T}\cdot\vr{x}_{S\cup T})(b) && \text{(\Cref{lem:mat prod fin})}
\\
& = (\vr{A}_{S\cup T}\cdot\vr{u})(b) && \text{(by definition of $\vr{x}$)} \\
& \leqslant \vr{b}_{S\cup T}(b) \\
& = \vr{b}(b) \ .
\end{align*}
%
This finishes the proof.
%
\end{proof}
%
%
\begin{proof}[Proof of \Cref{clm:row opt sE < opt sET}]
%
Recall \Cref{cor:row fin solv} which says that $\transpose{\cU}$ and
$\transpose{\orbSum{S}(\cU)}$ are equivalent.
Hence to show that the optimum of $\transpose{\cU}$ is smaller than or equal to the optimum of $\transpose{\cU_{S\cup T}}$,
it is enough to prove that the optimum of $\transpose{\orbSum{S}(\cU)}$ is smaller than the latter.

Pick an arbitrary solution $\vr{v}$ of $\transpose{\cU_{S\cup T}}$.
We will construct a solution $\vr{z}$ of $\transpose{\orbSum{S}(\cU)}$ such that
\[
\transpose{\orbSum{S}(\vr{b})} \cdot \vr{z} = \transpose{\vr{b}_{S\cup T}} \cdot \vr{v} \ .
\]
Extend $\vr{v}$ to a $\vr{y} : B\to\R$ by assigning $0$ outside $B_{S\cup T}$:
\[
\vr{y}(b) \defeq
\begin{cases}
\vr{v}(b) & \text{if $b\in B_{S\cup T}$}\\
0          & \text{otherwise.}%when $b\in B\setminus B_{S\cup T}$.}\\
\end{cases}
\]
%
We show $\vr{z} = \orbsmt{S}{T}(\vr{y})$ satisfies our requirement.
%
Since $\vr{A}$ is a column-finite matrix,
$\transpose{\vr{A}}\cdot\vr{y}$ is well-defined (\Cref{cor:col fin prod}).
Define $\vr{c}' : C\to\R$ as
\[
\vr{c}'(c) =
\begin{cases}
\vr{c}(c)                       & \text{if $c\in C_{S\cup T}$}\\
(\transpose{\vr{A}}\cdot\vr{y})(c)  & \text{otherwise.}%when $c\in C\setminus C_{S\cup T}$.}\\
\end{cases}
\]
%
\begin{claim}\label{clm:c' constraint}
$\transpose{\vr{A}}\cdot\vr{y} \geqslant \vr{c}'$
\end{claim}
%
\begin{claim}\label{clm:orbsum c c'}
$\transpose{\orbsmt{S}{T}(\vr{c}')} = \orbSum{S}(\transpose{\vr{c}})$.
\end{claim}
%
Before proving the above claims we show that they imply $\orbsmt{S}{T}(\vr{y})$ is a solution of $\transpose{\orbSum{S}(\cU)}$ such that
\[
\transpose{\orbSum{S}(\vr{b})} \cdot \orbsmt{S}{T}(\vr{y}) = \transpose{\vr{b}_{S\cup T}} \cdot \vr{v} \ .
\]
%
The vector $\vr{v}$ is non-negative and hence $\vr{y}$ is also so.
Using the definition of $\orbsmt{S}{T}$ we conclude $\orbsmt{S}{T}(\vr{y})$ is non-negative as well.
Now
\begin{align*}
&& &\ \transpose{\orbSum{S}(\vr{A})}\cdot\orbsmt{S}{T}(\vr{y})
& \\
&&= &\ \orbsmt{S}{T}(\transpose{\vr{A}}\cdot \vr{y})& \text{\Cref{lem:orbsmt mat vec}} \\
&&\geqslant\ & \orbsmt{S}{T}(\vr{c}') &\text{\Cref{lem:orbsmt lin supp} and \Cref{clm:c' constraint}}\\
&&= &\ \transpose{\orbSum{S}(\transpose{\vr{c}})} &\text{\Cref{clm:orbsum c c'}} & .
\end{align*}
%
%It remains to show $\transpose{\orbSum{S}(\vr{b})} \cdot \orbsmt{S}{T}(\vr{y}) = \transpose{\vr{b}_{S\cup T}} \cdot \vr{v}$.
Again using \Cref{lem:orbsmt mat vec} we get
\[
\transpose{\orbSum{S}(\vr{b})} \cdot \orbsmt{S}{T}(\vr{y}) =
\transpose{\vr{b}} \cdot \vr{y} \ .
\]
%
Since $\vr{y}$ is $0$ outside $B_{S\cup T}$ and agrees with $\vr{v}$ inside $B_{S\cup T}$ we have
\[
\transpose{\vr{b}} \cdot \vr{y} =
\sum_{b\in B}\vr{b}(b)\cdot\vr{y}(b) =
\sum_{b\in B_{S\cup T}}\vr{b}_{S\cup T}(b)\cdot\vr{v}(b) =
\transpose{\vr{b}_{S\cup T}} \cdot \vr{v} \ .
\]
As a consequence we get
\[
\transpose{\orbSum{S}(\vr{b})} \cdot \orbsmt{S}{T}(\vr{y}) = \transpose{\vr{b}_{S\cup T}} \cdot \vr{v} \ .
\]
%
This finishes the proof of \Cref{clm:row opt sE < opt sET} modulo the proofs of \Cref{clm:c' constraint,clm:orbsum c c'},
which we do now.
%
\begin{claimproof}[Proof of \Cref{clm:c' constraint}]
%
We have to show that for $c\in C$, we have $\vr{c}'(c) \leqslant (\vr{A}\cdot\vr{y})(c)$.
For $c\in C\setminus C_{S\cup T}$,
it follows from the definition of the $\vr{c}'$.
We now show it for $c\in C_{S\cup T}$.

Pick $c\in C_{S\cup T}$.
Using \Cref{lem:fin vec supp} we conclude $\vr{y}$ is a finite vector supported by $S\cup T$.
The matrix $\vr{A}$ and  is also supported by $S$.
We have,
\begin{align*}
(\vr{A}\cdot\vr{y})(c) &=
(\vr{A}\cdot\vr{y})_{S\cup T}(c) \\
& = (\vr{A}_{S\cup T}\cdot\vr{y}_{S\cup T})(c) && \text{(\Cref{lem:mat prod fin})}\\
& = (\vr{A}_{S\cup T}\cdot\vr{v})(c) && \text{(by definition of $\vr{y}$)} \\
& \geqslant \vr{c}_{S\cup T}(c) && \text{($\vr{v}$ is a solution of $\transpose{\cU_{S\cup T}}$)} \\
& = \vr{c}(c) \ .
\end{align*}
This finishes the proof.\end{claimproof}
%
\begin{claimproof}[Proof of \Cref{clm:orbsum c c'}]
The vectors $\vr{c}$ and $\vr{c}'$ agree on $C_{S\cup T}$.
Since the function $\orbsmt{S}{T}$ only looks at values inside $C_{S\cup T}$,
we have
\[
\orbsmt{S}{T}(\vr{c}') = \orbsmt{S}{T}(\vr{c})  \ .
\]
The vector $\vr{c}$ is supported by $S$.
Hence \Cref{lem:orbsmt orbsum} implies
\[
\transpose{\orbsmt{S}{T}(\vr{c})} = \orbSum{S}(\transpose{\vr{c}}) \ .
\]
Combining the above two equalities we get $\transpose{\orbsmt{S}{T}(\vr{c}')} = \orbSum{S}(\transpose{\vr{c}})$.
\end{claimproof}
%
With the proofs of \Cref{clm:c' constraint,clm:orbsum c c'},
the proof of \Cref{clm:row opt sE < opt sET} is also complete.
%
\end{proof}
%
Now that we have proven \Cref{clm:opt sE > opt sET,clm:row opt sE < opt sET},
the proof of \Cref{thm:col row approx} is also finished.
\hfill\qedsymbol
%
%\section{Duality with atoms other than equality atoms}
%%
%\arka{Ignore this section}
%
%Although we were interested in duality between orbit-finite linear programs with \emph{equality atoms} (\Cref{rem:other atoms}),
%some of the results extend to other atomic structures as well.
%In particular, weak duality (\Cref{thm:weak o.f. duality}) also holds for orbit-finite linear programs defined with oligomorphic atoms.
%We briefly describe how this can be done.
%
%
%\arka{start with extending definitions and simple facts }
%%
%\begin{definition}\label{def:oligo}
%A structure $\X$ is called \emph'def{oligomorphic} if for any $n\in\N$ the number of equivariant orbits in $\otufrom{X}{n}$ is finite. 
%\end{definition}
%%
%\begin{example}
%Recall the examples of atoms given in \Cref{rem:other atoms}.
%Equality atoms, ordered atoms, bit-vector atoms and graph atoms are oligomorphic by virtue of being homogeneous (\cite[Chapter 7]{atombook}).%(\cite{homsurvey}).
%The structure $(\Z,+)$ is not oligomorphic since $\otufrom{\Z}{2}$ splits into infinitely many equivariant orbits of the form $\setof{(n,n+k)}{n\in\Z}$ for $k\in\Z$.
%\end{example}
%%
%%In \Cref{def:oligo}, instead of taking equivariant orbits,
%%we could have taken $S$-orbits for any $S\subseteqfin\X$.
%%%
%%\arka{maybe the lemma can be skipped}
%%%
%%\begin{lemma}[\mbox{\cite[Lemma 3.19]{atombook}}]\label{lem:oligo S}
%%For an oligomorphic structure $\X$ and finite subset $S\subseteqfin \X$ the number of $S$-orbits of $\X^n$ is also finite.
%%\end{lemma}
%%
%In \Cref{rem:other atoms} we showed how the definition of orbit-finiteness can be extended to an arbitrary structure of atoms.
%For oligomorphic atoms it extends particularly well
%since for these atoms orbit-finiteness is independent of orbit-refinement by increasing the size of the support.
%%
%\begin{lemma}[\protect{\cite[Theorem 3.16]{atombook}}]\label{lem:oligo ind supp}
%For any oligomorphic structure $\X$ and finite subsets $S\subseteq T\subseteqfin \X$,
%every $S$-orbit is a finite union of $T$-orbits.
%\end{lemma}
%%
%\fixed{For the rest of this section we fix an arbitrary oligomorphic structure $\X$.}
%Recall that a vector/function is viewed as a set of (input, output) pairs and is called orbit-finite if the set of pairs is orbit-finite.
%A consequence of the above theorem is the following lemma:
%\begin{lemma}
%Orbit-finite vectors are closed under addition.
%\end{lemma}
%\begin{proof}
%Pick an orbit-finite set $B$ and orbit-finite vectors $\vr{u}$ and $\vr{v}$ in $\lin{B}$ respectively supported by finite subsets $S$ and $T$ of $\X$.
%%
%\begin{claim}
%Both $S$ and $T$ supports $B$.
%\end{claim}
%%
%\begin{claimproof}
%\arka{TODO}
%\end{claimproof}
%%
%%\Cref{lem:fun supp dom} extends to this general setting, which means both $S$ and $T$ has to support $B$.
%Since $\X$ is oligomorphic, by \Cref{lem:oligo ind supp} $B$ is a finite union of $(S{\cup}T)$-orbits.
%The vector $\vr{u} + \vr{v}$ is supported by $S{\cup}T$ and can be written as the orbit-finite set of pairs
%\[
%\setof{ (\vr{u} + \vr{v})(O)}{O\in\orbits[S\cup T]{B}}
%\]
%where $(\vr{u} + \vr{v})(O)$ is the value of the vector $(\vr{u} + \vr{v})$ at any(every) element of $O$.
%\end{proof}
%This seemingly simple fact may not be true for atoms that are not oligomorphic.
%%
%\begin{example}
%Consider the structure $(\R^2, (R_r)_{r\in\R})$,
%where for $r\in\R\setminus\set{0}$ the relation $R_r$ is a ternary relation defined as
%\[
%R_r(\a,\b,\g) \quad\defiff\quad \frac{\norm{\a - \g}}{\norm{\b - \g}} = r \ .
%\] 
%It is not oligomorphic since for each $r\in\R\setminus\set{0}$ the set of triples
%\[
%\setof{\a\b\g\in(\R^2)^{(3)}}{R_r(\a,\b,\g)}
%\]
%can intersect at most one equivariant orbit.
%
%Pick two distinct points $\a$ and $\b$ in $\R^2$.
%We show the vectors $\idvec{\R^2\setminus\set{\a}}$ and $2\cdot\idvec{\R^2\setminus\set{\b}}$ are orbit-finite,
%however their sum is not.
%
%We prove $\idvec{\R^2\setminus\set{\a}}$ is orbit-finite.
%The proof that $\idvec{\R^2\setminus\set{\b}}$ is orbit-finite can be done symmetrically.
%It is enough to show that $\R^2\setminus\set{\a}$ is an orbit-finite set,
%in fact we show that it is a single orbit.
%\Wlog{} we assume $\a$ is the origin.
%Any rotation of the plane around the origin preserves the relations $R_r$ and also the set $\R^2 \setminus \set{\a}$.
%So does scaling the plane by any non-zero constant.
%But for any two points $\g,\delta\neq\a$,
%there exists a rotation $\pi$ and scaling $\sigma$ such that $(\sigma\circ\pi)(\g) = \delta$.
%Which means the set $\R^2 \setminus \set{\a}$ is a single orbit.
%
%Now we show that the vector
%\[
%\vr{x} =
%\idvec{\R^2\setminus\set{\a}} +  2\cdot\idvec{\R^2\setminus\set{\b}}
%=
%3\cdot\idvec{\R^2\setminus\set{\a,\b}} +  \idvec{\set{\a}} + 2\cdot\idvec{\set{\b}}
%\]
%is not orbit-finite.
%Assume otherwise, say $S\subseteq \R^2$ be such that the vector/function $\vr{x}$,
%seen as a set of (input, output) pairs is a finite union of $S$-orbits;
%equivalently $S$ supports $\vr{x}$ and $\R^2$ is a finite union of $S$-orbits.
%Any automorphism $\pi\in\aut[S]{\R^2}$ must fix both $\a$ and $\b$ since $\vr{x}(\a)\neq\vr{x}(\b) \neq \vr{x}(\g)$ for any $\g\in\R^2\setminus\set{\a,\b}$.
%But then, for each $r\in\R\setminus\set{0}$,
%the set
%\[
%\setof{\g\in\R^2\setminus\set{\a,\b}}{
%\frac{\norm{\a - \g}}{\norm{\b - \g}} = r}
%\]
%is non-empty, and can intersect at most one $S$-orbit.
%This contradicts the fact that $\R^2$ is a finite union of $S$-orbits.
%\end{example}
%
%
%
%
%
%\arka{as a consequence orbit-finite vectors are closed under addition }
%
%\arka{maybe fix a structure}
%
%
%\arka{``to avoid complications with representation'' in this section we work with orbit-finite sets which are straight.
%define orbit-finite linear programs indexed with just straight sets,}
%
%\arka{not necessary. Just write the theorem that orbit-finiteness is independent under increasing support}
%
%\section{Cones and Solution Sets}
%
%\pagebreak
%
%\huge
%\arka{Experimental Part}
%\normalsize
%%
%\begin{definition}\label{def:fsum}
%The subspace $\fsum{B}\subseteq (B\to \R)$ consists of all the vectors $\vr{x}:B\to\R$ such that $\sum_{b\in B} |\vr{x}(b)|$ is finite.
%\end{definition}
%%
%\arka{Problem:to define inner product we need to fix an ordering of the tuples.
%This would create lot of unnecessary complications}
%%
%\begin{theorem}\label{thm:col fin opt 1}
%Consider an $S$-supported column-finite linear program of atom dimension $d$.
%For any $T\subseteqfin(\A\setminus S)$ of size at least $d$:
%\begin{enumerate}
%\item The optimum of the linear program does not change if we restrict to solutions which are finite and supported by $S\cup T$.
%\item The optimum of the linear program does not change if we allow orbit-infinite solutions of bounded norm.
%\item If the optimum is finite then,
%it has an optimal solution supported by ${S\cup T}$.
%\end{enumerate}
%\end{theorem}
%
%\huge
%\arka{OLD PART}
%\normalsize
%
%\begin{definition}\label{def:orbdis row}
%Define $\orbrow{S} : \R^{\orbits[S]{C}}\to\lin{C}$ as
%\[
%\orbrow{S}(\vr{x}) : c \mapsto \vr{x}(\orbit[S]{c}) 
%\]
%\end{definition}
%%
%\begin{lemma}
%For any $\vr{x}\in\R^{\orbits{C}}$ the vector $\orbrow{S}(\vr{x})$ is supported by $S$.
%\end{lemma}
%%
%\begin{proof}[Proof of \Cref{lem:orbrow sol}]
%\arka{TODO}
%\end{proof}
%%
%Finite linear programs with finite optimum admit optimal solutions \footnote{\arka{same citation as before}}.
%Let $\vr{z}$ be an optimal solution of \eqref{eq:duality proof orbsum dual}.
%Then \Cref{lem:orbrow sol} implies $\orbrow{S}(\vr{z})$ is a $S$-supported solution of \eqref{eq:duality proof dual} with
%$\transpose{\vr{c}}\cdot\orbrow{S}(\vr{z}) = r$.
%%
%\arka{note here that the lemmas also prove $S$-supp of row-finite.
%The remaining case is when the column-finite is not solvable and the row-finite is solvable} 
%%
%\section{Unused lemmas and theorems on $\orbsum{S}$}
%%
%\begin{definition}\label{def:cone}
%For $G\subseteq \flin{B}$ define
%\[
%\cone(G) \defeq
%\setof{\sum_{i = 1}^n r_i \cdot\vr{g}_i}{r_1,\dots,r_n\geqslant 0,\ \vr{g}_1,\dots,\vr{g}_n\in G} \ .
%\] 
%\end{definition}
%%
%%For any  $G\subseteq \flin{B}$ and $\vr{t}\in\flin{B}$,
%%if $\vr{t}\in\cone(G)$ then $\orbsum{S}(\vr{t})\in\cone(\orbsum{S}(G))$.
%%Indeed, if there exists $r_1,\dots,r_n \in \R_+$ and $\vr{g}_1,\dots,\vr{g}_n \in G$ such that
%%\[
%%\vr{t} = \sum_{i = 1}^n r_i \cdot \vr{g}_i \ .
%%\]
%%By linearity of $\orbsum{S}$ (\Cref{lem:orbsum lin S supp})
%%\[
%%\orbsum{S}(\vr{t}) = \sum_{i = 1}^n r_i \cdot \orbsum{S}(\vr{g}_i)
%%\in \cone(\orbsum{S}(G)) \ .
%%\]
%%%
%%In fact, both directions hold when $\vr{t}$ and $G$ is supported by $S$.
%%
%\begin{theorem}\label{thm:orbsum cone}
%For any set of vectors $G\subseteq \flin{B}$ and vector $\vr{t}\in\flin{B}$,
%both supported by $S$, we have
%\[
%\vr{t} \in \cone(G)
%\quad\iff\quad
%\orbsum{S}(\vr{t}) \in \cone(\orbsum{S}(G))
%\]
%\end{theorem}
%%
%%We have already shown the easier direction of the Theorem.
%%The harder direction will follow from the following Lemma.
%The proof of this theorem uses the following lemma,
%which is also useful on its own.
%%
%\begin{lemma}\label{lem:orbsum cone}
%Consider finitely many vectors $\vr{g}_1,\dots,\vr{g}_n \in \flin{B}$.
%Let $\vr{t}\in\flin{B}$ be a vector supported by $S$.
%Let $T\subseteqfin (\A\setminus S)$ be a finite subset of atoms such that $S\cup T$ supports $\vr{g}_1,\dots,\vr{g}_n$.
%If there exists $r_1,\dots,r_n\geqslant 0$ such that
%\[
%\orbsum{S}(\vr{t}) =
%\sum_{i=1}^n r_i \cdot \orbsum{S}(\vr{g}_i)
%\]
%then,
%\begin{equation}\label{eq:orbsum cone}
%\vr{t} = \frac{1}{|\aut{T}|}
%\left(
%\sum_{\pi\in\aut{T}}
%\pi(\vr{t}')
%\right)
%\end{equation}
%where
%\[
%\vr{t}' = \sum_{i = 1} r_i \cdot \vr{g}_i
%\]
%\end{lemma}
%%
%Before proving \Cref{lem:orbsum cone} we show how it finishes the proof of \Cref{thm:orbsum cone}.
%%
%\begin{proof}[Proof of \Cref{thm:orbsum cone}]\label{proof:thm:orbsum cone}
%$(\implies)$
%Assume $\vr{t}\in\cone(G)$.
%Then there exists $r_1,\dots,r_n \geqslant 0$ and $\vr{g}_1,\dots,\vr{g}_n \in G$ such that
%\[
%\vr{t} = \sum_{i = 1}^n r_i \cdot \vr{g}_i
%\]
%By linearity of $\orbsum{S}$ (\Cref{lem:orbsum lin S supp})
%\[
%\orbsum{S}(\vr{t}) = \sum_{i = 1}^n r_i \cdot \orbsum{S}(\vr{g}_i)
%\in \cone(\orbsum{S}(G))
%\]
%\smallskip
%
%\noindent
%$(\impliedby)$
%Say $\orbsum{S}(\vr{t}) \in \orbsum{S}(G)$.
%Then there exists $r_1,\dots,r_n \geqslant 0$ and $\vr{g}_1,\dots,\vr{g}_n \in G$ such that
%\[
%\orbsum{S}(\vr{t}) = \sum_{i = 1}^n r_i \cdot \orbsum{S}(\vr{g}_i) \ .
%\]
%Let $\vr{t}' = \sum_{i = 1}^n r_i \cdot\vr{g}_i$.
%\Cref{lem:orbsum cone} says
%\[
%\vr{t} = \frac{1}{|\aut{T}|}
%\left(
%\sum_{\pi\in\aut{T}}
%\pi(\vr{t}')
%\right)
%\]
%Since $r_i\geqslant 0$ and $g_i\in G$ we get $\vr{t}'\in\cone(G)$.
%$G$ is supported by $S$.
%Which means $\cone(G)$ is also supported by $S$.
%\arka{should this be expanded?}.
%Since $T\cap S =\emptyset$, we have $\aut{T}\subseteq\aut[S]{\A}$.
%Hence $\pi(\vr{t}')\in\cone(G)$ for every $\pi\in\aut{T}$.
%Since $\cone(G)$ is a cone and
%$\vr{t}$ is a convex combination of the finite set of vectors
%\[
%\setof{\pi(\vr{t}')}{\pi\in\aut{T}}
%\subseteq \cone(G)
%\]
%it must be in $\cone(G)$ as well.
%\arka{should there be a lemma saying cones are closed under convex combination}
%\end{proof}
%%
%\begin{proof}
%[Proof of \Cref{lem:orbsum cone}]
%\label{proof:lem:orbsum cone}
%We start with the following easy claim.
%%
%\begin{claim}\label{clm:t prime t}
%$\orbsum{S}(\vr{t}') = \orbsum{S}(\vr{t})$.
%\end{claim}
%%
%\begin{claimproof}
%By definition of $\vr{t}'$ and linearity of $\orbsum{S}$ (\Cref{lem:orbsum lin S supp})
%\[
%\orbsum{S}(\vr{t}')
%= \sum_{i = 1}^n r_i\cdot\orbsum{S}(\vr{g}_i)
%= \orbsum{S}(\vr{t})
%\]
%\end{claimproof}
%%
%Let
%\[
%\vr{t}'' = \frac{1}{|\aut{T}|}
%\left(
%\sum_{\pi\in\aut{T}}
%\pi(\vr{t}')
%\right)
%\]
%We show $\vr{t}'' = \vr{t}$.
%Choose arbitrary $b \in B$.
%We prove $\vr{t}''(b) = \vr{t}(b)$.
%We split into two cases.
%%
%\paragraph{(Case 1: $b$ is not supported by $S\cup T$)}
%
%The vector $\vr{t}$ is supported by $S$.
%\Cref{lem:fin vec supp} implies $\vr{t}(b) = 0$.
%We show $\vr{t}''(b) = 0$ as well.
%\Cref{lem:add supp} implies the vector $\vr{t}'$ is supported by $S\cup T$.
%Using \Cref{lem:supp fun equiv} we conclude $\pi(\vr{t}')$ is supported by $S\cup T$ for every $\pi\in\aut{T}$.
%\Cref{lem:fin vec supp} implies $(\pi(\vr{t}'))(b) = 0$ for every $\pi\in\aut{T}$.
%Hence $\vr{t}''(b) = 0$.
%%
%\paragraph{(Case 2: $b$ is supported by $S\cup T$)}
%
%Let $D_b = \setof{\pi(b)}{\pi\in\aut{T}}$.
%$\aut{T}$ is a group which acts transitively on $D_b$.
%\arka{should this be expanded?}
%Let $d_b = |\setof{\pi\in\aut{T}}{\pi(b) = b}|$.
%\begin{claim}\label{clm:stab}
%For every $b' \in D_b$
%\[
%|\setof{\pi\in\aut{T}}{\pi(b) = b'}|
%=
%d_b
%\]
%\end{claim}
%%
%\begin{claimproof}
%Pick arbitrary $b'\in D_b$.
%There exists $\sigma\in\aut{T}$ such that $\sigma(b) = b'$.
%Then $\pi\mapsto \sigma^{-1}\circ\pi$ is a bijection from the set
%\[
%\setof{\pi\in\aut{T}}{\pi(b) = b'}
%\] to the set
%\[
%\setof{\pi\in\aut{T}}{\pi(b) = b}
%\]
%with inverse $\pi\mapsto \sigma\circ\pi$.
%\end{claimproof}
%%
%\begin{claim}\label{clm:zero outside Db}
%For any $b'\in\orbit[S]{b}$, if $\vr{t}'(b')\neq 0$ then $b'\in D_b$.
%\end{claim}
%%
%\begin{claimproof}
%Pick $b'\in\orbit[S]{b}$ such that $\vr{t}'(b')\neq 0$.
%The vector $\vr{t}'$ is supported by $S\cup T$.
%\Cref{lem:fin vec supp} implies $b'$ is supported by $S\cup T$.
%Using \Cref{lem:same supp orb} we conclude that there exists $\pi\in\aut{T}$ such that $\pi(b) = b'$.
%In other words $\pi(b) = b'$. 
%\end{claimproof}
%%
%We have the following sequence of equations.
%\[
%\begin{aligned}
%& &&  \left(
%\sum_{\pi \in \aut{T}} \pi(\vr{t}')
%\right)(b) &\\
%& = &&
%\sum_{\pi \in \aut{T}} \pi(\vr{t}')(b) & \\
%& = &&
%\sum_{\pi \in \aut{T}} \vr{t}'(\pi^{-1}(b))
%& \quad(\text{\Cref{lem:perm fun}})\\
%& = && \sum_{b'\in D_b}
%\left|
%\setof{\pi\in\aut{T}}{\pi^{-1}(b) = b'}
%\right| \cdot\vr{t}'(b') & \\
%& = && \sum_{b'\in D_b} \left|
%\setof{\pi\in\aut{T}}{\pi(b) = b'}
%\right| \cdot\vr{t}'(b')
%& \\
%& = && \sum_{b' \in D_b} d_b\cdot \vr{t}'(b') &
%\quad(\text{\Cref{clm:stab}})\\
%& = && d_b \cdot (\orbsum{S}(\vr{t}'))(\orbit[S]{b}) &\quad(\text{\Cref{clm:zero outside Db}})\\
%& = && d_b \cdot (\orbsum{S}(\vr{t}))(\orbit[S]{b})
%& \quad(\text{\Cref{clm:t prime t}})
%\end{aligned}
%\]
%To finish the proof now need to show
%\[
%d_b \cdot (\orbsum{S}(\vr{t}))(\orbit[S]{b}) = |\aut{T}|\cdot \vr{t}(b)
%\]
%for all $b\in B$.
%We split the proof of this into two subcases.
%
%\paragraph{Case 2.1 ($b$ is supported by $S$)}
%The vector $\vr{t}$ is supported by $S$.
%Hence $(\orbsum{S}(\vr{t}))(\orbit[S]{b}) = (\orbsum{S}(\vr{t}))(\{b\}) = \vr{t}(b)$, and $d_b = |\aut{T}|$.
%Hence
%\[
%d_b \cdot (\orbsum{S}(\vr{t}))(\orbit[S]{b}) = |\aut{T}|\cdot \vr{t}(b)
%\]
%%
%\paragraph{Case 2.2 ($b$ is not supported by $S$)}
%Since $\vr{t}$ is supported by $S$ and is a finite vector,
%\Cref{lem:fin vec supp} implies $\vr{t}(b') = 0$ for all $b'\in D_b$.
%Hence
%\[
%d_b \cdot (\orbsum{S}(\vr{t}))(\orbit[S]{b}) = 0 = |\aut{T}|\cdot \vr{t}(b)
%\]
%\end{proof}
%%
%\begin{theorem}\label{thm:orbsum ineq}
%Consider a column-finite matrix $\vr{A}\in\lin{B{\times} C}$ and vector $\vr{b}\in\flin{B}$,
%both supported by $S$.
%Let
%\[
%d = \max\{\text{atom-dimension of }\vr{A},
%          \text{atom-dimension of }\vr{b}\}
%\]
%The following are equivalent:
%\begin{enumerate}
%\item There exists a non-negative vector $\vr{x}\in\flin{C}$ such that 
%      $\vr{A}\cdot\vr{x}\leqslant\vr{b}$.
%\item There exists a non-negative vector $\vr{z}\in\R^{\orbits[S]{B}}$ such that
%      $\orbSum{S}(\vr{A})\cdot\vr{z}\leqslant\orbSum{S}(\vr{B})$.
%\item For any $T\subseteqfin(\A\setminus S)$  of size at least $d$,
%      there exists a non-negative vector $\vr{x}\in\flin{C}$
%      supported by $S\cup T$ such that 
%      $\vr{A}\cdot\vr{x}\leqslant\vr{b}$.
%\end{enumerate}
%\end{theorem}
%%
%\begin{proof}[Proof of \Cref{thm:orbsum ineq}]
%It is trivial to show that (3)$\implies$(1).
%We show \\ (1)$\implies$(2) and (2)$\implies$(3).
%
%First we prove (1)$\implies$(2).
%Say there exists a non-negative vector $\vr{x}$ in $\flin{C}$ such that 
%$\vr{A}\cdot\vr{x}\leqslant\vr{b}$.
%We need to find a non-negative vector $\vr{z}\in\R^{\orbits[S]{B}}$ such that
%$\orbSum{S}(\vr{A})\cdot\vr{z}\leqslant\orbSum{S}(\vr{B})$.
%We show $\vr{z} = \orbsum{S}(\vr{x})$ does the job.
%\Cref{cor:orbsum non-neg} implies $\orbsum{S}(\vr{x})$ is non-negative.
%Hence, we just need to show
%$\orbSum{S}(\vr{A})\cdot\orbsum{S}(\vr{x})\leqslant\orbSum{S}(\vr{B})$.
%\Cref{lem:orbsum mat vec} implies
%$\orbSum{S}(\vr{A})\cdot\orbsum{S}(\vr{x}) = \orbsum{S}(\vr{A}\cdot\vr{x})$.
%Since $\vr{A}\cdot\vr{x}\leqslant\vr{b}$,
%\Cref{lem:orbsum ord} implies
%$\orbsum{S}(\vr{A}\cdot\vr{x}) \leqslant \orbSum{S}(\vr{B})$.
%Hence $\orbSum{S}(\vr{A})\cdot\orbsum{S}(\vr{x})\leqslant\orbSum{S}(\vr{B})$.
%This finishes the proof.
%
%Now we prove (2)$\implies$(3).
%Say there exists a non-negative vector ${\vr{z}\in\R^{\orbits[S]{C}}}$ such that
%$\orbSum{S}(\vr{A})\cdot\vr{z}\leqslant\orbSum{S}(\vr{B})$.
%We need to find a non-negative vector $\vr{x}\in\flin{C}$ such that $\vr{A}\cdot\vr{x}\leqslant\vr{b}$.
%Let $\vr{d} = \orbSum{S}(\vr{b}) - \orbSum{S}(\vr{A})\cdot\vr{z}$.
%Since $\orbSum{S}(\vr{A})\cdot\vr{z}\leqslant\orbSum{S}(\vr{B})$,
%we get $\vr{d}\geqslant\vr{0}$.
%We have
%\begin{equation}\label{eq:thm:orbsum ineq:b A d}
%\orbSum{S}(\vr{b}) = \orbSum{S}(\vr{A})\cdot\vr{z} + \vr{d}
%\end{equation}
%Expanding the expressions in the RHS
%\begin{equation}\label{eq:thm:orbsum ineq:b sum E D}
%\orbSum{S}(\vr{b}) = \sum_{E\in\orbits[S]{C}}\vr{z}(E)\cdot\orbSum{S}(\vr{A})(-,E) +
%                  \sum_{D\in\orbits[S]{B}}\vr{d}(D)\cdot\idvec{D}
%\end{equation}
%For every $E\in\orbits[S]{C}$ pick $c_E\in E$ such that
%\[
%\supp{c_E}\subseteq S\cup T
%\]
%Similarly, for every $D\in\orbits[S]{B}$ pick $b_D\in D$ such that
%\[
%\supp{b_D}\subseteq S\cup T
%\]
%\Cref{lem:pick elem supp} guarantees existence of such elements.
%By \Cref{def:orbsum matrix}, for every $E\in\orbits[S]{B}$ we have
%$\orbSum{S}(\vr{A})(-,E) = \orbsum{S}(\vr{A}(-,c_E))$.
%Applying \Cref{def:orbsum vector} we get $\orbsum{S}(\idvec{b_D})=\idvec{D}$ for every $D\in\orbits[S]{B}$.
%Rewriting the RHS of equation \eqref{eq:thm:orbsum ineq:b sum E D} using these facts we get
%\begin{equation}\label{eq:thm:orbsum ineq:orbsum b cE bD}
%\begin{aligned}
% & \orbsum{S}(\vr{B})
% & = & \sum_{E\in\orbits[S]{C}}\vr{z}(E)\cdot\orbsum{S}(\vr{A}(-,c_E)) \\
% & & & \hspace{70pt} + \\
% & & &\sum_{D\in\orbits[S]{B}}\vr{d}(D)\cdot\orbsum{S}(\idvec{b_D})
%\end{aligned}
%\end{equation}
%Define $\vr{x}'\in\flin{C}$ as
%\[
%\vr{x}' = \sum_{E\in\orbits[S]{C}} \vr{z}(E)\cdot c_E
%\]
%Define $\vr{y}\in\flin{B}$ as
%\[
%\vr{y} = \sum_{D\in\orbits[S]{B}} \vr{d}(D)\cdot b_D
%\]
%Define $\vr{b}' = \vr{A}\cdot\vr{x}' + \vr{y}$.
%Expanding the expression $\vr{A}\cdot\vr{x}'$ and $\vr{y}$ using the definition of $\vr{x}'$ and $\vr{y}$
%\begin{equation}\label{eq:thm:orbsum ineq:b' cE bD}
%\vr{b}' = \sum_{E\in\orbits[S]{C}}\vr{z}(E)\cdot\vr{A}(-,c_E) +
%          \sum_{D\in\orbits[S]{B}}\vr{d}(D)\cdot\idvec{b_D}
%\end{equation}
%Applying \Cref{lem:orbsum cone} with \eqref{eq:thm:orbsum ineq:orbsum b cE bD} and \eqref{eq:thm:orbsum ineq:b' cE bD} we get
%\begin{equation}\label{eq:thm:orbsum ineq 5}
%\vr{b} = \frac{1}{d!}\cdot\left(\sum_{\pi\in\aut{T}} \pi(\vr{b}')\right)
%\end{equation}
%Expanding the expression $\sum_{\pi\in\aut{T}} \pi(\vr{b}')$
%using the definition of $\vr{b}'$
%\begin{equation}\label{eq:thm:orbsum ineq 6}
%\sum_{\pi\in\aut{T}} \pi(\vr{b}') =
%\sum_{\pi\in\aut{T}} \pi(\vr{A}\cdot\vr{x}' + \vr{y})
%\end{equation}
%Using \Cref{lem:mult equiv,lem:add equiv}
%\begin{equation}\label{eq:thm:orbsum ineq 7}
%\sum_{\pi\in\aut{T}} \pi(\vr{A}\cdot\vr{x}' + \vr{y}) =
%\sum_{\pi\in\aut{T}}
%\left(
%\pi(\vr{A})\cdot\pi(\vr{x}') + \pi(\vr{y})
%\right)
%\end{equation}
%The matrix $\vr{A}$ is supported by $S$.
%Hence $\pi(\vr{A}) = \vr{A}$ for every $\pi\in\aut{T}\subseteq\aut[S]{\A}$.
%\arka{define the $\aut{T}$ to be the shorthand for $\aut[\A\setminus T]{\A}$}
%Using this fact we get
%\begin{equation}\label{eq:thm:orbsum ineq 8}
%\begin{aligned}
%       & \sum_{\pi\in\aut{T}} (\pi(\vr{A})\cdot\pi(\vr{x}') + \pi(\vr{y})) \\
%       & \\
%=\quad & \vr{A}\cdot\left(
%         \sum_{\pi\in\aut{T}}\pi(\vr{x}')\right) +
%         \sum_{\pi\in\aut{T}} \pi(\vr{y})
%\end{aligned}
%\end{equation}
%Combining equations \eqref{eq:thm:orbsum ineq 6}, \eqref{eq:thm:orbsum ineq 7} and \eqref{eq:thm:orbsum ineq 8} we get
%\begin{equation}\label{eq:thm:orbsum ineq 9}
%\begin{aligned}
%       & \sum_{\pi\in\aut{T}} \pi(\vr{b}')
%=\quad & \vr{A}\cdot\left(
%         \sum_{\pi\in\aut{T}}\pi(\vr{x}')\right) +
%         \sum_{\pi\in\aut{T}} \pi(\vr{y})
%\end{aligned}
%\end{equation}
%Using equations \eqref{eq:thm:orbsum ineq 5} and \eqref{eq:thm:orbsum ineq 9} we get
%\begin{equation}\label{eq:thm:orbsum ineq:b x y}
%\vr{b} = 
%\vr{A}\cdot
%\left(\frac{1}{d!}\cdot\sum_{\pi\in\aut{T}}\pi(\vr{x}')\right) +
%\left(\frac{1}{d!}\cdot\sum_{\pi\in\aut{T}} \pi(\vr{y})\right)
%\end{equation}
%Define two vectors $\vr{x}$ and $\vr{y}$ as
%\[
%\vr{x} = \frac{1}{d!}\cdot\left(\sum_{\pi\in\aut{T}}\pi(\vr{x}')\right)
%\]
%and
%\[
%\vr{y} = \frac{1}{d!}\cdot\left(\sum_{\pi\in\aut{T}} \pi(\vr{y})\right)
%\]
%To finish the proof we show that $\vr{x}$ is a non-negative vector in $\flin{C}$ such that $\vr{A}\cdot\vr{x} \leqslant \vr{b}$ and $\supp{\vr{x}}\subseteq S\cup T$.
%Since $\vr{x}'$ is a non-negative vector in $\flin{C}$, $\pi(\vr{x})$ is a non-negative vector in $\flin{C}$ for every $\pi\in\aut{T}$.
%Hence $\vr{x}$ is also a non-negative vector in $\flin{C}$.
%Similarly, one can show $\vr{y}$ to be a non-negative vector in $\flin{B}$.
%Equation \eqref{eq:thm:orbsum ineq:b x y} implies $\vr{b} = \vr{A}\cdot\vr{x} + \vr{y}$.
%Since $\vr{y}$ is non-negative we get $\vr{A}\cdot\vr{x}\leqslant\vr{b}$.
%It only remains to prove that $\vr{x}$ is supported by $S\cup T$, which we do now.
%For every $E\in\orbits[S]{C}$ the element $c_E\in E$ is supported by $S\cup T$.
%The vector $\vr{x}$ is zero outside the finite set $\setof{c_E}{E\in\orbits[S]{C}}$.
%Using \Cref{lem:fin vec supp} we conclude $\vr{x}'$ is supported by $S\cup T$.
%\Cref{lem:perm supp mat} implies $\pi(\vr{x}')$ is also supported by $S\cup T$ for every $\pi\in\aut{T}$.
%Using \Cref{lem:add supp} we conclude $\vr{x}$ is also supported by $S\cup T$.
%This finishes the proof.
%\end{proof}
%%
%\section{Old part}
%
%We have the following computational problems regarding these linear programs.
%%
%\probOut{\colLinProg}{A column-finite linear program.}
%{Optimum of the linear program. If the optimum is finite, then\\
%& an optimal solution.}
%%
%\probOut{\rowLinProg}{A row-finite linear program.}
%{Optimum of the linear program. If the optimum is finite,
%then \\ & an optimal solution.}
%%
%For both of the above problems, we assume the input linear programs are given in standard representation.
%\arka{define standard representation at some place. Don't assume straightness in the standard representation}
%Notice that, as opposed to the computational problems related to orbit-finite linear programs (\linProg{} and \flinProg{}),
%in the above problems we also ask for an optimal solution.
%We have the following results regarding these problems.
%%
%\begin{theorem}\label{thm:col fin fadp}
%\textsc{Col-Fin-Max} is in \fadp.
%\end{theorem}
%%
%\begin{theorem}\label{thm:row fin fadp}
%\textsc{Row-Fin-Max} is in \fadp.
%\end{theorem}
%%
%\Cref{thm:col fin fadp} follows from \Cref{thm:col fin opt,thm:fLinProg fadp} and \footnote{\arka{appropriate theorem saying orbit-finite linear programs can be aproximated}}.
%Similarly \Cref{thm:row fin fadp} follows from \Cref{thm:row fin opt}, \Cref{cor:linProg fadp} and \footnote{\arka{appropriate theorem saying orbit-finite linear programs can be aproximated}}.
%However, the techniques we develope for proving \Cref{thm:col row duality},thm:col fin opt,thm:row fin opt} will lead to simpler proofs.
