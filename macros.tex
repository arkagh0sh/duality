% !TEX root = main.tex
%
%\usepackage{amssymb,amsmath,amsthm}
%\usepackage{mathrsfs}
%\usepackage[dvipsnames]{xcolor}
%\usepackage{enumitem}
\usepackage[shortlabels]{enumitem}
\usepackage{hyperref}
\usepackage{cleveref}
%\usepackage[hyperref,amsthm]{ntheorem} %use either this or ntheorem
\usepackage{thmtools}
\addtotheorempostheadhook[thm]{\crefalias{thmlisti}{thm}}
\addtotheorempostheadhook[lem]{\crefalias{thmlisti}{lem}}
%\usepackage{textcomp}
%\usepackage{nicematrix}
\usepackage{blkarray}
\usepackage{tabularx,lipsum,environ}
%\usepackage{ragged2e}
\usepackage{mathrsfs}
%\usepackage{xspace}
\usepackage[normalem]{ulem} % for strikethroughing text using \sout{}
\usepackage{color,soul} %for underlining
\setul{}{1pt}
\usepackage[all]{xy}
\usepackage{tikz}
\usetikzlibrary{arrows.meta,positioning,petri}
\usepackage{algorithm}
\usepackage[noend]{algpseudocode} %for algorithm environment
\usepackage{etoolbox}
\usepackage{chngcntr}
%\usepackage{todonotes}
\usepackage{leftindex}
\usepackage[symbols,nogroupskip,sort=none]{glossaries-extra}

\usepackage{tikz-cd}
\usepackage[nonewpage]{imakeidx}
%\usepackage[numbib,notindex]{tocbibind}

\usepackage{titletoc}
\usepackage[toctitles]{titlesec}

\allowdisplaybreaks

% For having only the figure number in caption
\usepackage{caption}
\DeclareCaptionFormat{empty}{#1}
\renewcommand*\figurename{Figure}
%
%
\newglossary[slg]{symbolslist}{syi}{syg}{List of commonly used symbols}
%
%
%\renewcommand{\emph}[1]{\textcolor{purple}{\textit{#1}}} %to make emphs more noticeable since we use them to define things outside definition environment
%
%\sethlcolor{olive}
\newcommand{\emphdef}[1]{\textcolor{purple}{\emph{#1}}}
\newcommand{\defind}[2]{\emphdef{#1}{\index[terms]{#2}}}
\newcommand{\defindS}[1]{\defind{#1}{#1}}
\newcommand{\defindKey}[2]{\defind{#1}{#2@#1}}
\newcommand{\defindSym}[2]{\emphdef{#1}{\index[symbols]{#2}}}
\newcommand{\defindSymS}[1]{\defindSym{#1}{#1}}
%\newcommand{\emphdef}[1]{\text{\hl{#1}}}
\newcommand{\orbsumlp}[1]{#1^{\Sigma}}
\colorlet{Mycolor1}{red!50!blue}
\newcommand{\indexcolor}[1]{\textcolor{Mycolor1}{#1}}
%
\usepackage[flushmargin,ragged]{footmisc} % this removes indentation from footnotes and makes them left aligned
%
%\usepackage{lineno}
%\linenumbers
\crefname{section}{§}{§§}
\Crefname{section}{§}{§§}
%
%
% Trying to add a line beside proofs
%
%\usepackage[most]{tcolorbox}
%\renewenvironment{proof}[1][Proof]
%{\begin{tcolorbox}[freelance,
%breakable,
%width=\dimexpr\textwidth+28pt\relax,
%before=\par\vspace{\bigskipamount}\noindent,
%colback=white,
%enlarge left by=-14pt,
%%overlay unbroken and first={
%%%  \node[
%%%  anchor=north east,
%%%  inner xsep=8pt,
%%%  xshift=8pt,
%%%  rounded corners=5pt,
%%%  font=\bfseries,
%%%  fill=white] at ([xshift=-0.2cm]frame.north west) (tit) {\strut Example:};
%%  \draw[
%%  line width=3pt,
%%  rounded corners=5pt,gray
%%  ] ([xshift=4pt]frame.north west) -- ([xshift=4pt]frame.south west);
%%},
%%overlay middle and last={
%%  \draw[
%%  line width=3pt,
%%  rounded corners=5pt,gray
%%  ] ([xshift=4pt]frame.north west) -- ([xshift=4pt]frame.south west);
%%},
%frame code={},
%interior code={},
%top=0pt,
%bottom=0pt,
%title=#1,
%borderline west={1pt}{0pt}{black},
%coltitle = black,
%opacityframe=0,
%opacityfill=0]
%}
%{\qed\end{tcolorbox}}
%
%
%%%%%%%%%%%%%%%%%%%%%%   symbols   %%%%%%%%%%%%%%%%%%%%%
%
\renewcommand{\a}{\alpha}
\renewcommand{\b}{\beta}
\newcommand{\g}{\gamma}
\newcommand{\A}{\mathbb{A}}
\newcommand{\cB}{\mathcal{B}}
\newcommand{\cC}{\mathcal{C}}
\newcommand{\N}{\mathbb{N}}
\newcommand{\Q}{\mathbb{Q}}
\newcommand{\Z}{\mathbb{Z}}
\newcommand{\R}{\mathbb{R}}
\newcommand{\F}{\mathbb{F}}
\newcommand{\X}{\mathbb{X}}
\newcommand{\sS}{\mathcal{S}}
\newcommand{\cU}{\mathcal{U}}
\newcommand{\W}{\mathbb{W}}
\newcommand{\id}{\textnormal{Id}}
\newcommand{\ptime}{\textsc{Ptime}}
\newcommand{\dlo}{\mathbb{D}}
\newcommand{\subseteqfin}{\subseteq_{\textsc{fin}}}
\newcommand{\supseteqfin}{\supseteq_{\textsc{fin}}}

\newglossaryentry{symb:fin sub}{
 	name=$\subseteqfin$,
 	description={``is a finite subset of''},
 	sort=subseteqfin,
 	type=symbolslist
}
%
\newglossaryentry{symb:fin sup}{
 	name=$\supseteqfin$,
 	description={``is a finite superset of''},
 	sort=supseteqfin,
 	type=symbolslist
}

\newglossaryentry{symb:disun}{
 	name=$\uplus$,
 	description={disjoint union},
 	sort=disun,
 	type=symbolslist
}

\newglossaryentry{symb:num real}{
 	name=$\R$,
 	description={the set of real numbers},
 	sort=numreal,
 	type=symbolslist
}

\newglossaryentry{symb:num real pos}{
 	name=$\R_+$,
 	description={the set of positive real numbers},
 	sort=numrealpos,
 	type=symbolslist
}

\newglossaryentry{symb:num real nonneg}{
 	name=$\R_{\geqslant 0}$,
 	description={the set of non-negative real numbers},
 	sort=numrealnonneg,
 	type=symbolslist
}

\newglossaryentry{symb:num nat}{
 	name=$\N$,
 	description={the set of natural numbers},
 	sort=num nat,
 	type=symbolslist
}

\newglossaryentry{symb:num int}{
 	name=$\Z$,
 	description={the set of integers},
 	sort=numint,
 	type=symbolslist
}

\newglossaryentry{symb:num int pos}{
 	name=$\Z_+$,
 	description={the set of positive integers},
 	sort=numintpos,
 	type=symbolslist
}
\newcommand{\symgr}[1]{\textnormal{\textbf{S}}_{#1}}
\newglossaryentry{symb:symgr}{
 	name=$\symgr{k}$,
 	description={symmetric group of degree $k$},
 	sort=symgr,
 	type=symbolslist
}



\newcommand{\otufrom}[2]{#1^{(#2)}}
%
\newcommand{\iden}{\textnormal{Id}}
\newcommand{\sE}{\mathcal{E}}
%%%%%%%%%%%%%%%%%%%%%%   macros   %%%%%%%%%%%%%%%%%%%%%

\newcommand{\aut}[2][]{\textnormal{Aut}_{#1}(#2)}
\newcommand{\cone}{\textnormal{cone}}
\newcommand{\col}{\textnormal{col}}
% the extra set of brackets around \left( and \right) is to prevent the extra space given before \left(
\newcommand{\flin}[2][]{\textnormal{FinLin}_{#1}{\left(#2\right)}}
\newcommand{\flinWB}[2][]{\textnormal{FinLin}_{#1}{#2}}
\newcommand{\fsum}[1]{\textnormal{FinSum}{\left(#1\right)}}
\newcommand{\lin}[2][]{\textnormal{Lin}_{#1}{\left(#2\right)}}
\newcommand{\row}{\textnormal{row}}
\newcommand{\set}[1]{\left\{#1\right\}}
\newcommand{\setwith}[1]{\textnormal{Set}#1}
%\newcommand{\setof}[2]{\left\{#1 \ \middle| \ #2\right\}}
\newcommand{\setof}[2]{\left\{#1 \ : \ #2\right\}}
\newcommand{\vr}[1]{\mathbf{#1}}
\newcommand{\transpose}[1]{#1^{\mathsf{T}}}
\newcommand{\supp}[1]{\textnormal{support}\left(#1\right)}
\newcommand{\orbit}[2][]{\textnormal{orbit}_{#1}{(#2)}}
\newcommand{\farkas}{Farkas\textquotesingle{ }}
\newcommand{\defeq}{\overset{\mathrm{def}}{=}}
\newcommand{\defiff}{\overset{\mathrm{def}}{\iff}}
\newcommand{\orbits}[2][]{\textnormal{Orbits}_{#1}{(#2)}}
\newcommand{\orbSum}[1]{\Gamma_{#1}}
\newcommand{\orbsum}[1]{\gamma_{#1}}
\newcommand{\orbdis}[2]{\delta^{#1}_{#2}}
\newcommand{\orbrow}[1]{\delta_{#1}}
\newcommand{\orbsmt}[2]{\gamma^{#1}_{#2}}
\newcommand{\orbres}{\zeta}
\newcommand{\idvec}[1]{\mathbf{1}_{#1}}
\newcommand{\del}[1]{\textnormal{delete}_{#1}}
\newcommand{\otu}[2]{(\A\setminus #1)^{(#2)}}
\newcommand{\otuequiv}[1]{\A^{(#1)}}
\newcommand{\otuquotient}[3]{(\otu{#1}{#2}/_{#3})}
\newcommand{\tuple}[1]{\overline{#1}}
\newcommand{\fadp}{\textsc{FADP}}
\newcommand{\Wlog}{\textsc{WLOG}}
\newcommand{\bit}[1]{\textnormal{bit}(#1)}
\newcommand{\npc}{\textsc{NP}\textnormal{-complete}}

%%%%%%%%%%%% Macros for drawing block matrices %%%%%%

\newcommand{\drawMat}[3]{
\left[
\begin{tabular}{#1}
\color{Mycolor1}#2 \color{black}\\
#3
\end{tabular}
\right]
}
%
%%%%%%%%%%%% Macros for basis %%%%%%
%
\newcommand{\dimm}[2]{#1\textnormal{-dim}{\left(#2\right)}}
\newcommand{\TO}[2]{\textnormal{TO}{\left(#1,#2\right)}}
\newcommand{\odom}[2]{{#1\textnormal{-orbit-dom}{\left(#2\right)}}}
\newcommand{\pair}[4]{\leftindex^{#1}_{#2} O^{#3}_{#4}}
\newcommand{\OaT}{\pair{}{\a}{T}{}}
\newcommand{\ObT}{\pair{}{\b}{T}{}}
\newcommand{\OTa}{\pair{T}{}{}{\a}}
\newcommand{\OTb}{\pair{T}{}{}{\b}}
\newcommand{\OTT}{\pair{T}{}{T}{}}

%
%%%%%%%%%%%% Macros for equations %%%%%%
%
\newcommand{\Span}[2][]{\textsc{Span}_{#1}(#2)} % note the capitalisation
\newcommand{\fSpan}[2][]{\textsc{Fin-Span}_{#1}(#2)}
\newcommand{\fSpanK}[3]{\textsc{Fin-Span}_{#1}^{#2}(#3)}
\newcommand{\np}{\textsc{NP}}
\newcommand{\eq}[1]{\textsc{Eq($#1$)}}
\newcommand{\finEq}[1]{\textsc{Fin-Eq($#1$)}}
\newcommand{\genEq}[1]{\textsc{Gen-Eq($#1$)}}
\newcommand{\equivEq}[1]{\textsc{Support-Preserving-Eq($#1$)}}
\newcommand{\cD}[1]{\mathcal{D}_{#1}}
%
\newcommand{\comp}[2]{#1_#2}
\newcommand{\main}[1]{\widetilde{#1}}
%\newcommand{\ordsolv}{\text{$<$-equivariant solvability}}
\newcommand{\ordprob}[1]{$\textsc{Fin-Eq}_{<}(#1)$}
\newcommand{\ordsolvtext}{order equivariant finitary solvability}
\newcommand{\Ordsolvtext}{Order equivariant finitary solvability}
\newcommand{\ordequiv}{order equivariant}
%
\newcommand{\maindom}[1]{\widetilde{\textnormal{d}}\textnormal{om}(#1)}
\newcommand{\lex}[1][]{<_{\ell\textnormal{ex}}^{#1}}
\newcommand{\lexeq}[1][]{\leqslant_{\ell\textnormal{ex}}^{#1}}
%%%%%%%%%%%% Macros for equivariant subspaces %%%%%%
%
\newcommand{\basis}[2][]{\textsc{Basis}_{#1}{\left(#2\right)}}
\newcommand{\length}[2][]{\textsc{Length}_{#1}{(#2)}}
%
% the brackers around \left( and \right) is to prevent the extra space around them
\newcommand{\dom}[1]{\textnormal{dom}{\left(#1\right)}}
\newcommand{\zerovec}{\vr{0}}
\newcommand{\image}[1]{\textnormal{Im}\left(#1\right)}
\newcommand{\spread}[1]{\widetilde{#1}}
\newcommand{\restr}[1]{\arrowvert_{#1}}
\newcommand{\dimred}[2][]{\textnormal{reduc}^{#1}_{#2}}
%
%%%%%%%%%%%%%%%%%%%%%%   environments   %%%%%%%%%%%%%%%%%%%%%
%
% No indentation after theorems
\makeatletter
\def\@endtheorem{\endtrivlist}
\makeatother

\renewcommand{\qedsymbol}{\ensuremath{\blacksquare}}
\makeatletter
\renewcommand{\paragraph}{%
	\@startsection{paragraph}{4}%
	{\z@}{1ex \@plus 1ex \@minus .2ex}{-1em}%
	{\normalfont\normalsize\bfseries}%
}
\makeatother


\theoremstyle{plain}
\newtheorem{theorem}{Theorem}
\newtheorem{lemma}[theorem]{Lemma}
\newtheorem{assumption}[theorem]{Assumption}
\newtheorem{question}[theorem]{Question}
\newtheorem{claim}{Claim}[theorem]
\newtheorem{observation}[theorem]{Observation}
\newtheorem{corollary}[theorem]{Corollary}
\newtheorem{hypothesis}[theorem]{Hypothesis}
\newtheorem{conjecture}[theorem]{Conjecture}

\Crefname{claim}{Claim}{Claims}

\theoremstyle{definition}
\newtheorem{definition}[theorem]{Definition}
\newtheorem{example}[theorem]{Example}
\AtBeginEnvironment{example}{%
  \pushQED{\qed}
  \renewcommand{\qedsymbol}{$\blacktriangleleft$}%
}
\AtEndEnvironment{example}{\popQED\endexample}
\newtheorem{convention}[theorem]{Convention}
\newtheorem{notation}[theorem]{Notation}
\newtheorem*{problem}{Problem}
\newtheorem{remark}[theorem]{Remark}

\newenvironment{subsecproof}[1]{\subsubsection{#1}\pushQED{\qed}}{\popQED\medskip

}
\newenvironment{secproof}[1]{\subsection{#1}\pushQED{\qed}}{\popQED\medskip

}

\newenvironment{latersecproof}[1]
{%
\renewcommand{\theclaim}{\ref{#1}.\arabic{claim}}%
\begin{secproof}{Proof of \Cref{#1}}%
}
{\end{secproof}}

\newenvironment{latersubsecproof}[1]
{%
	\renewcommand{\theclaim}{\ref{#1}.\arabic{claim}}%
	\begin{subsecproof}{Proof of \Cref{#1}}%
	}
{\end{subsecproof}}

%\Crefname{theorem}{Theorem}{Theorems}
%\Crefname{thmenumi}{Theorem}{Theorems}
%\AtBeginEnvironment{theorem}{%
%    \crefalias{enumi}{thmenumi}%
%    \setlist[enumerate,1]{
%        label={\textit{(\roman*)}},
%        ref={\thetheorem.(\roman*)}
%    }%
%}
%
%\Crefname{lemma}{Lemma}{Lemmas}
%\Crefname{lemmaenumi}{Lemma}{Lemmas}
%\AtBeginEnvironment{lemma}{%
%    \crefalias{enumi}{lemmaenumi}%
%    \setlist[enumerate,1]{
%        label={\textit{(\roman*)}},
%        ref={\thelemma.(\roman*)}
%    }%
%}

% IMPORTANT NOTES
%
% 1. Use \newline instead of \\ inside environment
% 2. Put \label inside the second argument
%
%\newcommand{\prob}[4]
%{\begin{problem}[\textsc{#1}]#2
%     $ $\newline
%     \begin{tabularx}{\textwidth}{l  l}
%       \textbf{Input:} & #3 \\
%       \textbf{Question:} & #4 \\
%     \end{tabularx}
% \end{problem}}

\newcommand{\probBasic}[4]
{
\begin{flalign*}
\quad
\begin{tabular}{l  l}
  \multicolumn{2}{l}{\textsc{#1}}\\
  \textbf{Input:}    & #2 \\
  \textbf{#4} & #3
\end{tabular}
&&
\end{flalign*}
}

\newcommand{\prob}[3]
{
\probBasic{#1}{#2}{#3}{Question:}
}

\newcommand{\probOut}[3]
{
\probBasic{#1}{#2}{#3}{Output:}
}
%
\newcommand{\claimQED}{\square}
\newenvironment{claimproof}[1][\proofname]
{\renewcommand\qedsymbol{$\claimQED$}\proof[#1]}
{\endproof}
%\newenvironment{claimproof}{\paragraph{\textnormal{Proof:}}}{
%\smallskip
%
%\hfill $\square$
%\medskip
%
%}
%%%%%%%%%%%%%%%%%   new macros for inequalities  %%%%%%%%%%%%%%%%

\newcommand{\ineq}[1]{\textsc{Ineq($#1$)}}
\newcommand{\finIneq}[1]{\textsc{Fin-Ineq($#1$)}}
\newcommand{\nNEq}[1]{\textsc{Nonneg-Eq($#1$)}}
\newcommand{\finNNEq}[1]{\textsc{Fin-Nonneg-Eq($#1$)}}
\newcommand{\polyineq}{\textsc{Poly-Ineq}}
\newcommand{\allpolyopt}{\textsc{All-Poly-Opt}}
\newcommand{\allpolyineq}{\textsc{All-Poly-Ineq}}

\newcommand{\headPar}[2]{\textnormal{\textsc{head}}_{#2}(#1)}
\newcommand{\head}[1]{\headPar{#1}{}}
\newcommand{\strictHead}[1]{\headPar{#1}{>}}
\newcommand{\tail}[1]{\textsc{tail}(#1)}
\newcommand{\ineqSize}[1]{|#1|}
\newcommand{\G}{\Gamma}
\newcommand{\conj}{\cup}
\newcommand{\ineqal}{\mathcal{E}}
\newcommand{\aasol}{almost-all-solution}
\newcommand{\proj}[1]{\Pi_{#1}}
\newcommand{\orbval}[1]{\dot{#1}}
\newcommand{\exptime}{\textsc{ExpTime}}
\newcommand{\nexptime}{\textsc{NExpTime}}
\newcommand{\twoExptime}{\textsc{2ExpTime}}
\newcommand{\acker}{\textsc{Ackermann}}

%%%%%%%%%%%%%%%%%%%%%%   new macros for LP  %%%%%%%%%%%%%%%%%%%%%

\newcommand{\linProg}{\textsc{Lin-Prog}}

\newcommand{\flinProg}{\textsc{FinLin-Prog}}

\newcommand{\lpMax}[2]
{
\begin{tabular}{l l}
\textnormal{maximise}   & #1\\
\textnormal{subject to} & #2
\end{tabular}
}

\newcommand{\lpMaxMin}[2]
{
\begin{tabular}{r l}
\textnormal{maximise/minimise}   & #1\\
\textnormal{subject to} & #2
\end{tabular}
}


\newcommand{\lpMin}[2]
{
\begin{tabular}{l l}
\textnormal{minimise}   & #1\\
\textnormal{subject to} & #2
\end{tabular}
}

%\lpMaxGen{A}{b}{c}{x}{C}
\newcommand{\lpMaxGen}[5]{
\lpMax{$\transpose{#3}\cdot #4$}{
  $#1 \cdot #4 \leqslant #2$ \\
& $#4\in\lin{#5}$
}}

%\lpMinGen{A}{b}{c}{x}{C}
\newcommand{\lpMinGen}[5]{
\lpMin{$\transpose{#3}\cdot #4$}{
  $#1 \cdot #4 \geqslant #2$ \\
& $#4\in\lin{#5}$ 
}}

\newcommand{\maxGenAbc}{
\lpMaxGen{\vr{A}}{\vr{b}}{\vr{c}}{\vr{x}}{C}
}
%
\newcommand{\minGenAbc}{
\lpMinGen{\vr{A}}{\vr{b}}{\vr{c}}{\vr{x}}{C}
}

\newcommand{\lpMaxFinGen}[5]{
\lpMax{$\transpose{#3}\cdot #4$}{
  $#1 \cdot #4 \leqslant #2$ \\
& $#4\in\flin{#5}$ 
}
}

\newcommand{\lpMinFinGen}[5]{
\lpMin{$\transpose{#3}\cdot #4$}{
  $#1 \cdot #4 \geqslant #2$ \\
& $#4\in\flin{#5}$ 
}
}

\newcommand{\maxFinGenAbc}{
\lpMaxFinGen{\vr{A}}{\vr{b}}{\vr{c}}{\vr{x}}{C}
}

\newcommand{\minFinGenAbc}{
\lpMinFinGen{\vr{A}}{\vr{b}}{\vr{c}}{\vr{x}}{C}
}


%\lpMaxFin{A}{b}{c}{x}{B}{C}
\newcommand{\lpMaxFin}[6]{
\lpMax{$\transpose{#3}\cdot #4$}{
  $#1 \cdot #4 \leqslant #2$ \\
& $#4 \geqslant \vr{0}$ \\
& $#4\in\flin{#6}$ 
}}

%\lpMinFinStan{A}{b}{c}{x}{B}{C}
\newcommand{\lpMinFin}[6]{
\lpMin{$\transpose{#3}\cdot #4$}{
  $#1 \cdot #4 \geqslant #2$ \\
& $#4 \geqslant \vr{0}$ \\
& $#4\in\flin{#6}$ 
}}

\newcommand{\maxFinAbc}{
\lpMaxFin{\vr{A}}{\vr{b}}{\vr{c}}{\vr{x}}{B}{C}
}

\newcommand{\minFinAbc}{
\lpMinFin{\vr{A}}{\vr{b}}{\vr{c}}{\vr{x}}{B}{C}
}

%\lpMatPrimal{A}{b}{c}{x}{B}{C}
\newcommand{\lpMatPrimal}[6]{
\lpMax{$\transpose{#3}\cdot #4$}{
  $#1 \cdot #4 \leqslant #2$ \\
& $#4 \geqslant \vr{0}$ \\
& $#4\in\lin{#6}$ 
}}

%\lpMinPrimal{A}{b}{c}{x}{B}{C}
\newcommand{\lpMinPrimal}[6]{
\lpMin{$\transpose{#3}\cdot #4$}{
  $#1 \cdot #4 \geqslant #2$ \\
& $#4 \geqslant \vr{0}$ \\
& $#4\in\lin{#6}$ 
}}

%\lpMatDual{A}{b}{c}{x}{B}{C}
\newcommand{\lpMatDual}[6]{
\lpMin{$\transpose{#2}\cdot #4$}{
  $\transpose{#1} \cdot #4 \geqslant #3$ \\
& $#4 \geqslant \vr{0}$ \\
& $#4\in\lin{#5}$ 
}}

\newcommand{\primalAbc}{\lpMatPrimal{\vr{A}}{\vr{b}}{\vr{c}}{\vr{x}}{B}{C}}
\newcommand{\minPrimalAbc}{\lpMinPrimal{\vr{A}}{\vr{b}}{\vr{c}}{\vr{x}}{B}{C}}
\newcommand{\dualAbc}{\lpMatDual{\vr{A}}{\vr{b}}{\vr{c}}{\vr{y}}{B}{C}}

%%%%%%%%%%%%%%   macros for column(row)-finite linear programs  %%%%%%%%%%%%%%%

\newcommand{\colLinProg}{\textsc{Col-Fin-Lin-Prog}}

\newcommand{\rowLinProg}{\textsc{Row-Fin-Lin-Prog}}

\newcommand{\colFinMax}[3]{
\begin{tabular}{l l}
\textnormal{Find a vector}  & #1 \\
\textnormal{that maximises} & #2 \\
\textnormal{subject to}     & #3
\end{tabular}
}

\newcommand{\colFinMaxAbc}{
\primalAbc
}

\newcommand{\colFinMinAbc}{
\lpMin{$\transpose{\vr{c}}\cdot \vr{x}$}{
  $\vr{A} \cdot \vr{x} \geqslant \vr{b}$ \\
& $\vr{x}\geqslant\vr{0}$                     \\  
& $\vr{x}\in\lin{C}$
}
}

\newcommand{\rowFinMinAbc}{
\dualAbc 
}

\newcommand{\rowFinMaxAbc}{
\lpMax{$\transpose{\vr{b}}\cdot \vr{y}$}{
  $\transpose{\vr{A}} \cdot \vr{y} \leqslant \vr{c}$ \\
& $\vr{y} \geqslant \vr{0}$ \\
& $\vr{y} : B\to\R$ 
}}

%%%%%%%%%%%%%%%%%%%%%%   comments   %%%%%%%%%%%%%%%%%%%%%

\newcounter{comment}
\setcounter{comment}{1}
\newcommand{\arka}[1]{\noindent\textnormal{\textcolor{red}{\textbf{Arka[\thecomment]:} #1}}\addtocounter{comment}{1}\noindent}
\newcommand{\slawek}[1]{\textnormal{\textcolor{magenta}{\textbf{S\l awek:} #1}}}
\newcommand{\piotrek}[1]{\textnormal{\textcolor{blue}{\textbf{Piotrek[\thecomment]:} #1}}\addtocounter{comment}{1}\noindent}

%%%%%%%%%%%%%%%%%%% For underlining %%%%%%%%%%%%%%%%%%%%%%%%%%%%
%\newcommand{\fixed}[1]{\setulcolor{violet}\ul{#1}\setulcolor{black}}
%\newcommand{\fixed}[1]{\setulcolor{violet}\ul{#1}\setulcolor{black}}
%\newcommand{\fixed}[1]{\textbf{#1}}
\colorlet{fixedcolor}{blue!70!black}
\newcommand{\fixed}[1]{\textcolor{fixedcolor}{#1}}
%%%%%%%%%%%%%%%% Alignment command(s) %%%%%%%%%%%%%%%

% The code below is to make the text left aligned,
% ensuring indententation of paragraphs
\makeatletter
\newcommand\iraggedright{%
  \let\\\@centercr\@rightskip\@flushglue \rightskip\@rightskip
  \leftskip\z@skip}
\makeatother
%
