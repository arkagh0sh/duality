%!TEX root = main.tex
%
\section{Homogenous structures}\label{sec:hom}
%
A \emphdef{structure} $\str = (X,R_1,R_2,\dots)$ is given by a countable set $X$ and relations $R_1,R_2,\dots$ defined on it.
\arka{should we explain the countability assumption?}
For example,
we can consider the structure \emphdef{$\dloQ = (Q,<)$} where $Q$ is the set of rational numbers and $<$ is the usual ordering.
We can also consider the structure \emphdef{$\A = (A,=)$} where $A$ is an countably infinite set and $=$ is the equality relation on $A$.

\fixed{From now on we use $\str = (X,R_1,R_2,\dots)$ to denote an arbitrary structure.}
We use $\str^n$ to denote the set $X^n$.
An \emphdef{automorphism} $\pi$ of $\str$ is a bijection of $\str$ which preserves and reflects the relations,
i.e.\ for every relation $R_i$ of arity $n_i$ (say), and $a_1,\dots,a_n\in \X$:
\[
R_i(a_1,\dots,a_{n_i}) \iff R_i(\pi(a_1),\dots,
\pi(a_{n_i})) \ .
\]
The set of all of all automorphisms of $\str$ is denoted as \emphdef{$\aut{\str}$}
%
\begin{example}
\arka{Illustrate automorphisms}
\end{example}
%
A subset $S\subseteq \str$ induces the \emphdef{substructure}
\[
\emphdef{$\subst{S}{\str}$} = 
(S,\subst{R_1}{S},\subst{R_2}{S},\dots) \ ,
\]
where \emphdef{$\subst{R_n}{S}$} is the restriction of $R_n$ to $S$.
If $S$ is finite, then $\subst{S}{\str}$ is a \emphdef{finite substructure}.
For $S,T\subseteq \str$,
a bijection $f : S\to T$ is called an \emphdef{isomorphism between substructures} $\subst{S}{\str}$ and $\subst{T}{\str}$ if it preserves and reflects all the relations,
i.e.\ for every relation $R_i$ of arity $n_i$ (say), and $a_1,\dots,a_n\in S$:
\[
R_i(a_1,\dots,a_{n_i}) \iff R_i(f(a_1),\dots,
f(a_{n_i})) \ .
\]

The structure $\str$ is called \emphdef{homogenous} if every partial isomorphisms between finite substructures of $\str$ extends to an automorphism of $\str$,
i.e.\ for every partial isomorphism $f : S\to T$ of substructures there exists an automorphism $\pi$ of $\str$ such that $\pi\restr{S} = f$.
%
\begin{example}
\arka{Illustrate substructures and homogeneity}
\end{example}
%
\begin{example}\label{eg:int atom}
\arka{Define integer atoms/domain $\Z = (Z,<)$ to give example of a structure that is not homogenous}
\end{example}
%
\begin{definition}\label{def:oligo}
\arka{Write definition of oligomorphicity.}
\end{definition}
%
\begin{remark}
Homogenous structures with finitely many relations are also oligomorphic \arka{add citation}.
However, the same is not true when there are infinitely many relations.
For example, one can define the structure $\Z' = (Z,D_1,D_2,\dots)$ where $D_n(a,b)$ is true if and only if $x - y = n$.
The structure $\Z'$ is homogenous.
However, it is not oligomorphic as each of the relations $D_n \subseteq Z^2$ forms disjoint orbit
\end{remark}
%
Throughout the article we use $\str = (X,R_1,R_2,\dots)$ to denote an arbitrary countable homogenous relational structure whose automorphism group is oligomorphic.
We use $\str^n$ to denote the set $X^n$.
%
\begin{definition}\label{def:fin supp orb}
\arka{Write definition of finitely supported orbits, and of orbit-finiteness}
\end{definition}
%
\begin{remark}
\arka{orbit-finite subsets of $\str^n$ are exactly the definable ones}
\end{remark}
%
\begin{theorem}
If $\str$ is oligomorphic then $\str^n$ is orbit-finite under the action of $\aut[S]{\str}$ for every $S\subseteqfin \str$.
\end{theorem}
%
\begin{definition}\label{def:ord prop}
The structure $\str$ is said to have the \emphdef{order property} if there is an orbit-finite relation $R\subseteq (\str^n \times \str^n)$ and an infinite subset $D \subseteq \str^n$ such that $R$ defines a linear order on $D$.
\end{definition}
%
\arka{Possibly the relation $R$ can even be assumed equivariant. We should check.}
%
\begin{example}
\arka{Illustrate \Cref{def:ord prop}}.

\arka{Example where $n\neq 1$?}
\end{example}
%
For a finite subset $S\subseteq \str$ we use \emphdef{$\subst{S}{\str}$} to denote the relational structure $(S,R_1,R_2,\dots)$.
The notions of homogeneity, automorphism group and orbits extend to $\subst{S}{\str}$ as well.
%
\begin{definition}\label{def:smooth}
The structure $\str$ is said to be \emphdef{smoothly approximated} if there is a chain $S_0 \subseteq S_1 \subseteq S_2 \subseteq \dots$ of subsets of $\str$ such that $\bigcup_{n\in\N} S_n = X$ and $\subst{S_n}{\str}$ is homogenous for every $n$.
\end{definition}  
%
\arka{add citation to above def}
%
\begin{example}
\arka{Write which of the structures mentioned before are smoothly approximated and which are not.}
\end{example}
%
\begin{example}
\arka{Bit vector atoms}
\end{example}
%
\begin{example}
\arka{Nestings of smoothly approximated structures are also smoothly approximated.}
\end{example}
%
\begin{theorem}\label{thm:smooth iff}
The structure $\str$ is smoothly approximated if and only if it does not have the order property.
\end{theorem}
%
\begin{proof}
\arka{TODO}
\end{proof}
%
For $S\subseteqfin \str$, let $\str_S$ be the structure where elements of $S$ are declared as constants.
Formally, for every $x\in S$, we add an unary relation $U_x$ to the signature of $\str$ which is satisfied only by $x$.
Then $\aut{\str_S} = \aut[S]{\str}$.
%
\begin{theorem}\label{thm:equiv wlog}
The automorphism group of $\str$ is oligomorphic if and only if the automorphism group of $\str_S$ is oligomorphic for any(every) $S\subseteqfin \str$.

The structure $\str$ is smoothly approximated if and only if $\str_S$ is smoothly approximated for any(every) $S\subseteqfin \str$.
\end{theorem}
%
\begin{proof}
\arka{TODO}
\end{proof}
%
\arka{The above theorem can allow us to assume all systems to be equivariant WLOG}
%