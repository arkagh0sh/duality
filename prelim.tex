%!TEX root = main.tex
%
\section{Homogenous structures}\label{sec:hom}
%
\begin{definition}\label{def:rel str}
\arka{Write definition of a countable relational structure.}
\end{definition}
%
\begin{example}\label{eg:eq atom}
\arka{Define equality atoms/domain $\A$.}
\end{example}
%
\begin{example}\label{eg:ord atom}
\arka{Define ordered atoms/domain $\dloQ$}
\end{example}
%
\begin{definition}\label{def:aut}
\arka{Write definition of automorphism group and orbits}
\end{definition}
%
\begin{example}
\arka{Illustrate \Cref{def:aut}}
\end{example}
%
\begin{definition}\label{def:homg}
\arka{Write definition of homogenous relational structure.}
\end{definition}
%
\begin{example}\label{eg:int atom}
\arka{Define integer atoms/domain $\Z = (Z,<)$ to give example of a structure that is not homogenous}
\end{example}
%
\begin{definition}\label{def:oligo}
\arka{Write definition of oligomorphicity.}
\end{definition}
%
\begin{remark}
Homogenous structures with finitely many relations are also oligomorphic \arka{add citation}.
However, the same is not true when there are infinitely many relations.
For example, one can define the structure $\Z' = (Z,D_1,D_2,\dots)$ where $D_n(a,b)$ is true if and only if $x - y = n$.
The structure $\Z'$ is homogenous.
However, it is not oligomorphic as each of the relations $D_n \subseteq Z^2$ forms disjoint orbit
\end{remark}
%
Throughout the article we use $\str = (X,R_1,R_2,\dots)$ to denote an arbitrary countable homogenous relational structure whose automorphism group is oligomorphic.
We use $\str^n$ to denote the set $X^n$.
%
\begin{definition}\label{def:fin supp orb}
\arka{Write definition of finitely supported orbits, and of orbit-finiteness}
\end{definition}
%
\begin{remark}
\arka{orbit-finite subsets of $\str^n$ are exactly the definable ones}
\end{remark}
%
\begin{theorem}
If $\str$ is oligomorphic then $\str^n$ is orbit-finite under the action of $\aut[S]{\str}$ for every $S\subseteqfin \str$.
\end{theorem}
%
\begin{definition}\label{def:ord prop}
The structure $\str$ is said to have the \emphdef{order property} if there is an orbit-finite relation $R\subseteq (\str^n \times \str^n)$ and an infinite subset $D \subseteq \str^n$ such that $R$ defines a linear order on $D$.
\end{definition}
%
\begin{example}
\arka{Illustrate \Cref{def:ord prop}}.

\arka{Example where $n\neq 1$?}
\end{example}
%
For a finite subset $S\subseteq \str$ we use \emphdef{$\subst{S}{\str}$} to denote the relational structure $(S,R_1,R_2,\dots)$.
The notions of homogeneity, automorphism group and orbits extend to $\subst{S}{\str}$ as well.
%
\begin{definition}\label{def:smooth}
The structure $\str$ is said to be \emphdef{smoothly approximated} if there is a chain $S_0 \subseteq S_1 \subseteq S_2 \subseteq \dots$ of subsets of $\str$ such that $\bigcup_{n\in\N} S_n = X$ and $\subst{S_n}{\str}$ is homogenous for every $n$.
\end{definition}
%
\arka{add citation to above def}
%
\begin{example}
\arka{Write which of the structures mentioned before are smoothly approximated and which are not.}
\end{example}
%
\begin{example}
\arka{Bit vector atoms}
\end{example}
%
\begin{example}
\arka{Nestings of smoothly approximated structures are also smoothly approximated.}
\end{example}
%
\begin{theorem}\label{thm:smooth iff}
The structure $\str$ is smoothly approximated if and only if it does not have the order property.
\end{theorem}
%
\begin{proof}
\arka{TODO}
\end{proof}
%